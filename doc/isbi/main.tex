% Template for ISBI-2011 paper; to be used with:
%          spconf.sty  - ICASSP/ICIP LaTeX style file, and
%          IEEEbib.bst - IEEE bibliography style file.
% --------------------------------------------------------------------------
\documentclass{article}
\usepackage{spconf,amsmath,graphicx}
\graphicspath{{pdf/}{png/}}
\usepackage[printonlyused,nolist]{acronym}
\usepackage[colorlinks=true,linkcolor=black,citecolor=black]{hyperref}
\usepackage{colortbl}

% Example definitions.
% --------------------
\def\x{{\mathbf x}}
\def\L{{\cal L}}

% Title.
% ------
\title{An Analysis of Blood-Oxygen-Level-Dependent Signal Parameter Estimation Using Particle Filters}
%
% Single address.
% ---------------
\name{M. Chambers, C. Wyatt}
\address{Virginia Tech\\
    Bradley Department of Electrical and Computer Engineering\\
    302 Whittemore Hall\\
    Blacksburg, VA 24061-0111}
%
% For example:
% ------------
%\address{School\\
%	Department\\
%	Address}
%
% Two addresses (uncomment and modify for two-address case).
% ----------------------------------------------------------
%\twoauthors
%  {A. Author-one, B. Author-two\sthanks{Thanks to XYZ agency for funding.}}
%	{School A-B\\
%	Department A-B\\
%	Address A-B}
%  {C. Author-three, D. Author-four\sthanks{The fourth author performed the work
%	while at ...}}
%	{School C-D\\
%	Department C-D\\
%	Address C-D}
%
% More than two addresses
% -----------------------
% \name{Author Name$^{\star \dagger}$ \qquad Author Name$^{\star}$ \qquad Author Name$^{\dagger}$}
%
% \address{$^{\star}$ Affiliation Number One \\
%     $^{\dagger}$}Affiliation Number Two
%


\begin{document}
%\ninept
%
\maketitle
%
\begin{abstract}
The Blood-Oxygen-Level-Dependent (BOLD) signal that is measured by functional 
magnetic resonance imaging (fMRI)
has been the subject of extensive research since the first development 
of the balloon model. While there are definite benefits to 
moving from the Canonical Hemodynamic Response function to a physiologically inspired
BOLD model, significant barriers remain. Optimizing even the simplest
balloon model requires searching within 7 dimensions, and even more complex
models exist. Whereas traditional methods of analyzing fMRI aims
to determine where activation occurs, BOLD models seek a parametric representation
of the signal. Unfortunately, the nonlinear nature of these models
makes it difficult to analyze, therefore this work demonstrates the 
use of a particle filter to regresses the simplest form of the BOLD 
model. The results show that the system of equations are not observable,
leading to a large range of parameters that are consistent with the measurements. 
\end{abstract}
%
\begin{keywords}
BOLD Response, Functional MRI, Nonlinear Systems, Particle Filter, Bayesian Statistics, System Identification
\end{keywords}
%

\begin{acronym}[CMRO2]
\acro{BOLD}{Blood-Oxygen-Level-Dependent}
\acro{CBF}{Cerebral Blood Flow}
\acro{CBV}{Cerebral Blood Volume}
\acro{CMRO2}{Cerebral Metabolic Rate of Oxygen}
\acro{DC}{Direct Current}
\acro{dHb}{Deoxygenated Hemoglobin}
\acro{EM}{Expectation-Maximization}
\acro{EPI}{Echo Planar Imaging}
\acro{fMRI}{Functional Magnetic Resonance Imaging}
\acro{FMRIB}{Oxford Centre for Functional MRI of the Brain}
\acro{FSL}{FMRIB Software Library}
\acro{FWHM}{Full-Width Half-Maximum}
\acro{GA}{Genetic Algorithms}
\acro{GLM}{General Linear Model}
\acro{HRF}{Hemodynamic Response Function}
\acro{Hb}{Hemoglobin}
\acro{MAD}{Median Absolute Deviation}
\acro{MI}{Mutual Information}
\acro{MR}{Magnetic Resonance}
\acro{MRI}{Magnetic Resonance Imaging}
\acro{MSE}{Mean Squared Error}
\acro{O2Hb}{Oxygenated Hemoglobin}
\acro{ODE}{Ordinary Differential Equation}
\acro{PDF}{Probability Density Function}
\acro{POSSUM}{Physics-Oriented Simulated Scanner for Understanding MRI}
\acro{RF}{Radio Frequency}
\acro{RMSE}{Root Mean Squared Error}
\acro{RMSR}{Root Mean Squared Residual}
\acro{SA}{Simulated Annealing}
\acro{SNR}{Signal-to-Noise Ratio}
\acro{SPM}{Statistical Parametric Mapping}
\acro{T1}{Longitudinal}
\acro{T2}{Spin-Spin}
\acro{TR}{Repetition Time}
\acro{UKF}{Unscented Kalman Filter}
\end{acronym}
\section{Introduction}
\label{sec:intro}
Traditional methods of analyzing 
\ac{fMRI} images perform regression using the linear combination of explanatory variables. 
Though moving to a state space model with more more degrees of freedom 
requires additional computation, in this paper Particle Filters will
be used to estimate the governing parameters of the \ac{BOLD} model 
at a computation cost 
that would still allow real time calculations for multiple voxels.
The parameters guiding the BOLD model could provide insight into 
functional differences between patients and brain regions.
More practically, this algorithm is an alternative to the General
Linear Model for activation detection. Though computation intensive,
this method is capable of modeling nonlinear effects
and provides more detailed output. 

This work focuses
on the observability of the \ac{BOLD} state equations from Friston et al.
\cite{Friston2000}. 
Because of the nonlinearities in the state equations,
no analytical method exists to determine observability. Therefore
Particle Filters are used to build a joint mixture \ac{PDF} of the model parameters. 
This work will show that this method is sufficient for predicting
the BOLD signal, without requiring excessive computation. 
Because particle filters estimate a full posterior distribution, 
it was possible to calculate correlation of model parameters, an
indicator of observability. 

\subsection{BOLD Model}
It is well known that the two types of \ac{Hb} act as contrast agents in 
\ac{EPI} imaging \cite{Buxton1998, WEISSKOFF1994, Ogawa}. Despite this, the connection
between \ac{dHb}/\ac{O2Hb} and neural activity is non-trivial. 
Intuitively, increased 
metabolism will increase \ac{dHb}, however blood vessels are quick
to compensate by permitting increased local blood flow. Increased inflow, accomplished by loosening 
capillary beds, precedes a matching outflow, driving increased 
blood volume.
Since the local \ac{MR} signal depends on the ratio of \ac{dHb} to \ac{O2Hb},
increased blood volume also affects this ratio. 
This was the impetus
for the ground breaking balloon model \cite{Buxton1998} and windkessel
model \cite{Mandeville1999}. These works derive 
the changes in \ac{dHb} ratio and volume of capillaries given \ac{CBF}
and \ac{CMRO2} waveforms.

Although Buxton et al. demonstrated that a well chosen flow waveform could 
explain most features of the \ac{BOLD} signal, it stopped short of proposing a
deterministic waveform for the \ac{CBF} and \ac{CMRO2} \cite{Buxton1998}. Friston et al. 
gave a simple expression for \ac{CBF} input based on a flow inducing signal,
and proposed that metabolic rate should be a linear function of the \ac{CBF} \cite{Friston2000}.
By combining these equations with the balloon model from Buxton et al.,
it is possible to predict the \ac{BOLD} signal from a stimulus function:
\begin{eqnarray}
\label{eq:bold1}
\dot{s} &=& \epsilon u(t) - \frac{s}{\tau_s} - \frac{f - 1}{\tau_f} \\
\label{eq:bold2}
\dot{f} &=& s\\
\label{eq:bold3}
\dot{v} &=& \frac{1}{\tau_0}(f - v^\alpha)\\
\label{eq:bold4}
\dot{q} &=& \frac{1}{\tau_0}(\frac{f(1-(1-E_0)^f)}{E_0} - \frac{q}{v^{1-1/\alpha}})
\end{eqnarray}
where $s$ is a flow inducing signal, $f$ is the input \ac{CBF},
$v$ is normalized \ac{CBV}, and $q$ is the normalized
local \ac{dHb} content. The 
parameters controlling blood flow are $\epsilon$, which is a neuronal 
efficiency term, $u(t)$ which is the stimulus, and $\tau_f$, $\tau_s$ 
which are time constants. The parameters for the evolution of blood 
volume are $E_0$ which is the resting metabolic
rate and $\alpha$ which is Grubb's parameter controlling the balloon model. 
$\tau_0$ is a single time constant controlling the speed of $v$ and $q$.

Obata refined the readout equation 
of the \ac{BOLD} signal based on the
\ac{dHb} content (q) and local blood volume (v), resulting in the
final \ac{BOLD} measurement \autoref{eq:boldout} \cite{Obata2004}.
\begin{eqnarray}
\label{eq:boldout}
y   &=& V_0((k_1 + k_2)(1-q) - (k_2 + k_3)(1-v))\\
k_1 &=& 4.3 \times \nu_0 \times E_0 \times TE = 2.8 \nonumber\\
k_2 &=& \epsilon_0 \times r_0 \times E_0 \times TE = .57 \nonumber \\
k_3 &=& \epsilon_0 - 1 = .43 \nonumber
\end{eqnarray}
Where $\nu_0 = 40.3 s^{-1}$  is the frequency offset in Hz for fully
de-oxygenated blood (at 1.5T), $r_0 = 25 s^{-1}$  is the slope relating
change in relaxation rate with change in blood oxygenation, $TE$ is the
echo time used by the \ac{EPI} sequence and $\epsilon_0 = 1.43$ is the 
ratio of signal \ac{MR} from intravascular to extravascular regions at rest. 
In Deneux et al. \cite{Deneux2006} it was found that the parameters are far from perpendicular,
and that very different parameters could give nearly identical \ac{BOLD} output;
this work builds on that by calculating a joint distribution for the
model parameters of this version of the model. 

There have been many previous attempts to learn the parameters of the
balloon model, although none enjoy the ubiquity of the \ac{GLM}. 
Friston et al. proposed a novel combination of 
linear and nonlinear modeling to generate parameter estimates
\cite{Friston2002b}.  Thus Friston et al. approximated the Jacobian, 
$\frac{\partial Y}{\partial \theta}$, by approximating the response
using a low-order Volterra Kernel \cite{Friston2002b}. 
Unfortunately, estimating a Volterra kernel
takes longer than simply integrating the state variables. Thus,
this method was simplified to a generic nonlinear-least squares algorithm.

In Johnston et al. \cite{Johnston2007}, a hybrid particle filter/gradient
descent algorithm was used to simultaneously derive the static and dynamic 
parameters, (more generally known as parameters and state variables, respectively).
A particle filter was used to calculate the time course of the state variables.
The estimated distribution of the state variables was then used in
a Maximum Likelihood fashion to find the most likely set of parameters. 
This process was repeated until the parameters converged. This
method is  more complex than using the particle
filter to calculate the parameters, takes longer to run and
does not account for uncertainty in parameters.

Hu et al. used an Unscented Kalman Filter to simultaneously calculate
the model parameters and state variables \cite{Hu2009}.
However, because the Kalman Filter assumes a multivariate 
Gaussian for the state variables, $X(t-1)$, the posterior 
distribution is limited. Additionally, a nonlinear transformation of 
a Gaussian, 
such as the state transition function of the Balloon Model,
can be significantly non-Gaussian. 

In Vakorin et al., a combination of Genetic Algorithms and 
Simulated Annealing was used to estimate not only the parameters, but the true 
stimuli driving the \ac{BOLD} signal  \cite{Vakorin2007}. 
This addresses the inherent uncertainty of exactly when 
stimuli get applied. Unfortunately this algorithm took
more than 16 hours per voxel to run.

In his PhD thesis, Murray \cite{Murray2008} used a particle filter based 
approach to integrate
the \ac{BOLD} equations. The method used in that work focused primarily on estimating
the \ac{BOLD} output and state equations as a nonlinear stochastic differential 
equation. The primary difference between that work and this is that
Murray \cite{Murray2008} took the parameters as a given. Thus, differences in the \ac{BOLD} output
were assumed to be driven by noise in the underlying state
equations. Because the parameters were held constant, the 
\ac{BOLD} estimate was poor. However, the primary purpose of that work
was to advance understanding of filtering techniques; and the filtering framework created
in Murray \cite{Murray2008}, dysii, forms the basis for the particle 
filter used in this work.

\section{Methods}
\label{sec:Methods}
\subsection{Particle Filter}
\label{sec:ParticleFilter}
Particle filters, a type of Sequential Monte Carlo (SMC) method,
are a powerful method for on line estimation of the posterior probability 
distribution of parameters given a timeseries and a model. The concept of 
particle filters is similar to that of the Unscented Kalman Filter; however,
distributions are stored as an empirical distribution rather than 
as the first two moments of a Gaussian. Thus, particle filters are 
preferred when the model is nonlinear, and non-Gaussian. 
The particle filters are a well established technique and are 
described in great detail in Arulampalam et al., Thrun et al. 
and Murray \cite{Arulampalam2002a, thrun2008probabilistic, Murray2008}.
\begin{table}[t]
\ninept
  \centering
\begin{tabular}{|c | c  c  c  c  c  c  |}
\hline
  & $\tau_0$ & $\alpha$ & $E_0$    & $V_0$    & $\tau_s$ & $\tau_f$  \\
\hline
$\alpha$                      & 0.880& & & & & \\
\rowcolor[gray]{.8} $E_0$     & -0.766& -0.523& & & & \\
$V_0$                         & 0.624& 0.424& -0.796& & & \\
\rowcolor[gray]{.8} $\tau_s$  & 0.620& 0.295& -0.748& 0.344& & \\
$\tau_f$                      & 0.000& -0.397& -0.431& 0.196& 0.699& \\
\rowcolor[gray]{.8} $\epsilon$& 0.616& 0.656& -0.641& 0.285& 0.446& -0.097\\
\hline
\end{tabular}
  \caption{Correlation Matrix of Posterior Distribution for 40 minute simulated BOLD signal.}
\label{tab:long_corr}
\end{table}
The particle filter was set with each particle representing one possible 
system realization: $\{\tau_0, \alpha, E_0, V_0, \tau_s, \tau_f,
\epsilon, s, f, v, q\}$. Initially 16,000 particles were drawn
from the prior, however after resampling the number was dropped
to 1000. This gave the prior sufficient density to
cover parameter space, but saved computation time once the solution space
became more compact. 
The parameters were drawn independently from the gamma distribution:
$\tau_0 \sim \Gamma(.98, .25)$, 
$\alpha \sim \Gamma(.33, .045)$, $E_0    \sim \Gamma(.34, .03)$,
$V_0    \sim \Gamma(.04, .03)$, $\tau_s \sim \Gamma(1.54, .25)$,
$\tau_f \sim \Gamma(2.46, .25)$, $\epsilon \sim \Gamma(.7, .6)$.
To reduce the size of the search space from 11 to 7, $\{s, f, v, q\}$
were assumed to start at resting state, $\{0, 1, 1, 1\}$. Although
it is common to add noise to the state during integration (to 
emulate noise in the state updates), this was not performed because 
regularization during resampling tended to expand the distribution.
Additionally the cloud of particles is able to account for some amount
of stationary randomness.
Initially each particle is weighted equally, however
as new measurements arrive, they alter the probability
of particles being a solution.
New measurements
are incorporated into the weights by \autoref{eq:weightevolve}:
\begin{equation}
w^i_k & \propto & w^i_{k-1}P(y_k| x^i_k) 
\label{eq:weightevolve}
\end{equation}
where $i$ is the particle index, and $k$ is the time index. $P(y_k | x^i_k)$
is the probability of the observed measurement coming from the $i^{th}$ 
particle and depends on the measurement noise and the observation function, 
\autoref{eq:boldout}.
$x^i_k$ is calculated from $x^i_{k-1}$ using the balloon model, 
\autoref{eq:bold1} through \autoref{eq:bold4}.
For all the analysis  in this work, $1400$ Euler integration points
per TR ($2.1$ s) were used. When the number of particles with significant
(non-zero) weights dropped below 50, regularized resampling was 
applied. Thus, a new set of particles was drawn from the smoothed
version of the mixture \ac{PDF}.

Because fMRI drift can add structure to the noise, a spline with 1 knot every
20 measurements was fitted to and then subtracted from the timeseries
\cite{Smith1999, Tanabe2002}. The level was then converted
to \% difference from the original mean. Since the BOLD signal is 
predominantly positive, a constant (the Gaussian normed \acl{MAD} 
of the \% difference signal) was added to each point. 

Given the state-space equations for the \ac{BOLD} signal, simulating a single time
series is straightforward. After generating a true signal,
identically and independently distributed (I.I.D.) Gaussian noise and a Wiener
process with Gaussian I.I.D. steps were added to the true signal. Finally a
carrier level was added, since \ac{BOLD} is 
measured as a \% difference from the baseline. For single voxel
analysis, the noise was kept relatively low ($\sigma = 0.001$  for
I.I.D. Gaussian Noise, and $\sigma = .0005$ for Wiener Steps) to
explore the properties of the model. This noise was added
to the the BOLD signal typically peaks at $0.01-0.03$ (1-3\%). 
To simulate a slice with \ac{POSSUM} \cite{Drobnjak}, it's code was modified to
output activation levels based on sets of parameters rather than
activation levels. For the POSSUM simulated data, SNR was significantly 
lower although the noise did not include drift ($<6dB$).

\section{Results}
\label{sec:Results}
A primary advantage of the particle filter is that it estimates
a complete joint distribution of the parameters and states. This 
allows for more detailed analyses not available to many other methods.
In particular it is possible to calculate the correlation of the
parameters in the posterior distribution. The correlation matrix
for the single 40 minute simulation is shown in \autoref{tab:long_corr}.
\begin{figure}
\begin{minipage}[b]{.5\linewidth}
  \centering
  \centerline{\includegraphics[width=\textwidth]{snr_hm.png}}
  \centerline{(a) SNR}\medskip
\end{minipage}
\hfill
\begin{minipage}[b]{.49\linewidth}
  \centering
  \centerline{\includegraphics[width=\textwidth]{sim_hm_mi.png}}
  \centerline{(b) Mutual Information}\medskip
\end{minipage}
\caption{Results of POSSUM simulated data. $a)$ 
SNR of POSSUM additive noise. $b)$ Mutual Information between
the signal predicted by the particle filter and the actual signal.}
\label{fig:PossumResult}
\end{figure}
\begin{table}[t]
\ninept
  \centering
\begin{tabular}{| c | c | c | c | c | c |}
\hline
& Sim Value & \multicolumn{2}{|c|}{Prior} &\multicolumn{2}{|c|}{Posterior} \\
\hline
        &       &$\mu$  & $\sigma$& $\mu$& $\sigma$\\
\hline
$\tau_0  $& 1.45  & .98   & .25     & 1.07319 & 0.297749  \\
\rowcolor[gray]{.8} 
$\alpha  $&0.3    & .33   & .045    & 0.317955& 0.07311  \\
$E_0     $& 0.47  & .34   & .03     & 0.364195& 0.0457019  \\
\rowcolor[gray]{.8} 
$V_0     $& 0.044 & .04   & .03     & 0.110542& 0.0500039  \\
$\tau_s  $& 1.94  & 1.54  & .25     & 1.86785 & 0.390455  \\
\rowcolor[gray]{.8} 
$\tau_f  $& 1.99  & 2.46  & .25     & 2.50166 & 0.348326  \\
$\epsilon$& 1.8   & .7    & .6      & 1.58716 & 1.14536  \\
\hline
\end{tabular}
\caption{Comparison of the true value (left), the initial (prior) distribution
(center two columns) and the 
mean/variance of the estimates across simulated voxels (Right two Columns).} 
\label{tab:beforeafter}
\end{table}
For POSSUM simulations, \autoref{fig:PossumResult} shows the
the mutual information between the estimated timeseries and the 
preprocessed signal. Note that in order to remove all false
positives a threshold of $0.15$ bits was applied, 
the regions of significant SNR are visible despite this.
Each estimate was generated from the 
mean of the particles' parameters at the end of the run. 
Additionally the mean and variance of the parameter estimates
(mean) across the spatial regions of \autoref{fig:PossumResult} are
given in \autoref{tab:beforeafter}.

\section{Discussion}
\label{sec:Conclusion}
The data shows substantial correlation present in the parameter 
estimates for the 40 minute simulated data. Correlation of 
constants is a definite sign of the model being under-determined
and thus not observable. This is in spite of low noise, and the 
particle filter's solution converging to a Root-Mean-Squared error
of less than $0.002$. 

On a larger scale, the particle filter also performed well, clearly
locating regions of activation in spite of Signal-to-Noise Ratio 
well below 6dB. However, the final parameter estimates 
varied more than the initial (prior) distribution (\autoref{tab:beforeafter}).
This effect is slightly deceiving because the distributions
were far from Gaussian, and many were bimodal. It is 
significant, however, that such a wide range of solutions
could give good estimates for the true BOLD signal. 

From the results, it is clear that the BOLD model as presented
by Friston et al. is not observable, and thus parameter estimates
suffer from very large variance \cite{Friston2000}. Therefore
any work attempting to estimate these parameters from the BOLD
signal alone is likely measuring the variance due the model
rather than the variance due to try physiological differences. 
There are two ways to deal with this problem. First, as
specified in Deneux et al., certain parameters could be held constant
\cite{Deneux2006}. This solution is relatively easy, and would
also reduce computation time. A better way of dealing with this 
problem is simultaneously measuring fMRI and \ac{CBF}.  
This additional measurement, which is closer to 
original stimulus, could make the Balloon Model observable.

Despite the lack of observability, the BOLD estimates generated 
by the particle filter stayed near the noise level. 
Thus, for the purpose of estimating BOLD levels between measurements,
the particle filter is sufficient. The on-line nature of the
particle filter could also be utilized, along with BOLD estimates,
for adaptive control. Thus the stimulus could be
adapted during the fMRI scan based on BOLD estimates within a small neural
region. The particle filter also has the distinct benefit of 
estimating a full posterior distribution. A joint distribution of 
parameters allows for more advanced analysis; such as 
non-parametric hypothesis testing.  
While the particle filter took a day of processing for full brain calculations, its speed
was sufficient on a quad core machine to perform real time calculations of small regions
(approximate run time .27 seconds per voxel-measurement). 

\bibliographystyle{IEEEbib}
\bibliography{library}

\end{document}
