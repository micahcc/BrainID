% Template for ISBI-2011 paper; to be used with:
%          spconf.sty  - ICASSP/ICIP LaTeX style file, and
%          IEEEbib.bst - IEEE bibliography style file.
% --------------------------------------------------------------------------
\documentclass{article}
\usepackage{spconf,amsmath,graphicx}
\usepackage[printonlyused,nolist]{acronym}
\usepackage[colorlinks=true,linkcolor=black,citecolor=black]{hyperref}

% Example definitions.
% --------------------
\def\x{{\mathbf x}}
\def\L{{\cal L}}

% Title.
% ------
\title{An Analysis of Blood-Oxygen-Level-Dependent Signal Parameter Estimation Using Particle Filters}
%
% Single address.
% ---------------
\name{Chambers, Micah C}
\address{Virginia Tech\\
    Bradley Department of Electrical and Computer Engineering\\
    302 Whittemore Hall\\
    Blacksburg, VA 24061-0111}
%
% For example:
% ------------
%\address{School\\
%	Department\\
%	Address}
%
% Two addresses (uncomment and modify for two-address case).
% ----------------------------------------------------------
%\twoauthors
%  {A. Author-one, B. Author-two\sthanks{Thanks to XYZ agency for funding.}}
%	{School A-B\\
%	Department A-B\\
%	Address A-B}
%  {C. Author-three, D. Author-four\sthanks{The fourth author performed the work
%	while at ...}}
%	{School C-D\\
%	Department C-D\\
%	Address C-D}
%
% More than two addresses
% -----------------------
% \name{Author Name$^{\star \dagger}$ \qquad Author Name$^{\star}$ \qquad Author Name$^{\dagger}$}
%
% \address{$^{\star}$ Affiliation Number One \\
%     $^{\dagger}$}Affiliation Number Two
%


\begin{document}
%\ninept
%
\maketitle
%
\begin{abstract}
The Blood-Oxygen-Level-Dependent (BOLD) Signal that is measured by functional 
magnetic resonance imaging (fMRI)
has been the subject of extensive research since the first development 
of the balloon model. While there would be definite benefits to 
moving from the General Linear Model to a physiologically inspired
BOLD model, significant barriers remain. Optimizing even the simplest
balloon model requires searching in 7 dimensions, and even more flexible
models are available. Whereas traditional methods of analyzing fMRI asks
where activation occurs, BOLD models would ask what the activation 
looks like. Unfortunately, the nonlinear nature of the Balloon Model
makes it difficult to analyze, therefore this work demonstrates the 
use of a particle filter to regresses the simplest form of the BOLD 
model. The results show that the system of equations are not observable,
leading to a large range of valid model parameters. 
\end{abstract}
%
\begin{keywords}
BOLD Response, FMRI, Nonlinear Systems, Particle Filter, Bayesian Statistics, System Identification
\end{keywords}
%

\begin{acronym}[CMRO2]
\acro{BOLD}{Blood-Oxygen-Level-Dependent}
\acro{CBF}{Cerebral Blood Flow}
\acro{CBV}{Cerebral Blood Volume}
\acro{CMRO2}{Cerebral Metabolic Rate of Oxygen}
\acro{DC}{Direct Current}
\acro{dHb}{Deoxygenated Hemoglobin}
\acro{EM}{Expectation-Maximization}
\acro{EPI}{Echo Planar Imaging}
\acro{fMRI}{Functional Magnetic Resonance Imaging}
\acro{FMRIB}{Oxford Centre for Functional MRI of the Brain}
\acro{FSL}{FMRIB Software Library}
\acro{FWHM}{Full-Width Half-Maximum}
\acro{GA}{Genetic Algorithms}
\acro{GLM}{General Linear Model}
\acro{HRF}{Hemodynamic Response Function}
\acro{Hb}{Hemoglobin}
\acro{MAD}{Median Absolute Deviation}
\acro{MI}{Mutual Information}
\acro{MR}{Magnetic Resonance}
\acro{MRI}{Magnetic Resonance Imaging}
\acro{MSE}{Mean Squared Error}
\acro{O2Hb}{Oxygenated Hemoglobin}
\acro{ODE}{Ordinary Differential Equation}
\acro{PDF}{Probability Density Function}
\acro{POSSUM}{Physics-Oriented Simulated Scanner for Understanding MRI}
\acro{RF}{Radio Frequency}
\acro{RMSE}{Root Mean Squared Error}
\acro{RMSR}{Root Mean Squared Residual}
\acro{SA}{Simulated Annealing}
\acro{SNR}{Signal-to-Noise Ratio}
\acro{SPM}{Statistical Parametric Mapping}
\acro{T1}{Longitudinal}
\acro{T2}{Spin-Spin}
\acro{TR}{Repetition Time}
\acro{UKF}{Unscented Kalman Filter}
\end{acronym}
\section{Introduction}
\label{sec:intro}
Traditional methods of analyzing timeseries images produced by 
\ac{fMRI} perform regression using the linear combination of explanatory variables. 
Though moving to a state space model with more more degrees of freedom 
requires additional computation, in this paper Particle Filters will
be used to estimate the governing parameters of the \ac{BOLD} model 
at a computation cost 
that would still allow real time calculations for multiple voxels.
In practical terms, this algorithm could be used as an alternative to \ac{SPM}
for activation detection. Though more computationally intense,
this method is capable of modeling nonlinear effects
and provides more detailed output. 

In particular this work focuses
on the observability of the \ac{BOLD} state equations from Friston et al.
\cite{Friston2000}. Because of the nonlinearities in the state equations,
no analytical method exists to determine observability, however little
movement in parameters was necessary to achieve high predictive quality from a
set parameters. High correlation as calculated from the joint \ac{PDF}
also indicates a lack of observability.  Future works will benefit from
restricting parameters and thus reducing the dimensionality of the system.
While this could reduce the plausibility of parameter estimates, it would
allow for better differentiation of parameters for analyzing pathologies
and reduce reduce computation time from the 40 seconds of analysis
per 5 minute timeseries (on a Core 2 Duo Q6600).

\subsection{BOLD Model}
It is well known that the two types of \ac{Hb} act as contrast agents in 
\ac{EPI} imaging \cite{Buxton1998, WEISSKOFF1994, Ogawa}, however the connection
between \ac{dHb}/\ac{O2Hb} and neural activity is non-trivial. 
Intuitively, increased 
metabolism will increase \ac{dHb}, however blood vessels are quick
to compensate by increasing local blood flow. Increased inflow, accomplished by loosening 
capillary beds, precedes increased outflow, driving increased 
blood storage.
Since the local \ac{MR} signal depends on the ratio of \ac{dHb} to \ac{O2Hb},
increased blood volume affects this ratio if 
metabolism doesn't exactly match the increased inflow of oxygenated blood.
This was the impetus
for the ground breaking balloon model \cite{Buxton1998} and windkessel
model \cite{Mandeville1999}. These works derive, from first principals,
the changes in \ac{dHb} ratio and volume of capillaries given an inflow waveform.
These were the first two attempts to quantitatively account for the shape of the 
\ac{BOLD} signal as a consequence of the lag between the \ac{CBV}
and the inward \ac{CBF}. 

Although Buxton et al. demonstrated that a well chosen flow waveform could 
explain most features of the \ac{BOLD} signal, it stopped short of proposing a
realistic waveform for the \ac{CBF} and \ac{CMRO2} \cite{Buxton1998}. Friston et al. 
gave a reasonable and simple
expression for \ac{CBF} input based on a flow inducing signal
and in the same work proposed a simple method
of estimating metabolic rate: as a direct function of the inward blood flow \cite{Friston2000}.
By combining these equations with the balloon model from Buxton et al.,
it is possible to predict the \ac{BOLD} signal directly from a stimulus time course:
\begin{eqnarray}
\dot{s} &=& \epsilon u(t) - \frac{s}{\tau_s} - \frac{f - 1}{\tau_f} \\
\dot{f} &=& s\\
\dot{v} &=& \frac{1}{\tau_0}(f - v^\alpha)\\
\dot{q} &=& \frac{1}{\tau_0}(\frac{f(1-(1-E_0)^f)}{E_0} - \frac{q}{v^{1-1/\alpha}})
\label{eq:bold}
\end{eqnarray}
where $s$ is a flow inducing signal, $f$ is the input \ac{CBF},
$v$ is normalized \ac{CBV}, and $q$ is the normalized
local \ac{dHb} content. The 
parameters controlling blood flow are $\epsilon$, which is a neuronal 
efficiency term, $u(t)$ which is the stimulus, and $\tau_f$, $\tau_s$ 
which are time constants. The parameters for the evolution of blood 
volume are $E_0$ which the resting metabolic
rate and $\alpha$ which is Grubb's parameter controlling the balloon model. 
$\tau_0$ is a single time constant controlling the speed of $v$ and $q$.

Obata refined the readout equation 
of the \ac{BOLD} signal based on the
\ac{dHb} content (q) and local blood volume (v), resulting in the
final \ac{BOLD} measurement \autoref{eq:boldout} \cite{Obata2004}.
\begin{eqnarray}
\label{eq:boldout}
y   &=& V_0((k_1 + k_2)(1-q) - (k_2 + k_3)(1-v))\\
k_1 &=& 4.3 \times \nu_0 \times E_0 \times TE = 2.8 \nonumber\\
K_2 &=& \epsilon_0 \times r_0 \times E_0 \times TE = .57 \nonumber \\
k_3 &=& \epsilon_0 - 1 = .43 \nonumber
\end{eqnarray}
Where $\nu_0 = 40.3 s^{-1}$  is the frequency offset in Hz for fully
de-oxygenated blood (at 1.5T), $r_0 = 25 s^{-1}$  is the slope relating
change in relaxation rate with change in blood oxygenation, $TE$ is the
echo time used by the \ac{EPI} sequence and $\epsilon_0 = 1.43$ is the 
ratio of signal \ac{MR} from intravascular to extravascular regions at rest. 

While this model is more accurate than the static hemodynamic model used in \ac{SPM},
there have been significant additions which add more degrees of freedom \cite{Deneux2006}. 
In Deneux et al. it was found that the parameters are far from perpendicular,
and that very different parameters could give nearly identical \ac{BOLD} output;
this work builds on that by calculating a joint distribution for the
model parameters of this basic version of the model\cite{Deneux2006}. 

There have been many previous attempts to learn the parameters of the
Balloon model, although none have been extremely successful. 
Friston et al. proposed a novel combination of 
linear and nonlinear modeling to generate parameter estimates
\cite{Friston2002b}.  Thus Friston et al. approximated the Jacobian, 
$\frac{\partial Y}{\partial \theta}$, by approximating the response
using a low-order Volterra Kernel \cite{Friston2002b}. 
Unfortunately, estimating a Volterra kernel
takes longer than simply integrating the state variables. Thus,
this method simplified to a generic nonlinear-least squares algorithm.

In Johnston et al. \cite{Johnston2007}, a hybrid particle filter/gradient
descent algorithm was used to simultaneously derive the static and dynamic 
parameters, (classically known as parameters and state variables, respectively)
A particle filter was used to calculate the state variables at each
time; then the estimated distribution of the particles was used to find
the most likely set of parameters that would give that distribution of state variables.
This process was repeated until the parameters converged. This
method is significantly more complicated than simply using the particle
filter to calculate the parameters; and certainly takes longer to run.

Hu et al. used an Unscented Kalman Filter to simultaneously calculate
the model parameters and state variables \cite{Hu2009}.
However, because the Kalman Filter assumes a multivariate 
Gaussian for the state variables, $X(t-1)$, the posterior 
distribution is limited. Additionally, a nonlinear transformation
of a Gaussian can be significantly non-Gaussian. 

In Vakorin et al., a combination of Genetic Algorithms and 
Simulated Annealing was used to estimate not only the parameters, but the true 
stimuli driving the \ac{BOLD} signal  \cite{Vakorin2007}. 
This addresses the inherent uncertainty of exactly where and when 
stimuli actually get applied. Unfortunately this algorithm took
in more than 16 hours per voxel.

In his PhD thesis, Murray \cite{Murray2008} used a particle filter based 
approach to integrate
the \ac{BOLD} equations. The method used in that work focused primarily on estimating
the \ac{BOLD} output and state equations as a nonlinear stochastic differential 
equation. The primary difference between that work and this is that
Murray \cite{Murray2008} took the parameters as a given. Thus, differences in the \ac{BOLD} output
were taken to be primarily driven by stochastic changes in the underlying state
equations. Because the parameters were not allowed to change, the estimate of 
the \ac{BOLD} signal was not very good. The filtering framework created
in Murray \cite{Murray2008}, dysii, forms the basis for the particle 
filter used in this work, 
and was well designed. The work also clearly presents the particle filter;
both its derivation and use. 

\section{Methods}
\label{sec:Methods}
\subsection{Particle Filter}
\label{sec:ParticleFilter}
Particle filters, a type of Sequential Monte Carlo (SMC) method,
are a powerful method for on line estimation of the posterior probability 
distribution of parameters given a timeseries and a model. The concept of 
particle filters is similar to Kalman Filters; however,
distributions are stored as an empirical distribution rather than 
as the first two moments of a Gaussian. Thus, particle filters are 
preferred when the model is nonlinear, and by implication non-Gaussian. 
The particle filters are a well established technique and are 
described in great detail in Arulampalam et al. and Thrun et al. 
\cite{Arulampalam2002a, thrun2008probabilistic}.

The particle filter was set with each particle having one possible 
system realization: $\{\tau_0, \alpha, E_0, V_0, \tau_s, \tau_f,
\epsilon, s, f, v, q\}$. Initially 16,000 particles were drawn
from the prior, with parameters independently gamma distributed:
$\tau_0 \sim \Gamma(.98, .25)$, 
$\alpha \sim \Gamma(.33, .045)$, $E_0    \sim \Gamma(.34, .03)$,
$V_0    \sim \Gamma(.04, .03)$, $\tau_s \sim \Gamma(1.54, .25)$,
$\tau_f \sim \Gamma(2.46, .25)$, $\epsilon \sim \Gamma(.7, .6)$.
To reduce the size of the search space from 11 to 7, $\{s, f, v, q\}$
were assumed to be at resting state ($\{0, 1, 1, 1\}$). Although
it is common to add noise to the state during integration (to 
emulate noise in the state updates), this was not performed because 
the cloud of particles is able to account for this to some degree.

\subsection{Configuration}
The original assumption regarding particle filter models (\autoref{sec:Particle Filter Model})
included noise in the update of $x$, however that is not included here.
The reason for the difference is that the cloud of particles is, to some extent,
able to account for that noise. It is common, however, to model that noise
in particle filters by adding a random value to each updated state variable.
Because the purpose of this particle filter is to learn the underlying distribution
of the static parameters, rather than precisely modeling the time course of the
in the dynamic state variables $\{s,f,v,q\}$ this noise is left out. It also helps
that detrending is applied before the particle filter and that the
\ac{BOLD} model is dissipative. When no stimuli are applied, all the particles
decay to $\{0,1,1,1\}$. Typical particle filters
also use this state noise as an exploratory measure; however, this method is
less necessary when good priors are available, and when regularized particle filtering
is being used.

For all the analysis  in this work, $1400$ integration points
per second were used.  Typically a step size of $0.001$ was sufficient,
however, from time to time $0.001$ can still result in problems for the
\ac{BOLD} model.

\subsection{Preprocessing}
\label{sec:Methods Preprocessing}
The normal pipeline for analyzing
\ac{fMRI} involves a several preprocessing steps to condition the signal.
The first and most important
task is motion correction. To do this, a single volume in time is chosen, and
volumes at every other time point are realigned to best match the target volume. This corrects
for motion by the patient as well as small changes in the magnetic
fields that cause the image to shift.
In conventional statistical parametric mapping, a Gaussian smoothing
filter is applied across the image as discussed in \autoref{sec:RFT}.
After this, detrending is performed, which is discussed in \autoref{sec:Detrend}.
Recall that \ac{fMRI} signal levels are unit-less and though detrending is not
always necessary, the data must always be converted
into \% difference from baseline.
The generally accepted method is to use a high pass filter, although the
cutoff frequency is application dependent and often applied haphazardly.
Before going into the detrending used in this work, it is necessary to
discuss the type of noise present in \ac{fMRI}.

The non-stationary
aspect of a Wiener process, presumably the result of integrating some
$\nu_x$, is difficult to compensate for, and so many methods
have been developed to compensate for it. Tanabe et al. \cite{Tanabe2002}
and Smith et al. \cite{Smith1999}
demonstrated that this component is prevalent, and may in fact be an inherent  characteristic
of \ac{fMRI}. It has been reported that in some studies as
many as half the voxels
benefited from detrending \cite{Smith2007}. In comparison of high pass 
filters, 
Tanabe et al. \cite{Tanabe2002} showed that in most cases subtracting off
a spline worked the best.
Unfortunately no method will
perfectly remove noise, and no method will leave the signal untouched.

The method used to calculate the spline was one knot for every 20
measurements in an image. Thus a 10 minute session at a repetition time of
2.1 seconds would have 19 knots. The first and last knots were each
given half the number of samples as the rest of the knots; which were all
located at the center of their sample group. The median of each sample group
was then taken and used as the magnitude for the group. Taking the median
versus the mean seemed to work better, given the presence of outliers.
There is potential to optimize the spline further using a canonical
HRF to find resting points; however, the experiment would have
to be designed with this in mind.

Problematically, after removing the \ac{DC} component of the signal,
by definition the signal will have a median near zero.
Unfortunately this is not the natural state of the \ac{BOLD} signal. More specifically,
when the signal is inactive, the \ac{BOLD} response should be at 0\% change from
the base level; activation may then increase, or for short periods decrease from this base.
Because most of the \ac{BOLD} signal is above baseline, after removing the spline
the resting state will be below 0\%.
One method of accounting for this is to simply add a \ac{DC} gain model parameter.
Like all the other model parameters, with enough measurements, the particle filter
would be able to settle on a good estimate. Yet adding another dimension increases the
complexity of the model, for a parameter that is relatively easy to estimate
by visual inspection.  In this work a simpler approach was used. To determine
the \ac{DC} gain a robust estimator of scale was used. The 
\ac{MAD} proved to be accurate in determining how much to shift the signal up
by. Both methods analysis were tested, and it was found that the increase
in model complexity far outweighed the slight increase in flexibility. Other
methods may work better, however the \ac{MAD} worked well,
as \autoref{fig:PreprocessedLowNoise} and 
\autoref{fig:PreprocessedHighNoise} show. The \ac{DC} gain was then set to:

\begin{equation}
y_{\text{gain}, 0:K} = 1.4826\underset{i=0:K}{\text{median}}(y_i - \text{median}(y_{0:K}))
\label{eq:mad}
\end{equation}

For more information on the \ac{MAD}, it is discussed in great detail in
Iglewicz's text on robust estimation \cite{iglewicz1983robust}.
A serious concern when adding constant values to
real data is whether this will create false positives. This is a legitimate
concern; however, a boosted response does not affect how well the \ac{BOLD} model
predicts the actual measurements; and as mentioned before, the \ac{DC} signal 
of \ac{fMRI} is never used.


\section{Results}
\label{sec:Results}

\section{Conclusion}
\label{sec:Conclusion}

% Below is an example of how to insert images. Delete the ``\vspace'' line,
% uncomment the preceding line ``\centerline...'' and replace ``imageX.ps''
% with a suitable PostScript file name.
% -------------------------------------------------------------------------
\begin{figure}[htb]

\begin{minipage}[b]{1.0\linewidth}
  \centering
%  \centerline{\includegraphics[width=8.5cm]{image1}}
%  \vspace{2.0cm}
  \centerline{(a) Result 1}\medskip
\end{minipage}
%
\begin{minipage}[b]{.48\linewidth}
  \centering
%  \centerline{\includegraphics[width=4.0cm]{image3}}
%  \vspace{1.5cm}
  \centerline{(b) Results 3}\medskip
\end{minipage}
\hfill
\begin{minipage}[b]{0.48\linewidth}
  \centering
%  \centerline{\includegraphics[width=4.0cm]{image4}}
%  \vspace{1.5cm}
  \centerline{(c) Result 4}\medskip
\end{minipage}
%
\caption{Example of placing a figure with experimental results.}
\label{fig:res}
%
\end{figure}

% To start a new column (but not a new page) and help balance the last-page
% column length use \vfill\pagebreak.
% -------------------------------------------------------------------------
%\vfill
%\pagebreak


\section{References}
\label{sec:ref}

% References should be produced using the bibtex program from suitable
% BiBTeX files (here: strings, refs, manuals). The IEEEbib.bst bibliography
% style file from IEEE produces unsorted bibliography list.
% -------------------------------------------------------------------------
\bibliographystyle{IEEEbib}
\bibliography{/home/micahc/library}

\end{document}
