%Journal Alticle Layout by Micah Chambers
%using the IEE bare_jrnl.tex file at 
%% http://www.michaelshell.org/tex/ieeetran/
%% http://www.ctan.org/tex-archive/macros/latex/contrib/IEEEtran/
%% and
%% http://www.ieee.org/

\documentclass[journal]{./IEEEtran}

\usepackage{cite}

\usepackage[pdftex]{graphicx}
\graphicspath{{pdf/}{png/}}

\usepackage[cmex10]{amsmath}
\interdisplaylinepenalty=2500

\usepackage{algorithmic}
\usepackage{array}

\usepackage[caption=false,font=footnotesize]{subfig}
\usepackage{fixltx2e}

\usepackage[colorlinks=true,linkcolor=black,citecolor=black]{hyperref}
\usepackage{url}
\hyphenation{op-tical net-works semi-conduc-tor}

%\begin{comment}
%:Author: Micah Chambers
%:Institution: Virginia Tech University
%\end{comment}

%\newcommand{\gettitle}{\text{Full Brain Blood-Oxygen-Level-Dependent Signal Parameter Estimation Using Particle Filters}}

\begin{document}

\title{Full Brain Blood-Oxygen-Level-Dependent Signal Parameter Estimation Using Particle Filters}

\author{{Micah~Chambers}% <-this % stops a space
\thanks{M. Chambers is with the Department
of Electrical and Computer Engineering, Virginia Tech University, Blacksburg,
VA, 24060 USA e-mail: (micahc@vt.edu}% <-this % stops a space
\thanks{Manuscript received April 19, 2005; revised January 11, 2007.}}

\markboth{Journal of \LaTeX\ Class Files,~Vol.~6, No.~1, January~2007}%
{Chambers: Full Brain Blood-Oxygen-Level-Dependent Signal Parameter Estimation Using Particle Filters}
% The only time the second header will appear is for the odd numbered pages
% after the title page when using the twoside option.
% 
% *** Note that you probably will NOT want to include the author's ***
% *** name in the headers of peer review papers.                   ***
% You can use \ifCLASSOPTIONpeerreview for conditional compilation here if
% you desire.

\maketitle

\begin{abstract}
Traditional methods of analyzing FMRI images use a linear combination of
just a few static regressors. This works demonstrates an alternative
approach using a physiologically inspired nonlinear model. By using a 
particle filter to optimize the model parameters, the computation time
is kept below a minute per voxel without requiring a linearization 
of the noise in the state
variables. The activation results show regions similar those found in 
SPM; however, there are some notable regions not detected by 
SPM. Though the parameters selected by the particle filter based approach
are more than sufficient to predict the BOLD response,
more model constraints are needed to uniquely identify a single set
of parameters. This ill-posed nature explains the large discrepancies
found in other research that attempted to characterize the model parameters.
For this reason the final distribution of parameters is more medically relevant
than a single estimate. Because the output of the particle filter is 
a full posterior probability, the reliance on the mean to estimate 
parameters is unnecessary. This work presents
not just a viable alternative to the traditional method of detecting
activation, but an extensible technique of estimating the joint probability
distribution of the BOLD parameters.
\end{abstract}

% Note that keywords are not normally used for peerreview papers.
\begin{IEEEkeywords}
BOLD Response, FMRI, Nonlinear Systems, Particle Filter, Bayesian Statistics, System Identification
\end{IEEEkeywords}

% For peer review papers, you can put extra information on the cover
% page as needed:
% \ifCLASSOPTIONpeerreview
% \begin{center} \bfseries EDICS Category: 3-BBND \end{center}
% \fi
%
% For peerreview papers, this IEEEtran command inserts a page break and
% creates the second title. It will be ignored for other modes.
\IEEEpeerreviewmaketitle

\section{Introduction}
\label{sec:Introduction}
% The very first letter is a 2 line initial drop letter followed
% by the rest of the first word in caps.
% 
% form to use if the first word consists of a single letter:
% \IEEEPARstart{A}{demo} file is ....
% 
% form to use if you need the single drop letter followed by
% normal text (unknown if ever used by IEEE):
% \IEEEPARstart{A}{}demo file is ....
% 
% Some journals put the first two words in caps:
% \IEEEPARstart{T}{his demo} file is ....
% 
% Here we have the typical use of a "T" for an initial drop letter
% and "HIS" in caps to complete the first word.
\IEEEPARstart{T}{raditional} methods of analyzing 
Functional Magnetic Resonance Imaging (FMRI)
time series perform regression using a linear 
combination of static explanatory variables. 
However, the static nature of the General Linear Model (GLM)
limits its potential use. It is well known that different 
Hemodynamic Response Functions are necessary 
for different regions of the brain to prevent excessive false
negatives \cite{Handwerker2004}. Additionally quite a few studies
have reported spatially varying nonlinearities \cite{Wager2005,Birn2001}.
Besides not allowing HRF differences between patients, there is no
reasonable way to incorporate other forms of physiological
data. Combined FMRI CBF or CBV imaging methods are improving,
as seen in Chen et al. \cite{Chen2009}. These techniques could shed light on
neural activation by providing extra measurements, yet a 
physiologically reasonable model is necessary to incorporate this extra data.
Activation detection methods also don't have the ability 
to identify pathologies based on state variables or parameters. For
example, decreased compliance of blood vessels could indicate, or 
even cause, a neurological condition that 
is not easily seen in other imaging modalities. 

It is well known that the changes in deoxy-hemoglobin content is the
primary driver of short term changes in MR signal for FMRI imaging
techniques \cite{Buxton1998, WEISSKOFF1994, Ogawa, Obata2004}. For
this reason, the nonlinear state equations that change Deoxy-Hb (DHb) 
content have been heavily studied, and are well-characterized. Attempts
at learning the parameters of the BOLD model have also been actively
studied but significantly less successful. The most wide-spread method
of calculating the parameters, from two papers by Friston et al., 
are part of the SPM toolkit; though this algorithm is rarely used 
as well \cite{Friston2000, Friston2002b}. 
This method depends on a quadratic convolution (two term Volterra Kernel)
based estimation of the BOLD output to obtain partials with respect 
to parameters. Problematically, quality of the Volterra estimate
is not well known, so it is difficult to quantify error. This method 
also doesn't account for noise in the parameters or the state variables.
These shortcomings have led to several other separate attempts. In particular 
Riera et al. linearized the noise and thus performed regression between
the model and steps of the BOLD output \cite{Riera2003}. 
In Vakorin et al., a combination of Genetic Algorithms and Simulated 
Annealing were used to estimate not only the parameters, but the 
true stimuli driving the BOLD signal \cite{Vakorin2007}. 
This addresses the inherent uncertainty of exactly where and when 
stimuli actually get applied. Unfortunately this algorithm was extremely slow.
In Johnston et al. a particle filter was applied with the purpose of
estimating the output distribution, which was then used to optimize
parameters by means of least squares based techniques \cite{Johnston2008}. 
Murray used a particle filter, similar to \cite{Johnston2008}, but
held the model parameters constant; thus all error in the output
was treated as the result of noise in the underlying parameters 
\cite{Murray2008}.  Hu et al. used an unscented kalman filter (UKF) 
was used to optimize parameters \cite{Hu2009}.

\section{Methods}
\label{sec:Methods}
\begin{itemize}
\item BOLD Model
\item Particle Filter Description 
\item Description of Simulation
\item Description of Real FMRI data
\end{itemize}

\section{Results}
\label{sec:Results}
\begin{itemize}
\item Simulation Results - Possum Only
\begin{itemize}
    \item Histogram of parameters for each region
    \item Final histogram of one run?
    \item Comparison of MI with SNR
\end{itemize}
\item FMRI Results 
\begin{itemize}
    \item Activation maps comparing M.I. with SPM
\end{itemize}
\end{itemize}

\section{Discussion}
\label{sec:Discussion}

\begin{itemize}
\item Parameters ill-defined
\end{itemize}

\section{Conclusion}
\label{sec:Conclusion}

%This article is organized as follows. The rest of the introduction will
%overview similar efforts and discuss their strengths ans weaknesses.
%\autoref{sec:Bold Model} will describe the genesis of the BOLD model,
%and its current state. \autoref{sec:Particle Filter} will explain the 
%particle filter and how it is being used in this case. 
%\auroref{sec:Methods} will describe in detail the experimental
%design, algorithm set up, and the preprocessing that was applied
%for the particle filter algorithm.
%The results are explored separately for simulated data
%and real FMRI data in \autoref{sec:SimulationResults} and 
%\autoref{sec:RealData}, respectively. 
%In \autoref{sec:Discussion} the results and their implications 
%for future works are interpreted. 

\appendices
\section{Particle Filter Algorithm}
Appendix one text goes here.

% you can choose not to have a title for an appendix
% if you want by leaving the argument blank
\section{}
Appendix two text goes here.


% use section* for acknowledgement
%\section*{Acknowledgment}

\bibliographystyle{IEEEtran}
\bibliography{IEEEabrv,./library}

\end{document}
