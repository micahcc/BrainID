%Journal Alticle Layout by Micah Chambers
%using the IEE bare_jrnl.tex file at 
%% http://www.michaelshell.org/tex/ieeetran/
%% http://www.ctan.org/tex-archive/macros/latex/contrib/IEEEtran/
%% and
%% http://www.ieee.org/

\documentclass[journal]{./IEEEtran}

\usepackage{cite}

\usepackage[pdftex]{graphicx}
\graphicspath{{pdf/}{png/}}

\usepackage[cmex10]{amsmath}
\interdisplaylinepenalty=2500

\usepackage{algorithmic}
\usepackage{array}

\usepackage[caption=false,font=footnotesize]{subfig}
\usepackage{fixltx2e}

\usepackage[colorlinks=true,linkcolor=black,citecolor=black]{hyperref}
\usepackage{url}
\hyphenation{op-tical net-works semi-conduc-tor}

%\begin{comment}
%:Author: Micah Chambers
%:Institution: Virginia Tech University
%\end{comment}

%\newcommand{\gettitle}{\text{Full Brain Blood-Oxygen-Level-Dependent Signal Parameter Estimation Using Particle Filters}}

\begin{document}

\title{Full Brain Blood-Oxygen-Level-Dependent Signal Parameter Estimation Using Particle Filters}

\author{{Micah~Chambers}% <-this % stops a space
\thanks{M. Chambers is with the Department
of Electrical and Computer Engineering, Virginia Tech University, Blacksburg,
VA, 24060 USA e-mail: (micahc@vt.edu}% <-this % stops a space
\thanks{Manuscript received April 19, 2005; revised January 11, 2007.}}

\markboth{Journal of \LaTeX\ Class Files,~Vol.~6, No.~1, January~2007}%
{Chambers: Full Brain Blood-Oxygen-Level-Dependent Signal Parameter Estimation Using Particle Filters}
% The only time the second header will appear is for the odd numbered pages
% after the title page when using the twoside option.
% 
% *** Note that you probably will NOT want to include the author's ***
% *** name in the headers of peer review papers.                   ***
% You can use \ifCLASSOPTIONpeerreview for conditional compilation here if
% you desire.

\maketitle

\begin{abstract}
Traditional methods of analyzing FMRI images use a linear combination of
just a few static regressors. This works demonstrates an alternative
approach using a physiologically inspired nonlinear model. By using a 
particle filter to optimize the model parameters, the computation time
is kept below a minute per voxel without requiring a linearization 
of the noise in the state
variables. The activation results show regions similar those found in 
SPM; however, there are some notable regions not detected by 
SPM. Though the parameters selected by the particle filter based approach
are more than sufficient to predict the BOLD response,
more model constraints are needed to uniquely identify a single set
of parameters. This ill-posed nature explains the large discrepancies
found in other research that attempted to characterize the model parameters.
For this reason the final distribution of parameters is more medically relevant
than a single estimate. Because the output of the particle filter is 
a full posterior probability, the reliance on the mean to estimate 
parameters is unnecessary. This work presents
not just a viable alternative to the traditional method of detecting
activation, but an extensible technique of estimating the joint probability
distribution of the BOLD parameters.
\end{abstract}

% Note that keywords are not normally used for peerreview papers.
\begin{IEEEkeywords}
BOLD Response, FMRI, Nonlinear Systems, Particle Filter, Bayesian Statistics, System Identification
\end{IEEEkeywords}

% For peer review papers, you can put extra information on the cover
% page as needed:
% \ifCLASSOPTIONpeerreview
% \begin{center} \bfseries EDICS Category: 3-BBND \end{center}
% \fi
%
% For peerreview papers, this IEEEtran command inserts a page break and
% creates the second title. It will be ignored for other modes.
\IEEEpeerreviewmaketitle

\section{Introduction}
\label{sec:Introduction}
% The very first letter is a 2 line initial drop letter followed
% by the rest of the first word in caps.
% 
% form to use if the first word consists of a single letter:
% \IEEEPARstart{A}{demo} file is ....
% 
% form to use if you need the single drop letter followed by
% normal text (unknown if ever used by IEEE):
% \IEEEPARstart{A}{}demo file is ....
% 
% Some journals put the first two words in caps:
% \IEEEPARstart{T}{his demo} file is ....
% 
% Here we have the typical use of a "T" for an initial drop letter
% and "HIS" in caps to complete the first word.
\IEEEPARstart{T}{raditional} methods of analyzing 
Functional Magnetic Resonance Imaging (FMRI)
time series perform regression using a linear 
combination of static explanatory variables. 
However, the static nature of the General Linear Model (GLM)
limits its potential use. It is well known that different 
Hemodynamic Response Functions are necessary 
for different regions of the brain to prevent excessive false
negatives \cite{Handwerker2004}. Additionally quite a few studies
have reported spatially varying nonlinearities \cite{Wager2005,Birn2001}.
Besides not allowing HRF differences between patients, there is no
reasonable way to incorporate other forms of physiological
data. Combined FMRI CBF or CBV imaging methods are improving,
as seen in Chen et al. \cite{Chen2009}. These techniques could shed light on
neural activation by providing extra measurements, yet a 
physiologically reasonable model is necessary to incorporate this extra data.
Activation detection methods also don't have the ability 
to identify pathologies based on state variables or parameters. For
example, decreased compliance of blood vessels could indicate, or 
even cause, a neurological condition that 
is not easily seen in other imaging modalities. 

It is well known that the changes in deoxy-hemoglobin content is the
primary driver of short term changes in MR signal for FMRI imaging
techniques \cite{Buxton1998, WEISSKOFF1994, Ogawa, Obata2004}. For
this reason, the nonlinear state equations that change Deoxy-Hb (DHb) 
content have been heavily studied, and are well-characterized. Attempts
at learning the parameters of the BOLD model have also been actively
studied but significantly less successful. The most wide-spread method
of calculating the parameters, from two papers by Friston et al., 
are part of the SPM toolkit; though this algorithm is rarely used 
as well \cite{Friston2000, Friston2002b}. 
This method depends on a quadratic convolution (two term Volterra Kernel)
based estimation of the BOLD output to obtain partials with respect 
to parameters. Problematically, quality of the Volterra estimate
is not well known, so it is difficult to quantify error. This method 
also doesn't account for noise in the parameters or the state variables.

These shortcomings have led to several other separate attempts. In particular 
Riera et al. linearized the noise and thus performed regression between
the model and steps of the BOLD output \cite{Riera2003}. This aproach
had the benefit of a Jacobian,  but at the same time removes all DC 
signal. For this reason, the type of activation limits the
algorithms' ability to learn the model.
In Vakorin et al., a combination of Genetic Algorithms and Simulated 
Annealing were used to estimate not only the parameters, but the 
true stimuli driving the BOLD signal \cite{Vakorin2007}. 
This addresses the inherent uncertainty of exactly where and when 
stimuli actually get applied. Unfortunately this algorithm was extremely
slow; taking hours or days per single time course.
In Johnston et al. alternating estimates of $P(X_t | \Theta, Y_t)$ and 
$P(\Theta | Y_t)$ were calculated, to maximize $P(X_t, \Theta | Y_t)$. 
$P(X_t | \Theta, Y_t)$ was calculated using a particle filter 
\cite{Johnston2008}. This however was quite slow, and the results
were inconsistent with other similar works. 
Murray used a particle filter, similar to \cite{Johnston2008}, but
held the model parameters constant; thus all error in the output
was treated as the result of noise in the underlying parameters 
\cite{Murray2008}. Murray's results showed that differences in
BOLD output cannot be well explained by error in the state variables.
Hu et al. used an unscented kalman filter (UKF) to estimate parameters,
and filtered parameter sets inconsistent with the output  \cite{Hu2009}.
 Furthermore, Birn et al. and Yacoub et al. showed that many
of the characteristics of the BOLD output depend on the region 
of interest \cite{Birn2001, Yacoub2006}. For this reason, a single
Hemodynamic Response Function is not likely to accurately estimate
activation across the full brain. 

With the exception of Hu et al., the works described thus far all
worked to estimate a single set of parameters that could
explain the output \cite{Hu2009}. Yet Deneux et al showed that the very similar
outputs can be achieved with very different parameters \cite{Deneux2006}.
This explains the inconsistencies in parameters across the different
studies. Thus, estimating parameters using a least squares
framework is not likely to be fruitful; and though the posterior
distributions will vary spatially, that difference will not be
discernible from a single parameter estimate. The problem with 
using an Unscented Kalman Filter to approximate the distributions
is that with a nonlinear, dissipative system, significant non-Gaussian
effects will be present. In tests, significant non-Gaussian effects 
began to appear within a second of state integration. With FMRI
repetition times being well above 1 second, a Gaussian approximation
will result in significant error. The particle filter, which can be
thought of an extension of UKF, uses non-parametric distributions
rather than a Gaussian. For this reason the particle filter, described
in \autoref{sec:ParticleFilters}, is used in this work.

\section{Methods}
\label{sec:Methods}
\subsection{Bold Model}
\label{sec:BoldModel}
In the past fifteen years, a steady stream of studies have built
on the original Blood Oxygen Level Dependent (BOLD) signal 
derivation first described by Ogawa et al. \cite{Ogawa}.
The seminal work by Buxton et al. attempted to explain the
time evolution of the BOLD signal using a windkessel model to
describe the local changes in Deoxygenated Hemoglobin content \cite{Buxton1998}.
Incremental improvements were made to this model until Friston et al.
brought all the changes together into a single complete 
set of equations \cite{Friston2000}. And while there have been numerous adaptations in the model, 
many of them summarized by Deneux et al., even the basic versions
have less bias error than the empirically driven 
\emph{Canonical Hemodynamic Model} \cite{Deneux2006,Handwerker2004}.
On the other hand BOLD signal models have numbers
of parameters ranging from seven \cite{Riera2003} to 50 \cite{Behzadi2005} 
for a signal as short as 100 samples long. This number of parameters presents
a significant risk of being under-determined and having high computation cost. 
In this work, only the simplest physiologically inspired model will be
used (with 7 parameters), and steps will be taken to make the most of computation
time.

It is well known that the two types of hemoglobin act as a contrast agents in 
EPI imaging \cite{Buxton1998, WEISSKOFF1994, Ogawa}, however the connection
between Deoxyhemoglobin/Oxygenated Hemoglobin and neural activity is non-trivial. 
Intuitively, increased metabolism will increase Deoxyhemoglobin, 
however blood vessels are quick to compensate by increasing local 
blood flow. Increased inflow, accomplished by loosening 
capillary beds, precedes increased outflow, driving increased 
blood storage.  Since the local MR signal depends on the ratio of 
Deoxyhemoglobin to Oxygenated Hemoglobin, increased blood volume 
affects this ratio if metabolism doesn't exactly match the increased 
inflow of oxygenated blood.  This was the impetus
for the ground breaking balloon model \cite{Buxton1998} and windkessel
model \cite{Mandeville1999}. These works derive, from first principals,
the changes in deoxyhemoglobin ratio and volume of capillaries given an 
inflow waveform.  These were the first two attempts to quantitatively 
account for the shape of the BOLD signal as a consequence of the lag 
between the cerebral blood volume (CBV) and the inward cerebral blood 
flow (CBF). 

Although Buxton et al. demonstrated that a well chosen flow waveform could 
explain most features of the BOLD signal, it stopped short of proposing a
realistic waveform for the CBF and CMRO2 \cite{Buxton1998}. Friston et al. 
gave a reasonable and simple expression for CBF input based on a flow 
inducing signal and in the same work proposed a simple method
of estimating metabolic rate: as a linear function of the inward 
blood flow \cite{Friston2000}. By combining these equations with 
the balloon model from Buxton et al., it is possible to predict 
the BOLD signal directly from a stimulus time course:
\begin{eqnarray}
\dot{s} &=& \epsilon u(t) - \frac{s}{\tau_s} - \frac{f - 1}{\tau_f} \\
\dot{f} &=& s\\
\dot{v} &=& \frac{1}{\tau_0}(f - v^\alpha)\\
\dot{q} &=& \frac{1}{\tau_0}(\frac{f(1-(1-E_0)^f)}{E_0} - \frac{q}{v^{1-1/\alpha}})
\label{eq:bold}
\end{eqnarray}
where $s$ is a flow inducing signal, $f$ is the input blood flow (CBF),
$v$ is normalized cerebral blood volume (CBV), and $q$ is the normalized
local deoxyhemoglobin. The parameters controlling blood flow are 
$\epsilon$, which is a neuronal efficiency term, $u(t)$ which is 
the stimulus, and $\tau_f$, $\tau_s$ which are time constants. 
The parameters for the evolution of blood volume are $E_0$ which 
the resting metabolic rate and $\alpha$ which is Grubb's parameter 
controlling the balloon model.  $\tau_0$ is a single time constant 
controlling the speed of $v$ and $q$.

This completed balloon model was assembled and analyzed
by Riera et al. \cite{Riera2003}. Obata refined the readout equation 
of the BOLD signal based on the deoxyhemoglobin content ($q$) and local 
blood volume ($v$), resulting in the final BOLD equation \cite{Obata2004}.
\begin{eqnarray}
y   &=& V_0((k_1 + k_2)(1-q) - (k_2 + k_3)(1-v))\\
k_1 &=& 4.3 \times \nu_0 \times E_0 \times TE = 2.8 \nonumber \\
K_2 &=& \epsilon_0 \times r_0 \times E_0 \times TE = .57 \nonumber \\
k_3 &=& \epsilon_0 - 1 = .43 \nonumber
\label{eq:boldout}
\end{eqnarray}
Where $\nu_0 = 40.3 s^{-1}$  is the frequency offset in Hz for fully
de-oxygenated blood (at 1.5T), $r_0 = 25 s^{-1}$  is the slope relating
change in relaxation rate with change in blood oxygenation, and
$\epsilon_0 = 1.43$ is the ratio of signal MR from intravascular to 
extravascular regions at rest. Although, these constants change with 
experiment ($TE$, $\nu_0$, $r_0$), patient, and brain 
region ($E_0$, $r_0$), often the estimated values by Obata are 
taken as the constants $a_1 = k_1 + k_2 = 3.4$, and $a_2 = k_2+k_3 = 1$ in 
studies using 1.5 Tesla scanners \cite{Obata2004}.
Additional compartments and parameters have been added since 
Riera et al., however Deneux et al. showed that only the viscoelastic
effects of Buxton et al. were necessary to model the primary 
elements of the BOLD response \cite{Riera2003, Deneux2006, Buxton2004}.
Despite this, only the basic balloon model of Buxton et al. is
used in this work, although future works may benefit from constraining
the parameters as described by Deneux et al. \cite{Buxton2004, Deneux2006}.

\subsection{Particle Filter}
\label{sec:ParticleFilter}
\subsection{Simulation}
\label{sec:MethodsSim}
\subsection{Data Acquisition}
\label{sec:MethodsData}

\section{Results}
\label{sec:Results}
\begin{itemize}
\item Simulation Results - Possum Only
\begin{itemize}
    \item Histogram of parameters for each region
    \item Final histogram of one run?
    \item Comparison of MI with SNR
\end{itemize}
\item FMRI Results 
\begin{itemize}
    \item Activation maps comparing M.I. with SPM
\end{itemize}
\end{itemize}

\section{Discussion}
\label{sec:Discussion}

\begin{itemize}
\item Parameters ill-defined
\end{itemize}

\section{Conclusion}
\label{sec:Conclusion}

%This article is organized as follows. The rest of the introduction will
%overview similar efforts and discuss their strengths ans weaknesses.
%\autoref{sec:Bold Model} will describe the genesis of the BOLD model,
%and its current state. \autoref{sec:Particle Filter} will explain the 
%particle filter and how it is being used in this case. 
%\auroref{sec:Methods} will describe in detail the experimental
%design, algorithm set up, and the preprocessing that was applied
%for the particle filter algorithm.
%The results are explored separately for simulated data
%and real FMRI data in \autoref{sec:SimulationResults} and 
%\autoref{sec:RealData}, respectively. 
%In \autoref{sec:Discussion} the results and their implications 
%for future works are interpreted. 

\appendices
\section{Particle Filter Algorithm}
Appendix one text goes here.

% you can choose not to have a title for an appendix
% if you want by leaving the argument blank
\section{}
Appendix two text goes here.


% use section* for acknowledgement
%\section*{Acknowledgment}

\bibliographystyle{IEEEtran}
\bibliography{IEEEabrv,./library}

\end{document}
