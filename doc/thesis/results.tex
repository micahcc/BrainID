\chapter{Results}
\section{Single-Voxel Simulation}
The results 

\section{Single-Voxel Analysis}
This section discusses the results when the particle filter was
applied on a single voxel. The parameters are the same as
those used later for entire image analysis; however, the results
are more in-depth. 

\section{Weighting Function Comparison}
\label{sec:Results Weights}

\section{Single Time-Series Simulation}

Graphs: 

For simulated data, single timeseries:

For \{delta, DC/Spline\}, \{exponential, gaussian, cauchy\}, \{biased, unbiased initial\},
\{100, 500, 1000\} particles
\begin{enumerate}
\item Ground truth vs. Estimated signal during particle filter run
\item Ground truth vs. Estimated signal with final parameter set
\item True Parameters vs. Final Parameter Sets
\item Variance of final parameters when faced with same ground truth, different noise
\item MSE of (a new timeseries based on X(t) vs. ground truth) for all t
\item Estimator Variance based on different noise runs
\item Final Particle Distribution
\end{enumerate}

For Simulated Data, Full Volume:

%note to self, epsilon should probably be uniform between 0 and something
\section{Simulated Volume}
\begin{enumerate}
\item Parameter Map 
\item Error map of parameters
\item Histogram of \%errors between parameters
\item Activation Map based on a single region with high $\epsilon$, compared with linear
\end{enumerate}

Final parameter distribution among active regions.
Q-Q plots?

\section{FMRI Data}
....

image comparing epsilon-map with GLM activation map

