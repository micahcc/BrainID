\documentclass[12pt]{article}
\usepackage[left=1in,top=1in,right=1in,nohead]{geometry}

\usepackage[]{times}
\usepackage{verbatim}
\usepackage{amsmath}
\usepackage{amssymb}
\usepackage{multicol}
\usepackage{mdwlist}
%\usepackage{setspace}

%\doublespacing

\newcommand{\R}{\mathbb{R}}
\newcommand{\Q}{\mathbb{Q}}
\newcommand{\N}{\mathbb{N}}
\newcommand{\Z}{\mathbb{Z}}
\newcommand{\qed}{\\$\Box$}
\newcommand{\qle}{\stackrel{?}{\le}}
\newcommand{\qeq}{\stackrel{?}{=}}
\newcommand{\closure}{\overline}
\newcommand{\intersect}{\cap}
\newcommand{\union}{\cup}
\newcommand{\nullset}{\emptyset}
\newcommand{\minus}{\ \backslash\ }


\begin{comment}

:Author: Micah Chambers
:Title: Personal Statement

\end{comment}

\author{Micah Chambers}
\title{Research Experience}

%What are all of your applicable experiences?
%For each experience, what were the key questions, methodology, findings, and conclusions?
%Did you work in a team and/or independently?
%How did you assist in the analysis of results?

\begin{document}
\Large
\begin{center}
Automatic Neural Causality Analysis Using Granger Causality Mapping and Mutual Information

\large
By: Micah Chambers
\end{center}
\normalsize

Recently there has been great interest in the area of causality
mapping of Neural Networks. While this development
is crucial to understanding how the brain works,
there are still many problems with the existing techniques.
Here I propose a new processing pipeline and a new method of performing
causality mapping that makes fewer assumptions 
and is still computationally tractable. The goal then is to
develop a set of tools that will:

\begin{enumerate*}
\item Find independent regions to input into the Granger Causality Mapping (GCM) algorithm.
\item Calculate the causality within the independent regions.
\item Create a dependency tree for intuitive display of causality.
\end{enumerate*}

%\subsubsection*{Background}
For the past ten years the primary method of analysis of Functional
MRI has been Statistical Parametric Mapping (SPM) using 
the General Linear Model (GLM). While this method
has been effective, in the interest of computational tractability
it makes certain assumptions that do not hold. For this reason, such
methods must be extremely conservative to avoid false positives, which means
that SPM is incapable of detecting small activation regions or regions of low 
activation. Additionally, SPM
only measures how well the data fits the GLM and not the actual activation
level. Finally, SPM is a univariate model, meaning it cannot determine
inter-voxel causality.  Because of these limitations, there is currently 
a lot of work being done to find the successor to SPM. 

Recently, "Dynamic Causal Modeling" (DCM) has been introduced, which attempts 
model activation of neural pathways.
Unfortunately DCM has not become anywhere near as ubiquitous a SPM, in no
small part because it is significantly more difficult to use. Because DCM works by 
testing FMRI data against a proposed network, the researcher must propose
a functional layout to perform DCM, which is often difficult.

Another recent development, Vector Autoregression has many desirable
features; it is able to infer dependencies, it requires no a priori knowledge of the system,
and versions exist for nonlinear systems \cite{NVAR}. 
Vector Autoregression is actually a generalized version of Granger Causality Mapping (GCM),
which Zhou et. al. \cite{granger} recently showed to be effective for
finding causal neural networks. The downside is computation time, which can
be extreme when performed on every pair of voxels. Previous studies have focused on small
regions, region-wise causality, or Principal Component Analysis (PCA)
to circumvent the computation issues. On the other hand, I propose
performing full, voxel-level GCM by minimizing the number of unrelated voxels
that are processed, without resorting to linear or Gaussian assumptions.

\bigskip
%\subsubsection*{Proposed Solution}
To minimize the number of non-related voxels that are processed by GCM, I propose
a preprocessing stage that breaks voxels up into independent blocks. Ordinarily
GCM is $O(N^2TD)$ where $N$ is the number of voxels, $T$ is the
number of timepoints, and $D$ is the number of time delays allowed. By breaking
the brain into independent blocks, computation only has to be performed within each
independent block, rather than between every voxel. 
Thus instead of the algorithm being $O(N^2TD)$ it will be $O(P^2TD)$ where $P \ll N$. 
To determine which regions are independent, I propose
the use of mutual information. Unlike PCA or clustering algorithms, 
Mutual Information is based
on Bayesian statistics and makes no assumptions about the underlying distributions.
Moreover, it has recently been demonstrated that mutual information works for larger 
regions of the brain \cite{regionalMI}. Therefore, it is possible to break the
region of interest up into blocks and perform mutual information between the blocks,
rather than individual voxels. Of course there is a trade off between larger blocks
and smaller blocks: larger blocks may miss 1 or 2 voxels that are interdependent with
outside voxels, whereas smaller blocks may be intractable. With the mutual
information between blocks known, a threshold can then be applied, and blocks with 
mutual information combined into a single independent block.
Once a set of independent blocks has been found, GCM may be applied to the internals
of each block without fear of missing voxel-wise causality.

The final step of the pipeline is to build a causation map, which can be displayed
to the user. One inherent problem with any technique measuring causation is 
intermediate terms. For instance, if $A = f(B)$, $B = g(C, D, E, F)$, then
$A$ will appear to be a function of $C$ even if it only a function of $B$.
To combat this problem, and
provide the simplest, most likely network, the algorithm will attempt to account
for intermediate causality. Thus, the algorithm will sever weak connections
when a stronger (albeit longer) path exists. The ultimate output from the algorithm
will be a 
causality graph that can then be simplified for statistical or graphical purposes.
Two potential confounds are regional changes in blood flow and slice timing.
Slice timing will result in causation during resting state, and can be fixed by
accounting for such delays in the model. 
Regional blood flow is more difficult; however, it may be possible to account for
by introducing bias in regions near larger blood vessels. 
Depending on the size of these effect, using time-series subtraction 
may be helpful. 

\bigskip
%\subsubsection*{Conclusion}
Concluding, the proposed method will be a comprehensive set of tools to develop
voxel-wise causality maps from FMRI images. The end goal is
a better understanding of how regions of the brain communicate. 
The scientific benefits of understanding neural connections is manifold, both for
clinical and research purposes. In particular, those with connection disorders 
would benefit greatly from this research.
Of course, understanding why one part of the brain triggers another is 
an important step in understanding sentience.
Many of the techniques I have included here have a history of success in the
investigation of neural networks; thus I believe these goals are feasible. 
By building on the knowledge gained from previous studies it
should be possible to avoid pitfalls, and lay the groundwork for future research.

\small
\bibliographystyle{ieeetr}
\bibliography{savedrecs}
\end{document}
