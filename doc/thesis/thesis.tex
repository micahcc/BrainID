% PROJECT: <ETD> Electronic Thesis and Dissertation Initiative
%   TITLE: LaTeX report template for ETDs in LaTeX
%  AUTHOR: Neill Kipp, nkipp@vt.edu
%     URL: http://etd.vt.edu/latex/
% SAVE AS: etd.tex
% REVISED: September 6, 1997
% 
%\documentclass[12pt,dvips]{report}
\documentclass[12pt]{report}
\setlength{\textwidth}{6.5in}
\setlength{\textheight}{8.5in}
\setlength{\evensidemargin}{0in}
\setlength{\oddsidemargin}{0in}
\setlength{\topmargin}{0in}

\setlength{\parindent}{0pt}
\setlength{\parskip}{0.1in}

%\usepackage[left=3cm,top=2cm,right=3cm]{geometry}

 \usepackage[hang,small,bf]{caption}

\usepackage[printonlyused,withpage]{acronym}
\usepackage{colortbl}
\usepackage{times}
\usepackage[siunitx]{circuitikz}
%\usetikzlibrary{trees}
%\usetikzlibrary{shapes,snakes}
\usepackage{verbatim}
\usepackage{amsmath}
\usepackage{amssymb}
\usepackage{setspace}
\usepackage{graphicx}
\usepackage{subfigure}
\usepackage{cancel}
\usepackage[colorlinks=true,linkcolor=blue,citecolor=blue]{hyperref}
\def\chapterautorefname{Chapter}
\def\sectionautorefname{Section}
\def\subsectionautorefname{Section}

\usepackage{algorithm}
\numberwithin{algorithm}{chapter}
\usepackage{algorithmic}

\newcommand{\R}{\mathbb{R}}
\newcommand{\Q}{\mathbb{Q}}
\newcommand{\N}{\mathbb{N}}
\newcommand{\Z}{\mathbb{Z}}

\begin{comment}

:Author: Micah Chambers
\end{comment}

\begin{document}
\thispagestyle{empty}
\pagenumbering{roman}
\begin{center}

% TITLE
{\Large 
Full Brain Blood-Oxygen-Level-Dependent Signal Parameter Estimation Using Particle Filters
}

\vfill

Micah C. Chambers

\vfill

Thesis submitted to the Faculty of the \\
Virginia Polytechnic Institute and State University \\
in partial fulfillment of the requirements for the degree of

\vfill

Master of Science \\
in \\
Electrical Engineering

\vfill

Chris L. Wyatt, Chair \\
William T. Baumann\\
Aloysius. A. Beex \\
Daniel J. Stilwell

\vfill

August 30, 2010\\
Blacksburg, Virginia

\vfill

Keywords: BOLD Response, FMRI, Nonlinear Systems, Particle Filter, Bayesian Statistics, System Identification
\\
Copyright 2010, Micah C. Chambers

\end{center}

\pagebreak

\thispagestyle{empty}
\begin{center}

{\large Full Brain Blood-Oxygen-Level-Dependent Signal Parameter Estimation Using Particle Filters}

\vfill

Micah C. Chambers

\vfill

(ABSTRACT)

\vfill

\end{center}

Traditional methods of analyzing functional Magnetic Resonance 
Images use a linear combination of
just a few static regressors. This work demonstrates an alternative
approach using a physiologically inspired nonlinear model. By using a 
particle filter to optimize the model parameters, the computation time
is kept below a minute per voxel without requiring a linearization 
of the noise in the state
variables. The activation results show regions similar to those found in 
Statistical Parametric Mapping; however, there are some notable regions not 
detected by that technique. Though the parameters selected by the particle filter based approach
are more than sufficient to predict the Blood-Oxygen-Level-Dependent signal
response,
more model constraints are needed to uniquely identify a single set
of parameters. This ill-posed nature explains the large discrepancies
found in other research that attempted to characterize the model parameters.
For this reason the final distribution of parameters is more medically relevant
than a single estimate. Because the output of the particle filter is 
a full posterior probability, the reliance on the mean to estimate 
parameters is unnecessary. This work presents
not just a viable alternative to the traditional method of detecting
activation, but an extensible technique of estimating the joint probability
distribution function of the Blood-Oxygen-Level-Dependent Signal parameters.

\vfill

% GRANT INFORMATION

%That this work received support from the Southeastern Universities
%Research Association (SURA) ``Monticello Library Project'' is purely
%coincidental.

\pagebreak

% Dedication and Acknowledgments are both optional
% \chapter*{Dedication}
\chapter*{Acknowledgments}

This thesis would not have been possible without the guidance, advice and
patience of my advisor, Chris Wyatt. I knew nothing about being a researcher
or about medical imaging when I came to him, and he taught me to read broadly
and to always dig deeper. Thank you Dr. Wyatt.

I would also like to thank William Baumann for his help and inspiration. I
have learned an incredible amount reading papers with you and Dr. Wyatt.

The Neuroscientists at Wake Forest have also been extremely helpful. In 
particular I would like to thank Paul Laurienti for help gathering fMRI
data and using SPM.

To my parents, for all the psychological (and financial) help along the way,
thank you so much. You were always there for me, I couldn't
ask for better parents. Love you guys.

Finally I would like to thank my friends, for the support, for the prayer,
and, yes, for the occasional stress-relieving game of Halo, SOASE, and every
type of SC.  You guys are the best. 

\tableofcontents
\pagebreak

\listoffigures
\pagebreak

\listoftables
\pagebreak

\huge
\bfseries
List of Acronyms
\mdseries
\normalsize
\begin{acronym}[CMRO2]
\acro{BOLD}{Blood-Oxygen-Level-Dependent}
\acro{CBF}{Cerebral Blood Flow}
\acro{CBV}{Cerebral Blood Volume}
\acro{CMRO2}{Cerebral Metabolic Rate of Oxygen}
\acro{dHb}{Deoxygenated Hemoglobin}
\acro{EPI}{Echo Planar Imaging}
\acro{fMRI}{Functional Magnetic Resonance Imaging}
\acro{GLM}{General Linear Model}
\acro{HRF}{Hemodynamic Response Function}
\acro{Hb}{Hemoglobin}
\acro{MR}{Magnetic Resonance}
\acro{MRI}{Magnetic Resonance Imaging}
\acro{O2Hb}{Oxygenated Hemoglobin}
\acro{RF}{Radio Frequency}
\acro{SNR}{Signal-to-Noise Ratio}
\acro{SPM}{Statistical Parametric Mapping}
\acro{T1}{Longitudinal}
\acro{T2}{Spin-Spin}
\acro{TR}{Repetition Time}
\end{acronym}
\pagebreak

\pagenumbering{arabic}
\pagestyle{myheadings}

\chapter{Introduction}
Traditional methods of analyzing timeseries images produced by 
Functional Magnetic Resonance 
Imaging (FMRI) perform regression using the linear combination of explanatory variables. 
Though adding nonlinear degrees of freedom naturally mandates more computation,
in this thesis I will discuss a Sequential Monte Carlo method of fitting a nonlinear
model a computation cost that would still allow real time calculations for multiple voxels.
More practically, this method is an alternative
method of detecting neural activity from the traditionally
used Statistical Parametric Mapping (SPM). Though more computationally intense,
this method is capable of modeling nonlinear effects, is conceptually simpler
and provides more detailed output. Additionally,
by using a separate particle filter for each single time series it 
is possible to estimate parameters and make real-time predictions
for small neural regions, a feature which could be useful towards real time FMRI 
\cite{DeCharms2005}. Future works will also benefit from the ability to 
apply conditions to the posterior distribution in post-processing without
recalculating parameters; for instance to impose additional 
physiological
constraints. Modeling the BOLD response as a nonlinear system is the
best way to determine the correlation of stimulus sequence with the BOLD
response; yet in the past doing this on a large scale has been far too
computationally taxing. The solution used here takes approximately 40 seconds
for a single voxel's time series (5 minutes in length, with Core 2 Duo Q6600). 

This thesis is organized as follows. In the introduction I will introduce
FMRI, the method by which neural time changing data is detected. This section
will also describe the basic form of the BOLD model - which drives the 
detectable changes in MR signal. \autoref{sec:Prior Works} will discuss other
methods of analyzing FMRI images as well as other techniques that have
been, or could be applied to the nonlinear regression model described here. 
\autoref{sec:Particle Filter} derives the particle filter using Bayesian 
statistics and discusses some practical elements of implementing the 
particle filter algorithm. \autoref{sec:Methods} then goes into further
detail about the specific particle filter configuration used in this work.
This section also describes the pre-processing pipeline. 
The results are described separately for simulated data
and real FMRI data in \autoref{sec:SimulationResults} and \autoref{sec:RealData},
respectively. Finally in \autoref{sec:Conclusion} there is a discussion of
the usefulness and implications of this technique as well as recommendations
for the direction of future works. 

\section{Historic Context}
For the past twenty years, Functional Magnetic Resonance Imaging (FMRI) 
has been at the forefront of cognitive research. Despite its
limited temporal resolution, FMRI is the standard tool for localizing 
neural activation.  Whereas other methods
of analyzing neural signals can be invasive or difficult to acquire, 
FMRI is quick and cheap, and its analysis straight forward.
By modeling the governing equations behind the neural response that
drives FMRI, it is possible to increase the power of FMRI.
The underlying state equations hold important information
about how individual brain regions react to stimuli. The model parameters
on the other hand, hold important information about the patients individual
physiology including existing and future pathologies. In short,
the long chain of events driving FMRI signals contain information 
beyond correlation with stimuli.

In the past fifteen years, a steady stream of studies have built
on the original Blood Oxygen Level Dependent (BOLD) signal 
derivation first described by \cite{Ogawa}.
The seminal work by \cite{Buxton1998} attempted to explain the
time evolution of the BOLD signal using a windkessel model to
describe the local changes in Deoxygenated Hemoglobin content.
Incremental improvements were made to this model until
\cite{Friston2000} brought all the changes together into a single complete 
set of equations. And while there have been numerous adaptations in the model, 
many of them summarized in \cite{Deneux2006}, even the basic versions
have less bias error than the empirically driven \emph{Canonical Hemodynamic Model}
\cite{Deneux2006,Handwerker2004}.
On the other hand BOLD signal models have numbers
of parameters ranging from seven \cite{Riera2004} to 50 \cite{Behzadi2005} 
for a signal as short as 100 samples long. This number of parameters presents
a significant risk of being under-determined and having high computation cost. 
In this work, only the simplest physiologically inspired model will be
used (with 7 parameters), and steps will be taken to make the most of computation
time.

\section{Overview}
\label{sec:Introduction Overview}
Detecting neural activity using the changes in FMRI images is based on 
the Blood Oxygen Level Dependent (BOLD) signal.
The BOLD signal is caused by minute changes in the ratio of Deoxygenated
Hemoglobin to Oxygenated Hemoglobin in blood vessels throughout the brain \cite{Ogawa}.
Because Deoxygenated Hemoglobin (DHb) is paramagnetic, higher concentrations
attenuate the signal detected during T2-weighted Magnetic Resonance Imaging (MRI)
techniques. The most common FMRI imaging technique, due to its rapid repetition 
time (TR), is Echo Planar Imaging (EPI). When axons becomes active,
a large amount of ions quickly flow out of the cell. In order for this
action potential to occur again (and thus for the neuron to fire again),
an active pumping process must move ions back into the
axon. This process of recharging the axon requires extra energy, which temporarily
increases the metabolic rate of oxygen. On a massive scale (cubic millimeter) 
this activation/recharge process happens continuously. However, when a 
particular region of the brain is significantly active, the action potentials
occur more often, resulting in a local increase of the 
Cerebral Metabolic Rate of Oxygen (CMRO2). Thus, if everything else 
stay the same, blood vessels in an active area will 
tend to have less oxygenated hemoglobin, and more deoxygenated hemoglobin
(due to the increased rate at which oxygen is being consumed).
This would then result in an attenuated FMRI signal. However, to
compensate for activation, muscles that
control blood vessels relax allowing increased blood flow,
which actually overcompensates.
This ultimately results in lower than average concentration of 
deoxyhemoglobin. Thus, the BOLD signal consists of a short initial
dip in the MR signal, followed by a prolonged increase in signal
that slowly settles out. It is the overcompensation that provides
the  primary signal detected with FMRI. This cascade of events
is believed to drive a prolonged increased in local metabolism, 
blood flow, blood volume, and oxygenated hemoglobin. The differences
in onsets and recovery times of these variables are what causes the 
distinguishing characteristics of the BOLD signal. Unfortunately, 
FMRI signal levels are all relative: within a particular
person, scanner and run. 

\section{FMRI}
Magnetic Resonance Imaging, MRI, is a method of building 3D images
non-invasively, based on the difference between nuclear spin
relaxation times in different molecules. First, the subject 
is brought into a large magnetic field which causes nuclear spins
to align. Radio Frequency (RF) signals may
then be used to excite nuclear spin away from the base alignment. 
As the nuclei precess back to the alignment of the magnetic
field, they emit detectable RF signals. Conveniently, the
nuclear spins return their original state at different
rates, called the T1 relaxation time, depending on the atoms excited.
Additionally, the
coherence of the spins also decay differently (and roughly an order of 
magnitude faster
than T1 relaxation) based on the properties of the region.
This gives two primary methods of contrasting substances,
which form the basis of T1 and T2 weighted images. Additionally, 
dephasing occurs at two different rates, the T2 relaxation time,
which is unrecoverable, and T2$^*$ relaxation, which is
much faster, but possible to recover from via special RF signals.
T1 relaxation times are typically on the order of seconds if 
a sufficiently strong excitation was applied, whereas T2 relaxation
times are usually less than 100ms. 
In order to rapidly acquire entire brain images, as done in Functional 
MRI, a single large excitation pulse is applied to the entire brain,
and the entire volume is acquired in a single T1 relaxation period. 
Because the entire k-space (spatial-frequency) volume is acquired 
from a single excitation, the signal-to-noise-ratio is low
in EPI. 

Increasing the spatial resolution of EPI necessarily 
requires more time or faster magnetic field switching. Increasing
magnet switching rates can result in
more artifacts and lower signal to noise ratios. The result is
that at best FMRI is capable of 1 second temporal resolution. 
The signal is also diluted because each voxel contains
a variety of neurons, capillaries and veins. 
Thus, the FMRI signal, which is sensitive to the chemical composition of 
materials, is the average signal from various types of tissue
in addition to the blood. As mentioned in \autoref{sec:Introduction Overview},
and explored in depth in \autoref{sec:BOLD Physiology},
the usefulness of FMRI comes from discerning of changes in 
Deoxyhemoglobin/Oxyhemoglobin. Therefore, it is necessary to assume
that the only chemical changes will be in
capillary beds feeding neurons. In practice this may not be the case, for
instance near large veins, and it may explain some of the
noise seen in FMRI imaging (see \autoref{sec:Introduction Noise}. 
Because MRI is unitless and certain
areas will have a higher base MR signal, all FMRI studies deal with
percent change from the base signal; rather than raw values. 

\section{BOLD Physiology}
\label{sec:BOLD Physiology}
It is well known that the two types of hemoglobin act as a contrast agents in 
EPI imaging
\cite{Buxton1998, WEISSKOFF1994, Ogawa}, however the connection
between Deoxyhemoglobin/Oxygenated Hemoglobin and neural activity is non-trivial. 
Intuitively, increased 
metabolism will increase Deoxyhemoglobin, however blood vessels are quick
to compensate by increasing local blood flow. Increased inflow, accomplished by loosening 
capillary beds, precedes increased outflow, driving increased 
blood storage.
Since the local MR signal depends on the ratio of Deoxyhemoglobin to Oxygenated
Hemoglobin, increased blood volume affects this ratio if 
metabolism doesn't exactly match the increased inflow of oxygenated blood.
This was the impetus
for the ground breaking balloon model \cite{Buxton1998} and windkessel
model \cite{Mandeville1999}. These works derive, from first principals,
the changes in deoxyhemoglobin ratio and volume of capillaries given flow waveform.
These were the first two attempts to quantitatively account for the shape of the 
BOLD signal as a consequence of the lag between the cerebral blood volume (CBV) 
and the inward cerebral blood flow (CBF). 

Although \cite{Buxton1998} demonstrated that a well chosen flow waveform could 
explain most features of the BOLD signal, it stopped short of proposing a
realistic waveform for the CBF and CMRO2. In \cite{Friston2000} Friston et. al.
gave a reasonable and simple
expression for CBF input based on a flow inducing signal.
Additionally, in the same work Friston et. al. proposed a simple method
of estimating metabolic rate: as a direct function of the inward blood flow.
By combining these equations with the balloon model from \cite{Buxton1998},
it is possible to predict the BOLD signal directly from a stimulus time course.
\begin{eqnarray}
\dot{s} &=& \epsilon u(t) - \frac{s}{\tau_s} - \frac{f - 1}{\tau_f} \\
\dot{f} &=& s\\
\dot{v} &=& \frac{1}{\tau_0}(f - v^\alpha)\\
\dot{q} &=& \frac{1}{\tau_0}(\frac{f(1-(1-E_0)^f)}{E_0} - \frac{q}{v^{1-1/\alpha}})
\label{eq:bold}
\end{eqnarray}
where $s$ is a flow inducing signal, $f$ is the input blood flow (CBF),
$v$ is normalized cerebral blood volume (CBV), and $q$ is the normalized
local deoxyhemoglobin. The 
parameters controlling blood flow are $\epsilon$, which is a neuronal 
efficiency term, $u(t)$, which is the stimulus, and $\tau_f$, $\tau_s$ 
which are time constants. The parameters for the evolution of blood 
volume are $E_0$ which the resting metabolic
rate and $\alpha$ which is Grubb's parameter controlling the balloon model. 
$\tau_0$ is a single time constant controlling the speed of $v$ and $q$.

This completed balloon model was well summarized again
in \cite{Riera2003}. Ogawa et. al. \cite{Obata2004} refined the readout equation 
of the BOLD signal based on the
deoxyhemoglobin content (q) and local blood volume (v), resulting in the
final BOLD equation:
\begin{eqnarray}
y   &=& V_0((k_1 + k_2)(1-q) - (k_2 + k_3)(1-v))\\
k_1 &=& 4.3 \times \nu_0 \times E_0 \times TE = 2.8\\
K_2 &=& \epsilon_0 \times r_0 \times E_0 \times TE = .57\\
k_3 &=& \epsilon_0 - 1 = .43
\label{eq:boldout}
\end{eqnarray}
Where $\nu_0 = 40.3 s^{-1}$  is the frequency offset in Hz for fully
de-oxygenated blood (at 1.5T), $r_0 = 25 s^{-1}$  is the slope relating
change in relaxation rate with change in blood oxygenation, and
$\epsilon_0 = 1.43$ is the 
ratio of signal MR from intravascular to extravascular regions at rest. Although,
these constants change with experiment ($TE$, $\nu_0$, $r_0$),
patient, and brain 
region ($E_0$, $r_0$), often the estimated values taken from \cite{Obata2004} are 
taken as the constants $a_1 = k_1 + k_2 = 3.4$, and $a_2 = k_2+k_3 = 1$ in 
studies using 1.5 Tesla scanners.
While this model is more accurate than the static Hemodynamic Model used in SPM,
there are other additions which give it more degrees of freedom. 

\section{Post Stimulus Undershoot}
\label{sec:Post Stimulus Undershoot}
Although the most widely used, the BOLD model described in \autoref{eq:bold}
and \autoref{eq:boldout} has been extended in various fashions. The most
significant feature missing from the original model is the 
post-stimulus undershoot.
The post-stimulus undershoot is the term used for a prolonged subnormal
BOLD response for a period of 10 to 60 seconds after stimulus has
ceased \cite{Chen2009,Mandeville1999a}.

Because \autoref{eq:bold} is not capable of producing such a prolonged undershoot,
additional factors must be at play. Two prominent theories exist to explain the post 
stimulus undershoot.  Recall
that a lower than base signal means that there is an increased deoxyhemoglobin
content in the voxel. The first and simplest explanation is that the post-stimulus
undershoot is caused by a prolonged increase in CMRO2 after CBV and CBF
have returned to their base levels. This theory is justified by 
studies that show CBV and CBF returning to the baseline before the BOLD signal
\cite{Frahm2008, Donahue2009, Buxton2004, Lu2004, Shen2008}. 
Unfortunately, because of limitations on FMRI and in vivo
CBV/CBF measurement techniques it is difficult to isolate whether CBF and
CBV truly have returned to their baseline. Other studies indicate
that there can be a prolonged supernormal CBV \cite{Mandeville1999a,
Behzadi2005, Chen2009a}, although none of these papers completely
rule out the possibility of increased CMRO2. The discrepancies may in part
be explained by a spatial dependence in the post-stimulus undershoot; described
by \cite{Yacoub2006}. In \cite{Chen2009}
a compelling case is made that most of the post stimulus undershoot can be 
explained by combination of a prolonged CBV increase, and a prolonged CBF 
undershoot, and that
the previous measurements showing a quick recovery of CBV 
were in fact showing a return to baseline by arterial CBV (which
has little effect on the BOLD signal).

Regardless of the probability that CMRO2 and CBF are detached,
research into the post-stimulus undershoot has led to the creation
of much more in depth models. In \cite{Zheng2002} additional state
variables model oxygen transport, whereas \cite{Buxton2004} models
CMR02 from a higher level, and somewhat more simply; though it 
still adds 9 new parameters. \cite{Behzadi2005}
introduces nonlinearities into the CBF equations as a method to
explain the post-stimulus undershoot, which falls in line with a 
prolonged increase in CBF observed in \cite{Chen2009}. Similarly
\cite{Zheng2005} adds additional compartments to model 
venous and arterial blood. 
In \cite{Deneux2006} Deneux et. al. compared these models and though 
that work did not deal extensively with the 
post-stimulus undershoot, it did show incremental improvements
in quality from the additional parameters. 
Importantly, \cite{Deneux2006} did show that by 
simply adding viscoelastic terms from \cite{Buxton2004}, a slowed return 
to baseline is possible to model, without greatly increasing
complexity. Regardless, because these models are more 
complex, and the parameters are not well characterized, in this work the simple
Balloon model is used. 

In summary, there have been extensive refinements to the Balloon
model; however, the increased complexity and lack of known priors 
make these models difficult to work with. Additional degrees of freedom 
could also make parameter estimation intractable.

\section{Properties of the BOLD Model}
\label{sec:BOLD Analysis}
Since the first complete BOLD model was proposed by \cite{Friston2002}, 
several studies have analyzed its properties. 
The most important property is that the system is dissipative, and given
enough time will converge to a constant value. This is found simply by
analyzing the eigenvalues of the state equation Jacobian, 
\cite{Deneux2006, Hu2009}. The steady state of the Balloon
model equations gives:

\begin{eqnarray}
s_{ss} &=& 0 \nonumber \\
f_{ss} &=& \tau_f\epsilon u + 1\nonumber \\
v_{ss} &=& (\tau_f\epsilon u + 1)^\alpha\nonumber \\
q_{ss} &=& \frac{(\tau_f\epsilon u + 1)^\alpha}{E_0}(1-(1-E_0)^{1/(\tau_f\epsilon u + 1)})\nonumber \\
y_{ss} &=& V_0((k_1+k_2)(1-q_{ss}) - (k_2+k_3)(1-v_{ss}))
\label{eq:steadystate}
\end{eqnarray}

where the parameters are all the same as in \autoref{eq:bold}

In real FMRI data, there is a significant nonlinearity in response; with short sequences
responding disproportionately strongly \cite{Birn2001, Wager2005, Deneux2006}.
This nonlinearity is accounted for in the Balloon model, although \cite{Deneux2006}
shows that when duration of stimuli varies greatly,
modeling Neural Habituation is necessary to fully capture the range of responses. 
In both \cite{Birn2001} and \cite{Deneux2006} it was found that 
stimuli lasting longer than 4 seconds 
tend to be more linear, which is why block designs are so well accounted for
by the General Linear Model (see \autoref{sec:Current Techniques General Linear Model}).

Another interesting result of \cite{Deneux2006} was the sensitivity analysis.
There it was found that the parameters are far from perpendicular,
and that very different parameters could give nearly identical BOLD output.
The means that that without constraining parameter values, they may not be 
precisely ascertainable. This could explain discrepancies in previous studies
(\autoref{tab:Params}).

%note to self, friston2002b's parameters are from a picture
\begin{table}[t]
\centering
\begin{tabular}{|c || c | c | c | c|}
\hline 
Parameter  & \cite{Friston2000} & \cite{Johnston2008} & \cite{Vakorin2007} & \cite{Deneux2006}\\
\hline
$\tau_0  $ &  $N(.98 , .25^2)$  & $8.38 \pm 1.5  $ & $.94$ & .27\\
$\alpha  $ &  $N(.33 , .45^2)$  & $.189 \pm .004 $ & $.4$ (NC) & .63 \\
$E_0     $ &  $N(.34 , .1 ^2)$  & $.635 \pm .072 $ & $.6$ (NC) & .33\\
$V_0     $ &  $.03$ (NC)        & $.0149 \pm .006$ & (NC) & .16\\
$\tau_s  $ &  $N(1.54, .25^2)$  & $4.98 \pm 1.07 $ & $2.2$ & 2.04 \\
$\tau_f  $ &  $N(2.46, .25^2)$  & $8.31 \pm 1.51 $ & $.45$ & 5.26\\
$\epsilon$ &  $N(.54 , .1 ^2)$  & $.069 \pm .014 $ & (NC) & .89\\
\hline
\end{tabular}
\caption{Parameters found by various studies. (NC) indicates that the value
wasn't calculated. \cite{Vakorin2007} made use of the values from \cite{Friston2002}
where not explicitly stated}
\label{tab:Params} 
\end{table}



\chapter{Current Techniques}
The ultimate purpose of this work is to provide a new set of tools
for analyzing FMRI data. Whereas esisting techniques have been 
successful at finding macroscopic regions of activation, :hese
techniques are not robust to significant noise and thus linear 
modeling carries a significant bias error due to lack of model
flexibility. While adding parameters can significantly increase
error due to model variance, this effect should be mitigated first
by the use of a highly constrained model based on first principals
and also by the
calculation of a full posterior distribution, rather than a single
estimate. The
purpose of this paper is thus to evaluate the potential of using
a particle filter along with the BOLD model to derive physical 
parameters. In so doing, we hope to be able to show that one or more
parameters are a suitable replacement for estimating voxel 
activation from a standard FMRI image. We also hope to show that 
estimated posterior distribution of the parameters, derived from
the particle filter, is able to provide an accurate measure of the
confidence interval.

\section{Previous Studies of Parameters}
There have been quite a few efforts to quantify the parameters of the
various BOLD models.  Although \cite{Buxton1998} and \cite{Friston2000}
both proposed physiologically reasonable values for the model parameters, 
\cite{Friston2002} was the first paper to calculate the parameters based 
on actual FMRI data. In that paper, Friston et. al. used a variation of
Expectation Maximization to find a normal distribution for the parameters:
\begin{eqnarray}
\epsilon &=& N(.54 , .1 ^2 )  \nonumber \\
\tau_s & =&  N(1.54, .25^2)   \nonumber \\
\tau_f & =&  N(2.46, .25^2)   \nonumber \\
\tau_0 & =&  N(.98 , .25^2 )   \nonumber \\
\alpha & =&  N(.33 , .45^2 )   \nonumber \\
E_0   & =&  N(.34 ,  .1 ^2 )   \nonumber \\
V_0  & = &  .03 (not\ estimated) \nonumber
\end{eqnarray}
Since then, several other methods of have been used to calculate
BOLD parameters from FMRI timecourses. In \cite{Riera2004}, a maximum
likelihood method for innovation processes was used, as described by
\cite{Ozaki1994}. \cite{Ozaki1994} uses a similar construction to a 
Kalman filter, to break the time series into a series of innovations,
for which Maximum Likelihood was performed. While this is in some since the
"right" way to find the solution, it comes with several caveats. First, every
step in parameter space requires a recalculation of all the state variables. With
two or three parameters this is fine, more than that, and calculations could go on
indefinitely. Second, it still assumes the parameters and noise are Gaussian, and
will only be optimal in that case. Third, depending on the nonlinearities present
in the system, local minima may be extremely common. Later \cite{Hu2009} used an 
Unscented Kalman Filter over all the parameters and state variables to find the 
parameter set/variable time series. While this method has the drawback of not necessarily
being optimal, unfortunately there is no general optimal solution to non-linear non
Gaussian models. Hu et. al.'s technique also will run significantly faster than
ML based techniques, since it does not require recalculating the entire timeseries
for every step in parameters. Both \cite{Hu2009} and \cite{Friston2002} came to results 
very similar to the expected values stated in \cite{Buxton1998} and \cite{Friston2000}.
One potential problem with all these techniques are that the depend heavily on the
priors. The starting point of the parameters could have a huge impact on the
results, and while one can be hopeful that this isn't the reason for the agreement
between \cite{Friston2000} and later results, there is no way to know. 

In \cite{Johnston2008}, a hybrid particle filter/gradient
descent algorithm was used to simultaneously derive the static (classicly called
parameters) and dynamic parameters (classically known as state variables).
Essentially a particle filter is used to calculate the state variables at some
time, and then the estimated distribution of the particles was used to find
the most likely set of parameters that would give that distribution of state variables.
\cite{Johnston2008} comes to a very different set of parameter estimates as compared
to the original \cite{Friston2000} guesses:
\begin{eqnarray}
epsilon &=& .069 \pm .014    \nonumber \\
tau_s & =& 4.98 \pm 1.07  \nonumber \\
tau_f & =& 8.31 \pm 1.51   \nonumber \\
tau_0 & =& 8.38 \pm 1.5    \nonumber \\
alpha & =& .189 \pm .004   \nonumber \\
E_0   & =& .635 \pm .072     \nonumber \\
V_0   & =& .0149 \pm .006     \nonumber 
\end{eqnarray}
Notably, the time constants are significantly longer. This could be a result of
some preprocessing to the stimulus timeseries performed in \cite{Friston2002} and
later works but not in \cite{Johnston2008}, or it could be that \cite{Johnston2008}
depended less on the priors. 

Its possible, although unlikely that the balloon model is not possible to learn
without detrimental cost in variance error. I do not think this is the case, however,
and I will work to dispell this possibility in the results with simulated time
series.

\section{Basic Statistical Parametric Mapping}
Although not strictly the same thing as parameter calculation from 
FMRI, activation detection is very similar. In fact, estimation of 
parameters is somewhat a generalization of the idea of activation detection.
Therefore, it is important to draw a comparison between the methods proposed
in this thesis with existing methods of activation detection.

The most basic method of analyzing FMRI data is through a standard T-test
between "resting state" and "active state" samples. This is done by 
taking the average and variance of the inactive period, and the 
period during which the stimulus was activate separately then treating 
them both as gaussian distributions.
If they are in fact Gaussian distributions, then a basic t-test will
give the probability that the samples came from the same distribution
(the null hypothesis). Of course, this test is fraught with problems; even if
the drift mentioned earlier has been removed, there is little reason
to believe that the noise is Gaussian, or even stable. Additionally, 
even if the noise were Gaussian, a t-test with a p-value of .05 over
50000 or more samples is on average going to generate $.05*50000$ false
positives. To compensate for this, bonferoni correction, also known as
multiple comparison tests are performed; essentially p-values are 
divided by the number of independent
tests being run. This, however, leads to extremely low p-values, so
low that it would be impossible for any biological system to satisfy. To
compensate, a Gaussian kernel is applied to the image, thus reducing
variance (and thus separating the active and inactive distributions)
as well as decreasing the effective number of voxels. Since t-tests are
now no longer being applied to n <I need to define n> independent voxels,
the factor by which the p-value must be divided by can be decreased.
<Do I need to mathematically define all this?> The derivation and application
of random field theory, and its use can be found in various papers \cite{Worsley2004}.

\subsection{General Linear Model}
\label{sec:Current Techniques General Linear Model}
The most used form of FMRI analysis is Statistical Parametric
Mapping, but is able to account for several different levels or types
of stimulus (see \cite{Hofmann1997}). By using hierarchical models
the output signal timeseries is considered the weighted
sum of the various input timeseries. Essentially every experimental factor
is considered as another level inputs (ex. different type of stimulus,
different run with the same patient, different patient). 
The equation for a general linear model is then
\begin{equation}
Y(t) = X(t)\beta + \epsilon(t)
\end{equation}
where $Y(t)$ is the smoothed or detrended timeseries of measurements,
$X(t)$ is a row vector of input, $\beta$ is a column vector of weights,
and $\epsilon$ is the error. Thus for every time, the measurement is
assumed to be a weighted sum of the columns of $X$ plus some error. The calculation
of $\beta$ is then performed using a maximum likelihood or gradient descent search 
to minimize the error.

\begin{figure}
\caption{GLM todo}
\label{fig:GLM}
\end{figure}

It is well known of course that a square wave stimulus does not result in a square wave
in the activation of brain regions. Thus, various methods are used to 
smooth the columns of $X$, and thus bandlimit the input. 
The best technique is convolving the stimulus input with a Hemodynamic 
Response Function (HRF), which mimics the basic shape of BOLD activation, including a delay
due to rise time and fall time. The downside of this method, however is that 
the shape of the Hemodynamic Signal is static, meaning the same Hemodynamic Function is
used for every region of the brain, and $Y(t)$ must linear combination 
of scalar the columns of $X$ to be appropriately identified. Additionally, 
it is well known that different Hemodynamic Response Functions are necessary for different 
regions of the brain. The "Canonical" HRF that is most used, has been fitted
for the visual cortex. Thus the shape of the HRF may differ significantly in
terms of onset and fall time in other areas of the brain \cite{Handwerker2004}. 
The inability to
fit parameters other than scale certainly hinders SPM's
ability to locate voxels that do not conform to the "canonical" hemodynamic 
response function. If there are different HRF's for different regions 
of the brain, might there also be more subtle differences even within
a particular region? It seems reasonable to think so. As a result, the 
most common use of SPM will be heavily biased toward the visual cortex.

\begin{figure}
\caption{Hemodynamic Response Function todo}
\label{fig:HRF}
\end{figure}

The GLM is extremely powerful at determining the linear dependence of
a set of regressors on the output. Unfortunately, there is significant evidence
that many of these dependencies are nonlinear, which means they may be
difficult or impossible to detect using linear techniques. This presents
a significant and unknown problem that is often left un-addressed in 
neural studies.

A static linear model is also unable to incorporate other forms of physiological
data that may be available. Combined FMRI
CBF or CBV imaging methods are not uncommon and they could shed much light on
neural activation. However, there is no way of actively combining that data 
into a static HRF. The fact of the matter is that the relation between the 
BOLD signal and CBV/CBF simply cannot be described by any linear relationship.
While it may be possible to determine some regionally varying set of parameters
for the HRF based on CBV/CBF, the lack of physiologically inspired parameters
impairs this. The importance of regional differences in the HRF cannot be
over-stated; it is well known that capillary beds are far from uniformly distributed
and thus blood perfusion and regional oxygenation cannot possibly be uniform
either. 

Finally, all these techniques require the noise to be Gaussian to reach
an optimal solution. In fact there is no known optimal solution to
nonlinear models with non-Guassian noise, so obviously some assumptions
are going to have to be made to reach a solution. However, it would be
nice to have an algorithm that is robust to these effects, and could
still give a good solution when Gaussianity is violated.

Because of these limitations, its entirely possible that activation exists
in regions that SPM doesn't detect, but because the activation does not conform
to the set HRF, it is impossible for these signals to be detected. Perhaps
because of this, it is not uncommon for data to be thrown out or averaged
together in FMRI studies because 
no significant activation could be seen in a single run \cite{Riera2004}
\cite{Johnston2008}. This practice highlights 
limitations in strictly linear approaches; and suggests that a single HRF is 
insufficiently flexible to account for relatively common variations of neural activation
\cite{Handwerker2004}. 

In general activation detection type methods also don't have the ability 
to find pathologies based on state variables or parameters.  It
is quite possible that physical properties such as decreased compliance of
blood vessels could indicate a neurological condition that is not easily
seen in a T1 or T2 map. In essence, this could make FMRI a much more 
useful clinical tool than it is now. 

\section{Nonlinear Least Squares}
\label{sec:Nonlinear Least Squares}
Rather than localizing activation, by using the physiologically
plausible BOLD model, it is possible to determine the values of
governing parameters.

Although there are certainly benefits to using a derived model, rather
than a purely empirical model, there are serious implications. The
first problem is that all the powerful classical techniques of gradient
descent are off limits; since the model is a true nonlinear dynamical
system with no closed form solution. The implication of this is that
the calculation of a Jacobian for residuals won't work; and thus powerful
techniques such as the Gauss-Newton method, which are helpful in many nonlinear
problems, are off limits. Additionally, a gradient descent is difficult to
perform without the ability to form partials of the output with respect
to all the parameters. 

Although anything requiring a Jacobian is out, there are other heuristic
techniques that could potentially illuminate the BOLD response. 
Simulated Annealing (SA) is a common method of optimizing high dimensional
regression problems. The idea is to pick a random start, and then
at each iteration pick a random nearby point, and if that point is
below some energy constraint (energy is a function of the residual), 
called the temperature, the algorithm moves
to that point and continues with the next iteration. The temperature
is slowly lowered until no nearby points below the temperature can
be found (or the temperature drops below the current point). There are
variations of this, for instance it is common to require every movement
to be in the downward direction (in terms of energy). Like most nonlinear
optimization problems, there is no guarantee of an optimal solution,
although the longer the algorithm is allowed to run, the better the solution
will get. Since every step requires an entirely new run of the 
BOLD model, it can be extremely time consuming, which is why we are
not using it here.

\begin{algorithm}
\caption{Simulated Annealing Algorithm}
\label{alg:Simulated Annealing}
\begin{algorithmic}
\STATE Initialize $\Theta$, or if there exists a decent estimate start there
\STATE Initialize temperature, T to value above initial energy
\WHILE{$E(\Theta) < T$}
    \REPEAT
        \STATE Pick $\theta$ near $\Theta$
        \STATE Calculate energy, $E$, of $\theta$
    \UNTIL{$E > T$}
    \STATE Move to new estimate: set $\Theta = \theta$
\ENDWHILE
\end{algorithmic}
\end{algorithm}

[simulated annealing image?]

Another potential method of interest is the use of Genetic Algorithms
(GA). Genetic algorithms are similar to Simulated Annealing, in
that they randomly move to better solutions based on a cost function.
However; in genetic algorithms a single point estimate isn't used. Instead
a population of estimates is generated, each with distinct parameters,
and then each set of parameters is rated with a fitness function. Parameter
sets that are good get a higher "fitness"; then new parameter sets are generated by 
randomly combining pieces of the old parameter sets. The pieces are typically
chosen at a rate proportional to the fitness of the donor; thus "fit"
parameter sets tend to pass on their properties. In addition to this,
random mutations may be introduced that come from no existing parent. 
The new generation is then rated with the fitness function again, and
the entire process starts over. The stop condition for a genetic algorithm
is typically based on some threshold for fitness or a maximum number 
of generations. This is problematic, since it doesn't really provide 
any guarantee of even reaching a local minima, although in practice
it can be quite effective.

[genetic algorithm picture]

\begin{algorithm}
\caption{Genetic Algorithm}
\label{alg:Genetic Algorithm}
\begin{algorithmic}
\STATE Initialize $N$ estimates, $E = \{\Theta_0, \Theta_1, ... \Theta_N\}$
\FOR{G generations}
    \STATE Calculate fitness for each $\Theta$, Ex. for residual $R$, $1/R$ or, $e^{-R}$
    \FOR{$i$ in $N$}
        \STATE Randomly select two parents (with higher probability for more fit $\Theta$'s)
        \STATE Randomly merge parts of the two parents to form a new $\Theta_i$
        \STATE At some low probability change one or two parameters in $\Theta_i$
    \ENDFOR
\ENDFOR
\end{algorithmic}
\end{algorithm}

Although both these methods can be highly effective, they have the downside of
requiring very high computation time. In this case of the BOLD model,
each time the energy or fitness needs to be calculated, a large number of cycles
must be spent re-simulating the BOLD model for the set of parameters. As we'll
discuss in \autoref{sec:Particle Filters}, the Particle Filter method is able
to circumvent this re-calculation to some degree.

It would not be unreasonable at this point to back off and work with a linearized
or more static version of the model. This is the approach taken by the
standard General Linear Model discussed in \autoref{sec:Current Techniques General Linear Model}.


\section{Unscented Kalman Filter}
Although the classical methods mentioned in \autoref{sec:Nonlinear Least Squares}
won't work for the nonlinear model presented in \autoref{sec:BOLD Physiology},
there is another Bayesian technique that is worth considering.
The Unscented Kalman Filter (UKF) is a powerful Gaussian/Bayes filter that attempts
to model the posterior distribution of dynamical systems as a multivariate
Gaussian. The Unscented Kalman Filter (UKF) generalizes the Extended Kalman
Filter by allowing the state update to be a function, $g$,

\begin{eqnarray}
X(t) &=& g(u(t), X(t-1))\\
Y(t) &=& h(X(t)
\end{eqnarray}

In order to estimate the posterior at $t$, a deterministic set of "sigma" points 
(often 2 per dimension, plus 1 at the mode of the multivariate distribution)
weighted according to a Gaussian estimate of $X(t-1)$ are passed through
the update equation. This set of points are then used to estimate the 
mean and covariance of $X(t)$. The benefit of this, is that it requires
no Jacobian and only a few extra calculations to get a decent esimate of
a posterior Gaussian. In the BOLD case, the set of equations we are modeling have no 
closed form solution, and finding the Jacobian is impossible without approximations. Although 
\cite{Riera2004}, \cite{Hu2009} mention a Jacobian of the BOLD response, this is
not strictly the case and is rather $\frac{\partial J}{\partial t}$
rather than a true Jacobian.  This is important because the Extended
Kalman filter depends on the Jacobian to map a Gaussian through the 
advancement of time. Thus the popular Extended Kalman Filter won't work
in this case, whereas the Unscented Kalman Filter still does. In fact
\cite{Hu2009} uses the UKF to perform a similar type of analysis to
the one performed in this work. 

\begin{figure}
\caption{Examples of update distributions, using an Kalman Filter, \cite{Thrun2005}}
\label{fig:EKFWorking}
\end{figure}

The difficulty
of using a Kalman Filter, however, is that it assumes a multivariate 
Gaussian for the state variables, $X(t-1)$. The problem with this is that 
when a system is significantly nonlinear, the state at 
$X(t)$ will almost certainly be non-Gaussian, and thus estimating
$X(t+1)$ with $X(t)$ as a multivariate Gaussian will perpetually introduce
error in the distribution. Furthermore, it is not really known what 
sort of underlying distributions may exist in such a mixed biological,
mechanical, chemical system such as the brain. Assuming the parameters
all to be Gaussian may in fact be a gross error. On the other hand, for
small variances and short time steps the gaussiann distribution is a good 
fit, and so in some limited cases the Unscented Kalman Filter could work
well. These are 
non-trivial issues given that the assumption of Gaussianity is what allows
the UKF to estimate the posterior using only the first and second moments;
two parameters that don't uniquely describe most distributions.

\begin{table}[t]
\centering
\begin{tabular}{|c || c |}
\hline 
Parameter & Run 1 \\
\hline
$\tau_0$ & .98  \\
$\alpha$ & .33 \\
$E_0$ & .34  \\
$V_0$ & .03  \\
$\tau_s$ & 1.54  \\
$\tau_f$ & 2.46  \\
$\epsilon$ & .54  \\
$V_t$ & N(1, .3)  \\
$Q_t$ & N(1, .3)  \\
$S_t$ & N(1, .3) \\
$F_t$ & N(1, .3) \\
\hline
\end{tabular}
\caption{Parameters used to test Gaussianity of variables after being transitioned through
the BOLD model}
\label{tab:steptable} 
\end{table}

To determine the amount of error incurred in a Gaussian estimate during
a typical sample period, the states of BOLD equations were assigned according
to four dimensional Gaussian. The states were then propagated through two
seconds of simulation (a typical TR in FMRI) and then the resulting
marginal distributions were compared with a Gaussian distribution. The 
purpose is to determine the degree to which the results of simulation will
result in non-Gaussian output, given a Gaussian input. This also demonstrates
the degree of nonlinearity present in the system. The parameters used are 
shown in \autoref{tab:steptable}

Notably $s_t$ has intentionally been set to a non-equilibrium, but physiologically
plausible value. The value of $u$ is left at zero the entire time, so the 
system will decay naturally (see \autoref{sec:BOLD Physiology}), though initializing 
$s$ at a non-zero level will drive the system for several seconds. 
\autoref{fig:transp1s} shows the results when the system is essentially
left on for 100 milliseconds after setting the variables according to \autoref{tab:steptable}.
The Q-Q plots fit very will with a Gaussian, demonstrating that at this short
time interval nonlinearities have not yet begun to effect the distribution.
However, \autoref{fig:trans1s} us the result after 1 second, which is faster
than most FMRI scanners are capable of sampling at. At that range the tails 
of the distributions for $v$ and $q$ are clearly starting to deviate from the
Gaussian distribution. As a result the uncertainty in $y$ is deviating from
the Gaussian distribution as well. This is important, because although 
approximating the distribution with a Gaussian based on the first two
moments will work in the short run, there will be residual error in the
distribution. 

On the other hand, this effect is more limited if the initial variance is somewhat
smaller. In tests with those cases, it took much longer for the nonlinearities to skew
the distribution. That result could be encouraging to those looking to use the
UKF, if the distributions are kept relatively thin.

Another potential problem with the UKF is that typically the sigma points,
used to estimate the posterior probability, $P(X(t) | X(t-1), u(t))$, are
located on the main axes. As a result, the covariances are not allowed to 
be effected by the state transition the same way the variances are. While
this may be reasonable in low-dimensional systems, high dimensional systems
have a much greater potential for interplay between the variables. While
this problem is relatively easy to fix (by using more sigma points off
the main axes), there is a definite cost in complexity.

Ultimately, there is a distinct possibility that the nonlinearities of 
the BOLD model make Gaussian estimates unrealistic and thus less effective.
More advanced tests where static variables such as $\alpha$ are varied
as well could shed even more light on the issue. The trouble with using
the UKF then to estimate parameters is that all eleven members of $X$ would
be treated like a single joint Gaussian distribution which certainly 
compound the issues of nonlinearity seen in \autoref{fig:transp1s} and 
\autoref{fig:trans1s}

\begin{figure}
\centering
\includegraphics[trim=6cm .75cm 6cm .75cm,width=16cm]{images/gauss_step_point1sec_3sigma.pdf}
\caption{Distributions of state variables after simulating for .1s}
\label{fig:transp1s}
\end{figure}

\begin{figure}
\includegraphics[trim=6cm .75cm 6cm .75cm,width=16cm]{images/gauss_step_1sec_3sigma.pdf}
\caption{Distributions of state variables after simulating for 1s}
\label{fig:trans1s}
\end{figure}

\section{Variational Filtering}
\cite{Friston2008}, \cite{Friston2008b}

\section{Hybrid Methods}
A large number of hybrid methods have been tried, in order to elicit 
parameter estimates from the BOLD model. \cite{Vakorin2007},
\cite{Johnston2007}, \cite{Hu2009}

\section{Conclusion}
In summary, several other approaches could be taken 

\chapter{Particle Filters}
\label{sec:Particle Filter}
\section{Introduction}
Particle filters, a type of Sequential Monte Carlo (SMC) method,
are a powerful method for estimating the posterior probability distribution
 of parameters given a timeseries of measurements. Unlike Markov 
Chain Monte Carlo (MCMC) estimation, particle filters are designed for 
time-varying random variables. The idea behind particle filters is
similar to Kalman Filters; however, unlike Kalman Filters,
distributions are stored as an empircal distribution rather than the
as the first two moments of a Gaussian. Thus, particle filters are 
preferred when the model is nonlinear, and non-gaussian. This section
is largely based on \cite{Arulampalam2002a} and \cite{Thrun2005}.

\section{Model}
\label{sec:Particle Filter Model}
The idea of the particle filter is to build an empirical distribution
out of a large number of parameter sets, called particles. Each
particle contains all the parameters and states needed to propagate
the model forward.  The particle filter begins with a wide distribution 
(called the Prior Distribution)
of possible particles and then, as measurements come in, weights 
particles based on the quality of their output estimates. Thus parameter sets 
that tend to give good estimations of the measurements get weighted higher
than parameter sets that give poor estimates. Although the reliance on
a prior distribution is often troublesome, when the system being modeled
has physical meaning, establishing reasonable ranges for parameters may be 
quite easy. Optimizing the prior distribution can be more difficult,
unless the system has been extensively studied.

Suppose a set or stream of measurements at discrete times are given, 
$\{Y_k, k = 1, 2, 3, ... K\}$, where $K$ may be infinite. 
Because $k$ is a discrete time, let $t_k$ define the continuous
time of $k$.
Suppose also that there is a hidden set of state variables,
$X(t)$ that drives the value of $Y(t)$. Throughout this section
with $X_k = X(t_k)$. The goal of the particle filter is to estimate the 
distribution of the
true parameters $\Theta$ that dictates the movement of $X(t)$.
The model also permits random motion in $X(t)$, so the 
particle filter also estimates the distribution of $X(t)$.
The only difference between the members of parameter vector
$\Theta$ and those of $X(t)$ is that the memebers of
$\Theta$ have no known update equation. Members of both vectors
are permitted to have some noise, although this
may not be explicitly stated in the model. The generic, continuous, nonlinear
system definition is shown in \autoref{eq:GenericNonlinear}.

\begin{eqnarray}
\dot{X}(t) = f(t, X(t), u(t), \theta, \nu_x) \nonumber \\
Y(t) = g(t, X(t), u(t), \theta, \nu_y)
\label{eq:GenericNonlinear}
\end{eqnarray}

$X(t)$ is vector of state variables, $\Theta$ is a vector of system
constants, $u(t)$ is an input, $Y(t)$ the observation, and
$\nu_x$ and $\nu_y$ are random variates. Although any of these
variables could be a vector, for the sake of simplicity only
$\Theta$ and $X(t)$ will be considered as such. 

I will also make a few  simplifying assumptions for this work. 
First, the systems are assumed to be 
time invariant. This 
assumption is based on the idea that if you paused the system for $\Delta t$
seconds, when unfrozen the system would continue as if nothing happened. 
Few biological systems are predictable enough for them to be summarized
by a time varying function, least of all the brain. While heart beats are certainly
periodic and have an effect on the BOLD signal, the period varies too much
for the system to be considered varying with time. 
Next, its assumes that input cannot directly
influence the output, which in the case of the BOLD signal is a good assumption.
Finally, because the only difference between the members of $X(t)$ and 
$\Theta$ is an update function, from now on $X$ will contain 
$\Theta$. The assumptions now allow for a simplified version of the
state space equations:

\begin{eqnarray}
\dot{X}_k = f(X_{k-1}, u_k, \nu_x)
\label{eq:stateass}\\
Y_k = g(X_k, \nu_y)
\label{eq:measass}
\end{eqnarray}

\section{Sequential Importance Sampling}
The goal of the particle filter is to evolve an empirical distribution 
$P(x_k | u_{0:k}, Y_{0:k})$,
that asymptotically approaches the true probability distribution $P(X_k | u_{0:k})$.
Note that capital $X$ will be used as the actual realizations of 
the state variable, whereas $x$ will denote estimates of $X$.
Additionally, the notation $a:b$ indicates the set $[a,b]$,
as in $u_{a:b}$, which would indicate all the inputs from time $a$ to time $b$.
Considering the noise present in $X$,
 $P(X_k | u_{0:k})$ is not a single true value but probability distribution. 

To begin with, the particle filter must be given a prior distribution, from
which the initial $N_p$ particles are drawn. A particle contains a weight
as well as an estimate of $X_k$, which as previously mentioned, contains every
variable needed to run the model. Then the prior is generated from a 
given distribution, $\alpha(X)$, by:

\begin{equation}
\{[x^i_0,w^i] : x^i_0 \sim \alpha(X), w^i = \frac{1}{N_p}, i \in \{1, 2, ... , N_p\} \}
\end{equation}

Where $N_p$ is the number of particles or points used to describe the prior 
using a Mixture PDF. 
Note that any exponents will be explicitly labeled as such, to avoid confusion with
the particle numbering scheme. 

Therefore, after the particle have been generated they should approximate $\alpha(X)$:

\begin{equation}
\alpha(X) \approx P(x_0) = \sum_{i=0}^{N_p} w^i\delta(X - x^i_0 ) dx
\end{equation}
Where $\delta(x-x_0)$ is 1 if and only if $x = x_0$ (the Kronecker delta function).

If a flat prior is 
preferred, then each particle's weight could be scaled to the reciprocal of the
density at the particle: 
\begin{equation}
w^i = \frac{1}{\alpha(x^i_0)}
\end{equation}
Whether or not to flatten the prior is a design decision. The reason this might 
be preferred over a direct
uniform distribution is that the distribution width will inherently scale 
for increased particle counts although some distributions
flatten out better than others. Either way, $\alpha(X)$ \emph{must} be
wide enough to incorporate any posterior that arises. If the prior is
not sufficiently dense, the particle filter can compensate, if it is
not sufficiently wide the particle filter won't converge. 

\subsection{Weighting}
For all the following areas, the probabilities implicitly depend on $u_{0:k}$, 
so those terms are left off for simplicity.

Whenever a measurement becomes available it permits refinement of the
posterior density.
This process of incorporating new data is called sequential importance sampling,
and eventually allows convergence. The weight is defined as
\begin{equation}
w^i_k \propto \frac{P(x^i_{0:k} | y_{0:k})}{q(x^i_{0:k} | y_{0:k})}
\label{eq:weightfunc}
\end{equation}
where $q$ is called an \emph{importance density}. The importance density
is the density of the points, thus by dividing by this value, the weight
should not depend on the location of the estimation points, but rather
only on $P(x^i_{0:k} | y_{0:k})$, the probability of that particle
being correct given all the measurements up to time $k$. 
Note that if there is a far off peak in
the posterior that $q$ does not have support points in, there will 
be quantization errors, and that part of the density cannot be modeled. This is why
it is absolutely necessary that $q$ fully covers $P(x^i_{0:k} | y_{0:k})$.

It is helpful
to consider how the importance density affects the initial distribution. 
In the initial distribution, the weights are all the same; and for
the sake of argument, let them all be scaled up to 1. Then
\begin{equation}
w^i_k q(x^i_{0:k} | y_{0:k}) = q(x^i_{0:k} | y_{0:k}) = P(x^i_{0:k} | y_{0:k})
\end{equation}
the estimated probability, $P(x^i_{0:k} | y_{0:k})$ depends only on the 
way the particles are distributed. As new measurements are incorporated,
the weight will accumulate probabilities through time, which will be discussed
next. 

\subsection{Calculating Weights}
To calculate the weight of a particular particle, it is necessary to 
calculate both $q(x^i_{0:k} | y_{0:k})$ and $P(x^i_{0:k} | y_{0:k})$.
Note that $q(x^i_{0:k} | y_{0:k})$ may be simplified by assuming that 
$y_k$ doesn't contain any information about $x_{k-1}$. Technically this 
could be false; since later measurements may shed light on currently hidden
changes in $x$. For practical applications though it is a helpful assumption.
\begin{equation}
q(x^i_{0:k} | y_{0:k}) = q(x^i_{0:k} | y_{0:k-1})
\label{eq:QAssump}
\end{equation}
The choice of the importance density is another design decision; however
it is common to use the integrated state equations. 
Although other importance density functions exist; for the particle filter
used here, the standard importance density will be used: the model
prior.
\begin{equation}
q(x_k | x_{k-1}, y_{0:k}) =  P(x_k | x_{k-1})
\label{eq:ImportanceDensity}
\end{equation}
The benefit of this choice for importance density is that an approximation for
$P(x_k | x_{k-1})$ is freely available: its simply the set of particles propagated
forward in time using the state equations. Additionally it makes
updating weights simple, as seen in \autoref{eq:weightevolve}.

The $q(x^i_{0:k} | y_{0:k})$ may then be simplified:
\begin{equation}
\begin{array}{cclr}
q(x_{0:k} | y_{0:k}) & = & q(x_k | x_{0:k-1}, y_{0:k})q(x_{0:k-1} | y_{0:k}) &  \\
& = & q(x_k | x_{0:k-1}, y_{0:k})q(x_{0:k-1} | y_{0:k-1})  & \text{[\autoref{eq:QAssump}]} \\
& = & q(x_k | x_{k-1}, y_{0:k})q(x_{0:k-1} | y_{0:k-1})  & \text{[Markov Property]}\\
& = & P(x_k | x_{k-1})q(x_{0:k-1} | y_{0:k-1})  & \text{[\autoref{eq:ImportanceDensity}]}
\end{array}
\end{equation}

Calculating $P(x_{0:k} | y_{0:k})$ is a bit more involved. 
First, using the assumption that the distribution of $y_k$ is 
fully constrained by $x_k$, and that $x_k$ is similarly fully 
constrained by $x_{k-1}$, I make the good assumptions that:
\begin{eqnarray}
P(y_k | x_{0:k}, y_{0:k-1}) &=& P(y_k | x_k) \nonumber \\
P(x_k | x_{0:k}, y_{0:k-1}) &=& P(x_k | x_{k-1})
\label{eq:MarkovProperty}
\end{eqnarray}
These are of course just re-statements of the state equations assumed by \autoref{eq:stateass}
and \autoref{eq:measass}.

Additionally, for the particle filter $y_k$ and $y_{0:k-1}$ are 
constant across all particles, thus $P(y_k| y_{0:k-1})$ can
be dropped when the equality is changed to a proportion. 
Using these properties, $P(x^i_{0:k} | y_{0:k})$ may be broken up as follows 
(primarily using Bayes' Theorem):
\begin{equation}
\begin{array}{lclr}
P(x_{0:k} | y_{0:k}) & = & \frac{P(y_{0:k}, x_{0:k})}{P(y_{0:k})} & \\
 & = & \frac{P(y_k, x_{0:k} | y_{0:k-1}) \cancel{P(y_{0:k-1})}}{P(y_k | y_{0:k-1}) \cancel{P(y_{0:k-1})}} & \\
 & = & \frac{P(y_k| x_{0:k}, y_{0:k-1}) P(x_{0:k} | y_{0:k-1})}{P(y_k | y_{0:k-1}) } & \\
 & = & \frac{P(y_k| x_{0:k}, y_{0:k-1}) P(x_k | x_{0:k-1}, y_{0:k-1}) P(x_{0:k-1} | y_{0:k-1})}{P(y_k | y_{0:k-1})} &  \\
& = & \frac{P(y_k| x_k) P(x_k | x_{k-1}) P(x_{0:k-1} | y_{0:k-1})}{P(y_k | y_{0:k-1})}  & [\text{\autoref{eq:MarkovProperty}}]\\
& \propto & P(y_k| x_k) P(x_k | x_{k-1}) P(x_{0:k-1} | y_{0:k-1}) & [P(y_k|y_{0:k-1}) \text{ is constant}]
 \end{array}
 \label{eq:UpdateBayes}
\end{equation}
Plugging \autoref{eq:ImportanceDensity} and the result of \autoref{eq:UpdateBayes}
into \autoref{eq:weightfunc} leads to:
\begin{eqnarray}
w^i_k & \propto & \frac{P(y_k| x^i_k) \cancel{P(x^i_k | x^i_{k-1})} P(x^i_{0:k-1} | y_{0:k-1})}
                         {\cancel{P(x^i_k | x^i_{k-1})}q(x^i_{0:k-1} | y_{0:k-1})} \nonumber \\
& \propto & w^i_{k-1}P(y_k| x_k) 
\label{eq:weightevolve}
\end{eqnarray}

Thus, by making the following simple assumptions, evolving a posterior
density  requires no knowledge of noise distribution.
\begin{enumerate}
\item $f(t, x(t), u(t)) = f(x(t), u(t))$ and $g(t, x(t), u(t)) = g(x(t))$ 
\item The prior distribution PDF, $q(x_i(0))$, covers $P(x_i(0))$
\item Markov Property: $P(x_k | x_{0:k-1}) = Pr(x_k | x_{k-1})$
\item $q(x_{0:k-1} | y_{0:k}) = q(x_{0:k-1} | y_{0:k-1})$
\end{enumerate}

\subsection{Basic Algorithm}
From the definition of $w_i$, the algorithm to calculate
an approximation of $P(X(t_k) | Y_{0:k})$ or $P(X(t_k + \delta t) | Y_{0:k})$
is simple. This basic form of the particle filter is given in 
algorithm \autoref{alg:BasicParticleFilter}.

\begin{algorithm}
\caption{Sequential Importance Sampling}
\begin{algorithmic}
\STATE Initialize Particles:
\FOR{$i$ : each of $N_p$ particles }
    \STATE $x^i_0  \sim \alpha(X)$
    \STATE $w^i_0 = \frac{1}{N_p}$
\ENDFOR
\FOR{$k$ : each measurement}
    \FOR{$i$ : each particle }
        \STATE $x^i_k = x^i_{k-1} + \int_{t-1}^t f(x(\tau), u(\tau)) d\tau $
        \STATE $w^i_k = w^i_{k-1}P(y_k | x_k)$
    \ENDFOR
\ENDFOR
\STATE $P(x(t_k+\Delta t)) \approx 
\sum_{i=0}^{N_p} w^i_k \delta\left(x - (x^i_k + \int_{t_k}^{t_k+\Delta t} f(x(\tau), u(\tau)) d\tau) \right)$
\end{algorithmic}
\label{alg:BasicParticleFilter}
\end{algorithm}

\section{Sequential Importance Resampling}
\label{sec:Particle Filter Resampling}
As a consequence 
of the wide prior distribution (required for a proper discretization of a continuous
distribution), in short order there will be a significant proportion of particles 
with insignificant weights. 
While this does help
describe the tails of the distribution, it means a great deal of computation will be wasted.
Instead, it would be preferable if most of the computation is spent on the most probable regions.
Ideally the computation time spent on tails would be proportional to the actual size of the
tails. In this case particle locations would match the true posterior and all weights would
be equal.  The case where a large number of the weights have become irrelevantly small
is called particle degeneracy. In  \cite{Liu1998b}
an ideal calculation of the effective number of particles is found based on the 
particles' true weight. However, given that only an approximation for the true weight 
exists, they also provide a simple heuristic calculation of $N_{eff}$.
\begin{equation}
N_{eff} \approx \frac{\sum_{i=0}^{N_p} w_i}{\sum_{i=0}^{N_p} w_i^2}
\label{eq:neff}
\end{equation}
Any quick run of a particle filter will reveal that unless the prior is particularly accurate,
$N_{eff}$ drops precipitously.  To alleviate this problem
a common technique known as resampling may be applied. The idea of resampling is to 
draw from the approximate posterior, thus generating a replica of the posterior with 
a better support. Therefore, a new set of particles may be drawn from the empirical
distribution as follows:
\begin{equation}
\hat{x}_j \sim \left(\sum_{i=0}^{N_p} w^i_k\delta(x - x^i_k)\right)
\end{equation}

\begin{algorithm}
\caption{Resampling Algorithm}
\begin{algorithmic}
\STATE Calculate total weight, $W = \sum_{i=0}^{N_p} w^i$
\FORALL{$0 < i < N_p$}
    \STATE Draw $V$ from uniform range $[0, W]$
    \STATE $C = W_t$
    \FORALL{$0 < j < N_p$ and $C < V$}
        \STATE $C = C - w^j$
    \ENDFOR
    \STATE Add $[x^j, \frac{1}{N_p}]$ to the new distribution
\ENDFOR
\STATE 
\end{algorithmic}
\label{alg:Resampling}
\end{algorithm}
For infinite particles this new distribution will match the old.
Unfortunately, this isn't the truth in practice: since the support is
still limited to the original particles, the number of \emph{unique} particles can only go down.
This effect, dubbed particle impoverishment can result in excessive quantization
errors in the final distribution. However, there is a solution. Instead of sampling from the
discrete distribution, a smoothing kernel is applied, and particles are drawn from
that distribution. Because it is continuous, particle impoverishment
cannot occur. The easiest way to sample from the continuous distribution is to break the 
re-sampling down into two steps. After calculating an estimate of the scale of the original
distribution, algorithm \autoref{alg:Resampling} is performed. Next, a distribution is generated
based on the variance of the original distributions.
Finally, for each particle in the discretely re-sampled distribution, a sample is drawn from 
the smoothing 
distribution and added to the particle.  The regularization process is defined as:

\begin{equation}
x_i = x_i + h\sigma \epsilon
\end{equation}

Where $h$ is an optional bandwidth, $\sigma$ is the standard deviation such that 
$\sigma \sigma^T = cov(x)$
and $\epsilon$ is drawn from the chosen kernel. \cite{Musso2001a} goes into
significant 
depth and proves the optimality of the Epanechnikov Kernel for reducing 
MSE between the original and resampled distributions. However, \cite{Musso2001a}
also espouses the usefulness of the Gaussian Kernel, due to the ease
drawing samples from it, which for this work was more important.

\begin{algorithm}
\caption{Regularized Resampling Algorithm}
\begin{algorithmic}
\STATE Calculate Covariance, $C$, of empirical distribution, $\hat{x}$
\STATE Find $D$ such that $DD^T = C$
\STATE Resample $\hat{x}$ using algorithm \autoref{alg:Resampling}
\FOR{$0 < i < N_p$}
    \STATE Draw $\epsilon$ from the standard normal, same dimensionality as $X$
    \STATE $x^i = x^i + hD\epsilon$
\ENDFOR
\end{algorithmic}
\label{alg:RegResampling}
\end{algorithm}

It has been proposed by \cite{Hurzeler1998} that if the underlying 
distribution is non-Gaussian, then using the original bandwidth will over-smooth. 
In reality, over smoothing
will only become an issue if re-sampling is performed very often. Thus
if resampling is performed at every step then this could certainly cause problems.
If the distribution is over-smoothed then the algorithm may not converge as rapidly;
however, because the bandwidth is still based on particle variance, which decays as 
particles are ruled out, it is still able to converge. In fact, over-smoothing is preferable
to under smoothing, since over-smoothing simply slows convergence while 
under-smoothing could leave gaps in the distribution.
Moreover, because of the high dimensionality of the BOLD model,
and limited measurements, it is helpful to have a broader bandwidth to explore the distribution. 

\begin{algorithm}
\caption{Regularized Particle Filter}
\begin{algorithmic}
\STATE Initialize Particles:
\FOR{$i$ : each of $N_p$ particles }
    \STATE $x^i_0  \sim \alpha(X)$
    \STATE $w^i_0 = \frac{1}{N_p}$
\ENDFOR
\FOR{$k$ : each measurement}
    \FOR{$i$ : each particle }
        \STATE $x^i_k = x^i_{k-1} + \int_{t-1}^t f(x(\tau), u(\tau)) d\tau $
        \STATE $w^i_k = w^i_{k-1}P(y_k | x_k)$
    \ENDFOR

    \STATE Calculate $N_{eff}$ with \autoref{eq:neff}
    \IF{$N_{eff} < N_R$ (recommend $N_R = min(50, .1N_p)$ )}
        \STATE Resample using algorithm \autoref{alg:RegResampling}
    \ENDIF
\ENDFOR

\STATE At $t + \Delta t$, $t \in T$, $P(x(t+\Delta t)) \approx 
\sum_{i=1}^{N_p} w_i(t)\delta\left(x - (x_i(t) + \int_t^{t+\Delta t} f(x(\tau), u(\tau)) d\tau) \right)$
 \end{algorithmic}
 \end{algorithm}

Nevertheless, because 
of the potentially wide smoothing factor applied by regularized resampling, performing this
step at every measurement would allow particles a great deal of mobility. This mobility could
hinder convergence, which is why regularized resampling should only be done when
$N_{eff}$ drops off (less than 50). Other than the periodic regularized
resampling, the regularized particle filter is identical to the basic sampling
importance sampling filter (SIS). 

With regularized resampling, it is possible to prevent both
particle degeneracy as well as particle impoverishment. An additional
bonus is that as the particle filter converges, the density of particles
in the area of the solution goes up. This has a similar effect to 
simulated annealing where, as the algorithm approaches the end, the
random steps get smaller and smaller. However, there is risk in 
resampling. If for some reason the solution is not covered by the 
new support the algorithm may not be able to reach the true value. 
The ultimate effect of this regularized resampling is a convergence similar to simulated annealing
or a genetic algorithm. Versions of $x$ that are fit (give good measurements) spawn more children 
nearby which allow for more accurate estimation near points of high likelihood. 
As the variance of the estimated
$x$'s decrease, the radius in which children are spawned also decreases. Eventually the radius
will approach the width of the underlying uncertainty.

\section{Weighting Function}
Because the distribution of $\nu_y$ in \autoref{eq:measass} is unknown,
it is necessary to choose a distribution for this. This distribution
is important because $\nu_y \sim P(y_k | x(T))$, which is used
for updating weights. Ideally this weighting function would exactly 
match the measurement error in the output. 
Typically it is assumed that this error is additive, centered at zero and 
has a scale comparable to the signal levels.
While a Gaussian function is the traditional choice, there are other reasonable
distributions, given the unpredictable nature of the noise present in FMRI.
The choice of this function will be discussed further in \autoref{sec:Methods Weighting Function}.

\begin{figure}
\includegraphics[width=16cm]{images/particle_filter}
\caption{Particle Filter progression, note that the initial support is flat; the particles
are equally spaced between -10 and 10}
\end{figure}

\section{Simple, Nonlinear Example}
A typical half wave rectifier takes a AC voltage circuit and removes
one half (say the negative half) of the signal. The resulting waveform
is still not DC, however it is then possible to use a capacitor to 
smooth the signal into something similar to DC, as shown in \autoref{fig:HalfWaveIO}.
There are other, more
complex circuits that convert the negative portion into positive and
waste less energy but here I will keep the system simple.
Thus, let us consider a simple half wave rectifier circuit, shown in 
\autoref{fig:HalfWaveRectifier}.

The half wave rectifier circuit smoothes the gaps between high voltage
with a capacitor. Thus, when $u(t)G$ is less than $v_t$, the circuit will 
discharge the capacitor and maintain a non-zero voltage,
but when $u(t)G$ is greater than $v_t$, the output voltage will be set
by $u(t)G$ and the capacitor will charge up. I will assume a simple
model for all the components, ignoring the complex nonlinear behavior
that can occur in a diode. 

\begin{figure}
\centering
\begin{circuitikz}[scale=2, american]
\draw
 (0,0)  node[transformer core] (T) {}
 (T.A1) -- (-1,0)
 (T.A2) -- (-1,-1.05)  to[V, v=$u(t)$] (-1, 0)
 (T.B1) -- (.5, 0) to[D, l=$v_t$] (1.5,0) to[C=$C$] (1.5, -1.05)
 (1.5, 0) -- (2.5, 0) to[R=$Rm$, v=$v_y$] (2.5, -1.05) -- (T.B2) 
 (T.base) node {G}
 (T.B1) to[open, *-*, v=$V_1$] (T.B2); 
\end{circuitikz}
\caption{An Example Half Wave Rectifier Circuit, where $G$ is the transformer
gain, $v_t$ is the activation voltage of the diode, $u(t)$ is the input at time $t$, 
$C$ is the capacitance, $R$ is the load resistance and $v_y$ is the output voltage}
\label{fig:HalfWaveRectifier}
\end{figure}

\begin{figure}
\centering
\caption{Example Input/Output of the Half Wave Rectifier}
\label{fig:HalfWaveIO}
\end{figure}

For this example I will assume that the transformer simply scales
the input by a constant factor, $G$. As discussed in the \autoref{sec:Particle Filter Model}
any variable with uncertainty must be part of the state variable. Therefore
the state variable is defined as: $X(t) = \{G, v_t, C, R_m, v_y\}$. 
The state equations would then be 

\begin{equation}
v_y(t)  = f(v_y(t-1, u(t)) =  \begin{cases} 
        u(t)G & \text{ if }  u(t)G-v_y \ge v_t\\
        v_y(t-1)\left(1 - \frac{\delta t}{R_mC}\right) & \text{ if }  u(t)G-v_y < v_t
    \end{cases} 
\end{equation}

To run the particle filter is easy, since there exists 
a recursive definition of the dynamic state variable, $v_y$. To start
with, an initial distribution must be assumed and while at first a 
Gaussian seems like a good idea, all the static state variables are strictly
positive and thus not well suited to the Gaussian. Thus, instead the prior 
will start with a Gamma distribution. The gamma is defined as follows:

\begin{equation}
X \sim Gamma(k, \theta) \rightarrow f(x) = x^{k-1}\frac{e^{-x/\theta}}{\theta^k\Gamma(k)}
\end{equation}

where $\Gamma$ is the gamma function.
The margin for error is decided by the weighting function, which
will be defined as $W(V_y, v_{yi})$, where $V_y$ is the actual measurement, $v_y$ is the 
estimate based on all the particles, and 
$v_{yi}$ is the estimate by a particular ($i^{th}$) particle. The choice of this function is difficult,
and although the Gaussian is typically used, in practice I found the exponential helpful
in preventing particle deprivation. The algorithm is then,

\begin{algorithmic}
\STATE Initialize $N_p$ Particles:
\FOR{$i$ in $N_p$}
    \STATE $G \sim Gamma(\frac{\mu^2_G}{\sigma^2_G}, \frac{\sigma^2_G}{\mu_G})$
    \STATE $v_t \sim Gamma(\frac{\mu^2_{v_t}}{\sigma^2_{v_t}}, \frac{\sigma^2_{v_t}}{\mu_{v_t}})$
    \STATE $C \sim Gamma(\frac{\mu^2_C}{\sigma^2_C}, \frac{\sigma^2_C}{\mu_C})$
    \STATE $R_m \sim Gamma(\frac{\mu^2_R}{\sigma^2_R}, \frac{\sigma^2_R}{\mu_R})$
    \STATE $v_y = 0$, (Assume the system has been off for a long time)
    \STATE let $X_i(0) = \{G, v_t, C, R_m, v_y\}$
    \STATE let $w_i(0) = 1$ or to make a flat prior, $w_i(0) = \frac{1}{Pr(X_i(0))}$ 
\ENDFOR
\STATE Run the Filter:
\FOR{$t$ in Set of Measurement Times}
    \FOR{$i$ in $N_p$}
        \STATE $v_{yi}(t) = f(v_{yi}(t-1), u(t))$
        \STATE (All other members of $X_i(t)$ remain the same)
        \STATE $w_i(t) = w_i(t-1)W(V_y(t), v_y(t))$
    \ENDFOR
\ENDFOR
\end{algorithmic}

Initially the particles will have the same output, $0$, however, as $u(t)$
changes, the response of each particle to that input will result in different
outputs. Particles that have a $v_{yi}$ near $V_y$ will be weighted higher,
and others farther away will be weighted lower. As the particle filter
runs, weights will compound converging to a distribution that asymptotically
approaches the true joint distribution of the $X(t)$.  As I
mentioned in \autoref{sec:Particle Filter Resampling}, particles with 
weights approaching zero do not significantly
contribute to the empirical distribution, so re-sampling will be necessary.


\chapter{Methods}
\label{sec:Methods}
Although the particle filter  is a standard Regularized
Particle filter, as described in \cite{Arulampalam2002a}, optimizing the 
particle filter for use with FMRI data is non-trivial. 


\section{Model}
As originally written in \autoref{sec:BOLD Physiology} the state variables
for the BOLD model are as follows:
\begin{eqnarray}
\dot{s} &=& \epsilon u(t) - \frac{s}{\tau_s} - \frac{f - 1}{\tau_f} \\
\dot{f} &=& s\\
\dot{v} &=& \frac{1}{\tau_0}(f - v^\alpha)\\
\dot{q} &=& \frac{1}{\tau_0}(\frac{f(1-(1-E_0)^f)}{E_0} - \frac{q}{v^{1-1/\alpha}})
\end{eqnarray}
The original assumption regarding particle filter models (\autoref{sec:Particle Filter Model})
included noise in the update of $x$, however that is not included here.
The reason for the difference is that cloud of particles is, to some extent,
able to account for that noise. It is common, however, to model that noise
in particle filters by adding a random value to each updated state variable. 
Because the purpose of this particle filter is to learn the underlying distribution
of the static parameters, rather than precisely model the time course of the 
in the dynamic parameters ($\{s,f,v,q\}$) this noise is left out. It also helps
that detrending is applied before the particle filter and that the
BOLD model is dissipative. When no stimuli are applied, all the particles 
decay to ($\{0,1,1,1\}$). Typical particle filters 
also use this state noise as an exploratory measure; however this method is
less necessary when good priors are available.

For all the analyses  in this work, $1400$ integration points
per second were used.  Typically a step size of $0.001$ was sufficient,
however, from time to time $0.001$ can still be too high for the
BOLD model.

\section{Preprocessing}
\label{sec:Methods Preprocessing}
The normal pipeline for analyzing
FMRI involves a several preprocessing steps. The first and most important
task is motion correction. To do this, a single volume in time is chosen, and
volumes at every other time are registered to this one volume. This corrects
for motion by the patient as well as small changes in the magnetic
fields that cause the image to shift. 
In conventional statistical parametric mapping, a Gaussian smoothing
filter is applied across the image as discussed in \autoref{sec:RFT}.
After this, detrending is performed which is discussed in \autoref{sec:Detrend}.
Recall that FMRI signal levels are unit-less and though detrending is not
always necessary, the data must always be converted 
into \% difference from baseline. 
The generally accepted method is to use a high pass filter, although the
cutoff frequency is application dependent and often applied haphazardly.
Before going into the detrending used in this work, it is necessary to 
discuss the type of noise present in FMRI.

\subsection{BOLD Noise}
\label{sec:Introduction Noise}
As demonstrated in \autoref{sec:BOLD Physiology} the BOLD response has been
extensively studied and despite minor discrepancies, the cause of the BOLD 
signal is well known. However, as FMRI detects an  
aggregate signal over the space of cubic centimeters, there are
plenty of noise sources . Though local neurons act
together (i.e. around the same time), the density of neurons, the
density of capillaries, and slight differences in activation across 
a particular voxel can all lead to signal attenuation and noise. 

A particularly difficult form of noise present in FMRI is a low frequency
drift, often characterized as a Wiener process (\cite{Riera2004}). 
Though not present in all regions, as many as ten to fifteen percent
of voxels can be affected (\cite{Tanabe2002}), thus it is prevalent enough to cause significant
inference problems \cite{Smith2007}. It is still not
clear what exactly causes this noise, although one possibility is 
the temperature difference in scanner magnetic coils\cite{Smith2007}. 
It is clear that this drift signal is not solely
due to a physiological effects, given its presence in cadavers and phantoms 
\cite{Smith1999}. Interestingly, it is usually spatially correlated, and
more prevalent at interfaces between regions. Though one potential source
could be slight movement, co-registration is standard, making this unlikely. 
Regardless, the problem mandates the use of a high pass filter \cite{Smith2007}.

In order to characterize the noise, I analyzed resting state data.
During resting state, the patient is shown no images, and he is asked
to avoid movement and complex thought.  Overall though there should be 
very little activation, and thus the signal consists entirely of noise. 
Therefore resting state data is perfect for analyzing noise. 
The locations were chosen from points all around the brain, 
all in grey matter voxels. These time
series were chosen because they were representative of different types
of noise found in the resting state data.

The resting state was gathered in the exact same way as the data in 
\autoref{sec:ExperimentConfig}, except without the stimuli.

\begin{figure}
\centering
\subfigure[]{\label{fig:QQDC:A}\includegraphics[trim=6cm 1cm 6cm 1cm,width=13cm]{images/noise2_0009_29_49_9}}
\subfigure[]{\label{fig:QQDC:B}\includegraphics[trim=6cm 1cm 6cm 1cm,width=13cm]{images/noise2_0009_34_43_24}}
\subfigure[]{\label{fig:QQDC:C}\includegraphics[trim=6cm 1cm 6cm 1cm,width=13cm]{images/noise2_0009_22_38_23}}
\subfigure[]{\label{fig:QQDC:D}\includegraphics[trim=6cm 1cm 6cm 1cm,width=13cm]{images/noise2_0009_37_29_24}}

%\subfigure{\includegraphics[trim=6cm 1cm 0 0cm,width=17cm]{images/noise_0009_19-24-10.pdf}}
%\subfigure{\includegraphics[trim=6cm 1cm 0 0cm,width=17cm]{images/noise_0009_20-45-18.pdf}}
%\subfigure{\includegraphics[trim=6cm 1cm 0 0cm,width=17cm]{images/noise_0009_23-47-18.pdf}}
%\subfigure{\includegraphics[trim=6cm 1cm 0 0cm,width=17cm]{images/noise_0009_35-49-9.pdf}}

\caption{Q-Q Plots of normalized resting state data}
\label{fig:QQDC}
\end{figure}

\begin{figure}
\centering
\subfigure[]{\label{fig:QQDDelta:A}\includegraphics[trim=6cm 1cm 6cm 1cm,width=13cm]{images/noise2_0009d_29_49_9}}
\subfigure[]{\label{fig:QQDDelta:B}\includegraphics[trim=6cm 1cm 6cm 1cm,width=13cm]{images/noise2_0009d_34_43_24}}
\subfigure[]{\label{fig:QQDDelta:C}\includegraphics[trim=6cm 1cm 6cm 1cm,width=13cm]{images/noise2_0009d_22_38_23}}
\subfigure[]{\label{fig:QQDDelta:D}\includegraphics[trim=6cm 1cm 6cm 1cm,width=13cm]{images/noise2_0009d_37_29_24}}
\caption{Q-Q Plots of resting state data, using the BOLD signal changes}
\label{fig:QQDelta}
\end{figure}


\begin{figure}
\centering
\subfigure[]{\label{fig:QQs:A}\includegraphics[trim=6cm 1cm 6cm 1cm,width=13cm]{images/noise2_0009s_29_49_9}}
\subfigure[]{\label{fig:QQs:B}\includegraphics[trim=6cm 1cm 6cm 1cm,width=13cm]{images/noise2_0009s_34_43_24}}
\subfigure[]{\label{fig:QQs:C}\includegraphics[trim=6cm 1cm 6cm 1cm,width=13cm]{images/noise2_0009s_22_38_23}}
\subfigure[]{\label{fig:QQs:D}\includegraphics[trim=6cm 1cm 6cm 1cm,width=13cm]{images/noise2_0009s_37_29_24}}
\caption{Q-Q Plots of resting state data, after the de-trending}
\label{fig:QQSpline}
\end{figure}

Because most methods (including the one used in this paper)
assume the noise realizations are independent of each other, the auto-
correlation is of particular interest (which is a necessary but not
sufficient condition for independence). Gaussianity is also a common
assumption made in studies of FMRI data, though that assumption is not
needed in this work. Regardless, comparing the distribution to a Gaussian
is informative, so Q-Q plots are used to compare example data with the
Normal distribution. Additionally, in FMRI data the noise is often considered 
to be Wiener \cite{Riera2003}. Recall that a Wiener random process is
characterized by steps that are Gaussian and independent. The simulations discussed in 
\autoref{sec:Single Voxel Simulation} make use of this, 
by adding a Wiener random process to the overall signal. To determine
whether the noise is in fact Wiener, the distribution of 
the steps were plotted against a Gaussian. 

Finally, removal of the drift is often performed with a high pass filter,
so analyzed the distribution after subtracting of a spline, (see \autoref{sec:Methods Preprocessing}).

\autoref{fig:QQDC} shows the 
results with a regression line fit to the points.
Recall that in a Quartile-Quartile (Q-Q) plot, if the points plotted on the 
x-axis and the points
on the y-axis come from the same type of distribution, then all the points will
be collinear. Differences in the variance will cause the line to have a slope
other than 1, while differences in the expected value will cause the fitted line
to be shifted. In these Q-Q plots, the points are being compared to the standard
Gaussian distribution. Note that in \autoref{fig:QQDC} the points have all been 
normalized (changed to percent difference).

Note that \autoref{fig:QQDC:A} and \autoref{fig:QQDC:B}
are well described by a Gaussian process with a small autocorrelation, 
\autoref{fig:QQDC:C} and \autoref{fig:QQDC:D} are not. In particular the tails of \autoref{fig:QQDC:C}
do not seem to fit the Gaussian well. Also note the significant autocorrelation in
\autoref{fig:QQDC:C} and \autoref{fig:QQDC:D}. As expected, the noise is not strictly
Gaussian white noise.  On the other hand, the steps do conform rather
closely to the normal distribution.
As expected, most of the autocorrelation disappears for the step data. Given
that the steps seem to fit the Normal distribution, the low autocorrelation
indicates that the steps could be Independently Distributed. 
Therefore, the noise does seem to come close to a Wiener process. 

De-trending the time-series by subtracting a spline fit to the distribution
removed much of the autocorrelation present in \autoref{fig:QQDC:C} and \autoref{fig:QQDC:D},
though not perfectly. Though the distributions still do not exactly fit
the Normal, \autoref{fig:QQs:D} is much improved compared to \autoref{fig:QQDC:D}.
In all, the de-trending is effectively removing Wiener noise. 

\subsection{Detrending}
\label{sec:Detrend}
The non-stationary
aspect of a Weiner process, presumably the result of integrating some
$\nu_x$ is difficult to compensate for, and so many methods
have been developed to compensate for it. \cite{Tanabe2002} and \cite{Smith1999} have
demonstrated that this component is prevalent, and may in fact be an inherent  characteristic
of FMRI. It has been reported that in some studies as many as half the voxels 
benefited from detrending (\cite{Smith2007}). In a head to head comparison, 
\cite{Tanabe2002}, showed that in most cases subtracting off
a spline worked the best. The benefit of the spline versus wavelets, high pass 
filtering or other DC removal techniques is that the frequency response is not set.
Rather, the spline is adaptive to the input. Unfortunately no method will 
perfectly remove noise, and no method will leave the signal untouched.

The method I used to calculate the spline was picking one knot for every 20
measurements in an image. Thus a 10 minute session at a repetition time of 
2.1 seconds would have 19 knots. The knot first and last knots were each 
given half the number of samples as the rest of the knots; which were all 
located at the center of their sample group. The median of each sample group
was then taken and used as the magnitude for the group. Taking the median 
versus the mean seemed to work better, given the presence of outliers. 
There is potential to optimize the spline further using a canonical 
HRF to find resting points; however, for this to work the experiment would have
to be designed with this in mind. 

Problematically, after removing the DC component of the signal,
by definition the signal will have a median near zero. 
Unfortunately this is not the natural state of the BOLD signal. More specifically,
when the signal is inactive, the BOLD response should be at 0\% change from
the base level; activation may then increase, or for short periods decrease from this base.
Because most of the BOLD signal is above baseline, after removing the spline
the BOLD resting state will be below 0\%.  This reduces the ability of an algorithm to learn.
One method of accounting for this is to simply add a DC gain model parameter.
Like all the other model parameters, with enough measurements, a viable
parameter would fall. Yet adding another dimension increases the
complexity of the model, when the parameter is relatively easy to estimate
by visual inspection.  In this work a simpler approach was used. To determine
the DC gain I used a robust estimator of scale. The Median Absolute Deviation (MAD)
proved to be accurate in determining how much to shift the signal up
by. I tested both methods during the course of analysis, and found that the increase 
in model complexity far outweighed the slight increase in flexibility. Other
methods may work better, however the MAD worked well, 
as \autoref{fig:PreprocessedLowNoise} and \autoref{fig:PreprocessedHighNoise} show. 

\begin{equation}
y_{\text{gain}, 0:K} = 1.4826\text{median}_{i=0:K}(y_i - \text{median}(y_{0:K}))
\label{eq:mad}
\end{equation}

A serious concern when adding and subtracting arbitrary values to 
real data is whether this will create false positives. This is a legitimate
concern; however, a boosted response does not effect how well the BOLD model 
predicts the actual measurements. 

%\subsection{Linearizing Noise}
%\label{sec:Methods Delta Based Inference}
%The alternative to these sorts of low frequency manipulation is to
%go around the problem in another way. Here, I propose a 
%different method of dealing with the drift. 
%Instead of comparing the direct output of the particle filter with the direct
%measurement, the algorithm would compare the change in signal over a single TR,
%with the result of integrating the model for the same period. 
%In most signal processing cases this would be foolish, but that is because the 
%general assumption is that all noise is high frequency. Considering 
%the fact that every BOLD analysis pipeline uses a high pass filter,
%whereas low poss temporal filter are rarely applied, it makes sense
%that a derivative type method could work. The benefit of particle filters
%is that they are a robust method of inference, and I would assert 
%that the particle filter ought to be given as \emph{raw} data as possible. 
%While taking direct measurements
%without de-trending would give awful results, using the difference removes the 
%DC component and turns what is usually assumed to be a Weiner process into 
%a simple Gaussian random variable. 
%
%\begin{equation}
%\Delta y = y(t) - y(t-1) = g(x(t)) - g(x(t-1)) + \nu_y(t) - \nu_y(t-1) + \nu_d(t) - \nu(t-1)
%\label{eq:measass_delta}
%\end{equation}
%
%Even if $\nu_d$ is some other additive process, the difference will still be closer
%to I.I.D. than a Wiener process, as the autocorrelation of the $\delta y$ shows
%in \autoref{fig:QQDelta} in \autoref{sec:Introduction Noise}. 
% All the assumptions made originally
%for the particle filter still hold, and all of the parameters may be distinguished based on
%the step sizes, thus it is not unreasonable to consider matching the string of step sizes
%rather than string of direct readings. 
%
%\begin{figure}
%\label{fig:FrequencyGraphs}
%\caption{frequency response graphs, highlighting noise frequency range and signal frequency range}
%\end{figure}

\section{Particle Filter Settings}
There are quite a few options when using a particle filter; those
options will be discussed in this section.

\subsection{Prior Distribution}
\label{sec:PriorDist}
For the BOLD model described in \autoref{sec:BOLD Physiology}, several
different studies have endeavored to calculate parameters. The results
of these studies may be found in \autoref{tab:Params}, and the methods 
used for each may be found in \autoref{sec:Prior Works}. Unfortunately,
\cite{Friston2000} only studied regions deemed active by the General 
Linear Model; and most other research (including \cite{Friston2001}) used these results as 
the source for their priors. 
The one exception is \cite{Johnston2008}, which came to a extremely different
distributions. For a particle filter, the choice of a prior is
the single most important design choice. A very wide prior will require
more particles to be sufficiently dense, a very thin (low variance) prior may miss
the true parameters. Consequently, for this work it was natural
to use priors that will give results consistent with previous works, 
\cite{Friston2000}. This constrains the usefulness of the model to
areas that fall within the prior distribution, yet will allow results
to be comparable to other works. There is a significant need for better
estimates of the physiological parameters; and, while physical experiments
may not be possible, it would not be unreasonable to do a study with
exhaustive simulated annealing or hill climbing tests for multiple
regions and multiple patients.

There is an interesting anomaly with the priors found in virtually all
the works that characterized the parameters, except \cite{Johnston2008}.
The BOLD signal is universally recognized to be around $2-3\%$, maybe
reaching $5\%$ in extreme activation. Yet using the mean priors
from \cite{Friston2000}, the signal response for a $0.1$ second
impulse only reaches half a percent, as \autoref{fig:MeanResponseF}
shows.

\begin{figure}
\centering
\includegraphics[trim=6cm 3cm 6cm 3cm,width=16cm]{images/mean_response}
\caption{Response to $0.1s$ impulses with the mean parameters from \cite{Friston2000}}
\label{fig:MeanResponseF}
\end{figure}

While this could be the result of a stimulus
being too short to lead to strong activation, a similar stimulus
scheme in real data showed a much larger response than 
half a percent as well. In fact, after applying de-trending,
converting the image to percent-difference, and removing 
outliers ($ BOLD > 10\% \text{ or } BOLD < -10\%$) the total variance
across all \emph{active} voxels was still around .02, indicating
that in active voxels a signal peaking below .005 seems unlikely. 
Of course, if more restriction were placed on the outliers, its possible
this standard deviations could be brought down. 
The parameter estimates by \cite{Johnston2008} are even more 
confusing, with peaks of well below $.1\%$ (\autoref{fig:MeanResponseJ}).

\begin{figure}
\centering
\includegraphics[trim=6cm 3cm 6cm 3cm,width=16cm]{images/mean_response_johnston}
\caption{Response to $0.1s$ impulses with the mean parameters from \cite{Johnston2008}}
\label{fig:MeanResponseJ}
\end{figure}

Its likely that these differences are due to some difference in preprocessing,
although in \cite{Deneux2006} the signals were found to be peaking around
$1\%$, unlike \cite{Friston2000} which shows signals peaking at up to
$3\%$ or $4\%$. In my own tests, it seemed necessary for $\epsilon$ to
reach well over $1.5$ and $V_0$ to reach more than $.4$ to reach these
peaks; of course other methods may be equally able. 
Therefore, to account for these discrepancies, somewhat broader
distributions are used than the numbers used in \cite{Friston2000}
(which are widely used, \cite{Hu2009}). The 
priors used in the particle filter may be found in \autoref{tab:Prior}.

\begin{table}[t]
\centering
\begin{tabular}{|c || c | c | c |}
\hline 
Parameter & Distribution & $\mu$ & $\sigma$ \\
\hline
$\tau_0$ & Gamma & .98 & .25 \\
$\alpha$ & Gamma & .33 & .045\\
$E_0$    & Gamma & .34 & .03  \\
$V_0$    & Gamma & .04 & .03 \\
$\tau_s$ & Gamma & 1.54  & .25\\
$\tau_f$ & Gamma & 2.46  & .25\\
$\epsilon$ & Gamma & .7  & .6 \\
\hline
\end{tabular}
\caption{Prior distributions used in the particle filter.}
\label{tab:Prior} 
\end{table}

Note that although the mean remains the same for all the 
parameters other than $\epsilon$, the standard deviation is set
much higher to account for the disagreement between studies
(\autoref{tab:Params}). 
Because all the parameters are taken to be strictly positive, and the
standard deviations are approaching the mean, I used a gamma distribution.
This prevents the Gaussian from placing parameters in the nonsensical 
territory of negative activation, or negative time constants.

Another aspect of the prior is using enough particles to get a 
sufficiently dense approximation of the prior. For 7 dimensions,
getting a dense prior is difficult. Insufficiently
dense particles will result in inconsistent results. Of course the
processing time will scale up directly with the number of particles.
A dense initial estimate is important so that some particles land
near the solution; but as the variance decreases the number of 
particles needed decreases as well. Thus, as a heuristic, initially
the number of particles was set to 16,000, but after resampling,
the number of particles was dropped to 1,000. Typically during the 
first few measurements the variance dropped precipitously because
most particles were far from a solution.  The particles that are left are in a
much more compact location, allowing them to be estimated with 
significantly fewer particles. These numbers aren't set in stone,
and depending on the complexity of the system or desired accuracy
they could be changed; however, they seem to be the minimum that
will give consistent results.

\subsection{Resampling}
\label{sec:Resampling}
The algorithm for resampling is described in \autoref{sec:Particle Filter Resampling}.
When regularizing, the Gaussian kernel is convenient,
because it is simple to sample from and long tailed.
As discussed in \autoref{sec:Particle Filter Resampling},
as long as resampling is kept as a last resort, some over-smoothing
doesn't impair convergence. Therefore, for this work I chose a Gaussian kernel of
bandwidth equal to the original distribution's covariance. Obviously this will
apply a rather large amount of smoothing to the distribution; however, on average
resampling is only applied every 20 to 30 measurements, and because randomization
is being applied to model updates this gives the filter some mobility. 

Re-sampling is a not strictly necessary, but increases the effectiveness
of the particle filter by adjusting the support to emphasize areas
of higher probability. Re-sampling is slow because it requires re-drawing
all the particles. It also closes off avenues of investigation, and is
designed to over-smooth to prevent overly thinning the support. For all these
reasons, resampling was only performed when the $N_{eff}$ dropped below
50 (for 1000 particles).  As a measure against sharp drops in the $N_{eff}$ 
caused by a large spike in error, resampling was only performed when 
two consecutive low ($<50$) $N_{eff}$'s were found. 

\subsection{Choosing $P(y_k | x_k)$}
\label{sec:Methods Weighting Function}
The choice of $P(y_k | x_k)$ is the second most important design decision, behind 
the prior. While the conventional choice for an unknown distribution is the 
Gaussian, there are reasons why it may not be the best in this case.  
As noted in \autoref{sec:Introduction Noise}, the noise is not strictly Gaussian,
nor is it strictly Wiener. As with any unknown noise however, it is necessary 
to make some assumption. If the weighting function ($P(y_k | x_k)$) exactly
matches the measurement error, then the ideal particle filter will result.
Particles with $x_k$'s that repeatedly estimate $y_k$ with large residual 
will quickly have weights near 0. Thus, a weighting function that
exactly matches $P(Y(t) | X(t))$ will easily, and correctly throw out incorrect
particles.  The cost of choosing an overly broad distribution for this
function is slow convergence.  On the other hand, an overly thin 
distribution will lead to particle deprivation (all particles
being zero-weighted).  I tested several weighting
functions: in addition to the Gaussian I also tested the Laplace and Cauchy
distributions, both of which have much wider tales than the Gaussian. 
Wider tailed distributions don't down-weight
particles as fast; and converge more slowly (and perhaps more accurately). 
The Laplace distribution also has the
benefit of a non-zero slope at the origin; which means that even
it will distinguish between particles even near the origin.

After trial and error, for this work I chose a zero-mean Gaussian with standard deviation 
of $.005$. While I made some attempts to automatically set the standard 
deviation, results were often unpredictable. If the weight function and scale aren't
fixed across voxels, very noisy time series with no actual signal 
converged to nonsensical results. 
In the future, it may be possible to set the standard deviation by
taking a small sample from resting data and using the sample standard deviation.
Since this is the first attempt at using particle filters for modeling the 
BOLD model, in this work I set the standard deviation manually at $.005$,
because it gave the best consistency. 

\subsection{Runtime}
The run-time for a single voxel depends on the several factors. First, the
overall length of the signal being analyzed. For 1000 measurements it takes
about 6 minutes. On the other hand, in real circumstances the
length is only around 150 measurements and takes around 40 seconds (for 1000 
particles, 1500 integration points and a Quad Core CPU). The size of 
local linearization steps are also crucial although going above $0.001$ seconds
per step is not recommended. In most cases millisecond resolution
is fine; however, when generating simulated data I found that at times it was
still not enough every once. This is problematic in the actual particle
filter since, given the large number of simultaneous integrations taking 
place, its probable that a few particles will fail and be unfairly thrown away.
To prevent such events, 1500 integration points per second were used throughout
the tests. 

Another crucial factor for run time is how long before the first re-sampling 
occurs. Because the prior is represented initially with significantly more
particles, if for some reason the effective
number of particles stays high, resampling could take a long time to occur.
For this reason, rather than allowing the particle filter to continue on 
with this large number of particles, after 20 seconds have passed the
algorithm forces resampling. The choice of 20 seconds is 
arbitrary, but at the very least it gives a more optimized version of the
prior. 


\chapter{Results}
\section{Single-Voxel Simulation}
The results 

\section{Single-Voxel Analysis}
This section discusses the results when the particle filter was
applied on a single voxel. The parameters are the same as
those used later for entire image analysis; however, the results
are more in-depth. 

The run-time for a single voxel depends on the several factors. First, the
overall length of the signal being analyzed. For 1000 measurements it takes
about 10 to 15 minutes. On the other hand, in real circumstances the
length is only around 150 measurements. The number of  integration
can certainly make a large difference, however dropping below 1000 (.001 seconds)
is definitely not recommended. 

For a period I considered 1000 to be a
fine number; however when generating simulated data I found that every once
in a while 1000 was not enough. This is problematic in the actual particle
filter since, given the large number of simultaneous integrations taking 
place, its likely that a few particles will fail and be weighted at 0 because
of this problem. Additionally, because the typical case where a failure would
occur is at fast moving times/parameters the particles all tend to fail together.
The result is particle deprivation - no particles with non-zero weights remain.
The other possible outcome is that low time constant particles get pruned resulting
in excessively smoothed estimates for the time series'. Its possible that a
kind of stop-gap measure could be put into place; wherein particles that are
about to be set to NaN are integrated again with finer grained steps. However
many times the non-real results don't occur until several time steps after the 
numbers get strange. So for instance, the timestep was too long, allowing 
$f$ to go negative, resulting in extremely large values of $q$. There are many
different ways where this sort of event can occur, and unfortunately sometimes
there is no way to get back to before the state starting going out of control.

Another crucial factor to run time is how long before the first re-sampling 
occurs. Because the prior is represented initially with significantly more
particles, if the model fits very well, or for some reason the effective
number of particles stays high, resampling could take a long time to occur.
When this happens the particles filter can take a factor of 10 longer to run.
However, if the particle count isn't initially set high, there is a much larger
chance of particle deprivation occurring. Since there is no real way to know
how long it will take to resmaple the first time, there is little the 
algorithm can do to fix this. On the other end of the specture, if the time
series don't match the model at all, particle deprivation will occur extremely
quickly, and even the recovery technique discussed in \autoref{sec:Resampling}
won't help. The upshot of this is that the particle filter is able to 
identify these sections very quickly, and thus not waste much time there.
Of course, if a fat-tailed distribution is used for the weighting function,
or the standard deviation of a Gaussian weighting function is very large,
the particle filter will simply converge to meaningless values.

\begin{figure}
\caption{Particle Filter converging to values that make little sense,
because the voxel did not correlate with the input in any known way}
\end{figure}

\section{Weighting Function Comparison}
\label{sec:Results Weights}

\section{Single Time-Series Simulation}

Graphs: 

For simulated data, single timeseries:

For \{delta, DC/Spline\}, \{exponential, gaussian, cauchy\}, \{biased, unbiased initial\},
\{100, 500, 1000\} particles
\begin{enumerate}
\item Ground truth vs. Estimated signal during particle filter run
\item Ground truth vs. Estimated signal with final parameter set
\item True Parameters vs. Final Parameter Sets
\item Variance of final parameters when faced with same ground truth, different noise
\item MSE of (a new timeseries based on X(t) vs. ground truth) for all t
\item Estimator Variance based on different noise runs
\item Final Particle Distribution
\end{enumerate}

For Simulated Data, Full Volume:

%note to self, epsilon should probably be uniform between 0 and something
\section{Simulated Volume}
\begin{enumerate}
\item Parameter Map 
\item Error map of parameters
\item Histogram of \%errors between parameters
\item Activation Map based on a single region with high $\epsilon$, compared with linear
\end{enumerate}

Final parameter distribution among active regions.
Q-Q plots?

\section{FMRI Data}
....

image comparing epsilon-map with GLM activation map


%\chapter{Real Data}
%Finally, we also performed inference on a real FMRI scan. The scanner we used...
%... more specifics...

% TODO include single?
%Before performing tests on a full image, I the particle filter
%on regions deemed active and inactive by statistical parametric mapping
%(SPM). This served the purpose of adjusting the priors as well as the 
%preprocessing based on real world signals. This was actually done before
%the simulations, and then results were carried back the simulations 
%to check consistency.
%After work adjusting parameters, most importantly the weighting function and the 
%priors, particle filter was applied to every voxel in an FMRI image.
%The results of this large scale analysis was a parameter map which was
%then used to calculate normalized square-root MSE image. 
%
%\section{Single-Voxel Analysis}
%The choice of a prior, as discussed previously, is extremely important. While a
%prior may have the potential to give good results, being a monte-carlo algorithm
%there is the possibility for inconsistencies. Thus, increasing the variance
%of the time-constants may allow additional flexibility, it will also cause
%additional model variance. Before running on a full volume I adjusted the 
%prior to ensure that the same input would give the same output 100 times in a 
%row. While this may seem like a given, with a random drawing of the prior,
%this can be difficult. Case in point, the exact same algorithm run twice
%with the time constants all having standard deviations of $.35$ resulted in two
%very different fits, shown in \autoref{fig:badfit_param1}.
%
%\begin{figure}
%\subfigure{\includegraphics[clip=true,trim=6cm 2cm 6cm 3.5cm,width=17cm]{images/badfit_param1}}
%\subfigure{\includegraphics[clip=true,trim=6cm 2cm 6cm 3.5cm,width=17cm]{images/goodfit_param1}}
%\caption{The same priors gave rise to both fits.}
%\label{fig:badfit_param1}
%\end{figure}
%
%For this reason, I actually lowered the standard deviations of the time
%constants to prevent over-smoothing. This resulted in more consistent,
%though potentially slightly worse fits, two examples of which are 
%shown in \autoref{fig:param2_var}. 
%
%\begin{figure}
%\subfigure{\includegraphics[clip=true,trim=6cm 2cm 6cm 3.5cm,width=17cm]{images/param2a}}
%\subfigure{\includegraphics[clip=true,trim=6cm 2cm 6cm 3.5cm,width=17cm]{images/param2b}}
%\caption{A poor fit, using the same parameters as }
%\label{fig:param2_var}
%\end{figure}
%
%todo: stats of the 100 fits?
%
%\section{Single Time-Series Simulation}
%
\chapter{Real Data}
\label{sec:RealData}
Modeling the BOLD response is of course not of much use if it is only done in a 
single voxel. Although this algorithm will hopefully lead to more novel methods 
of analysis, the standard use for modeling the BOLD signal is to locate "activation".
By activation I really mean areas where the input correlates with the BOLD response, 
thus indicating that the stimuli is driving some sort of neural activity. Of course 
much more is going on behind the scenes, brain regions are stimulating other brain
regions, but before that can be analyzed we must validate the ability of the particle
filter to be able to accurately find and model activation. Once areas where the BOLD 
model may be accurately estimated are found, integrating the model will allow for accurate
simulation between measurements, which then may be used for more advanced analysis.
Again though, it all begins with localizing the first activation regions in the
chain. Therefore here I compare the output of the particle filter based modeling
solution to the conventional SPM method. 

In reality, the SPM8 method is a different animal from the algorithm described here.
First SPM preprocesses the image by spatially smoothing the FMRI image (here with 
an $8mm x 8mm x 8mm$ Gaussian kernel), whereas
this is not done in the particle filter algorithm. Additionally, a spline
is used to de-trend, rather than SPM8's high pass filter with a cut
off based on a globally estimated autocorrelation. Thus the preprocessing pipelines 
are different, and the output of SPM8 is single t-statistic, whereas the output 
of the particle filter is a posterior probability distribution of the parameters
at every single voxel. To validate the quality of the particle filter results though
it is necessary to compare the location of
"activated" voxels, and the goodness of fit provided by each method. 

The results from SPM8 are shown in \autoref{fig:hm_canon_spm}, and the results from 
the particle filter are shown in \autoref{hm_canon_pfilter}.

\begin{figure}
\subfigure[]{\label{fig:hm_spm} \includegraphics[scale=.66]{images/spm_hm}}
\subfigure[]{\label{fig:hm_canon_spm_x} \includegraphics[scale=.85]{images/spm_hm_x}}
\subfigure[]{\label{fig:hm_canon_spm_y} \includegraphics[scale=.85]{images/spm_hm_y}}
\subfigure[]{\label{fig:hm_canon_spm_z} \includegraphics[scale=.85]{images/spm_hm_z}}
\subfigure{\label{fig:scale_spm} \includegraphics[scale=.3]{images/scale1}}
\caption{Sagittal, coronal and axial slices of SPM results (\autoref{fig:hm_canon_spm_x} \autoref{fig:hm_canon_spm_y} 
         \autoref{fig:hm_canon_spm_x}), as well as a series of axial slices, \autoref{fig:hm_spm}. Units
         of activation are in Student's T-scores. Higher indicates higher assurance that the signal cannot
         have occurred through noise alone.}
\label{fig:hm_canon_spm}
\end{figure}

\begin{figure}
\subfigure[]{\label{fig:hm_pfilter} \includegraphics[scale=.66]{images/pfilter_hm}}
\subfigure[]{\label{fig:hm_canon_pfilter_x} \includegraphics[scale=.85]{images/pfilter_hm_x}}
\subfigure[]{\label{fig:hm_canon_pfilter_y} \includegraphics[scale=.85]{images/pfilter_hm_y}}
\subfigure[]{\label{fig:hm_canon_pfilter_z} \includegraphics[scale=.85]{images/pfilter_hm_z}}
\subfigure{\label{fig:scale_pfilter} \includegraphics[scale=.3]{images/scale2}}
\caption{Sagittal, coronal and axial slices of SPM results (\autoref{fig:hm_canon_pfilter_x} \autoref{fig:hm_canon_pfilter_y} 
         \autoref{fig:hm_canon_pfilter_x}), as well as a series of axial slices, \autoref{fig:hm_pfilter}. 
         Units of match is normalized $\sqrt{MSE}$. The lowest (best) levels were $.7$,
         whereas the worst levels could go higher than 100 (not shown).}
\label{fig:hm_canon_pfilter}
\end{figure}

\begin{figure}
\subfigure[]{\label{fig:hm_pfilter85} \includegraphics[scale=.66]{images/pfilter85_hm}}
\subfigure[]{\label{fig:hm_canon_pfilter85_x} \includegraphics[scale=.85]{images/pfilter_hm85_x}}
\subfigure[]{\label{fig:hm_canon_pfilter85_y} \includegraphics[scale=.85]{images/pfilter_hm85_y}}
\subfigure[]{\label{fig:hm_canon_pfilter85_z} \includegraphics[scale=.85]{images/pfilter_hm85_z}}
\subfigure{\label{fig:scale_pfilter85} \includegraphics[scale=.3]{images/scale3}}
\caption{Sagittal, coronal and axial slices of SPM results (\autoref{fig:hm_canon_pfilter_x} \autoref{fig:hm_canon_pfilter_y} 
         \autoref{fig:hm_canon_pfilter_x}), as well as a series of axial slices, \autoref{fig:hm_pfilter}. 
         Units of match is normalized $\sqrt{MSE}$. The lowest (best) levels were $.7$,
         whereas the worst levels could go higher than 100 (not shown).}
\label{fig:hm_canon_pfilter85}
\end{figure}

\begin{figure}
\subfigure[Particle Filter]{\label{fig:comp1pfilter} \includegraphics[clip=true,trim=5cm 1cm 4cm 1cm,width=15cm]{images/1_pfilter_37_14_7}}\\
\subfigure[SPM]{\label{fig:comp1spm} \includegraphics[clip=true,trim=5cm 1cm 4cm 1cm,width=15cm]{images/1_spm_37_14_7}}
\caption{}
\label{fig:comp1}
\end{figure}

\begin{figure}
\subfigure[Particle Filter]{\label{fig:comp2pfilter} \includegraphics[clip=true,trim=5cm 1cm 4cm 1cm,width=15cm]{images/2_pfilter_34_12_7}}\\
\subfigure[SPM]{\label{fig:comp2spm} \includegraphics[clip=true,trim=5cm 1cm 4cm 1cm,width=15cm]{images/2_spm_34_12_7}}
\caption{}
\label{fig:comp2}
\end{figure}

\begin{figure}
\subfigure[Particle Filter]{\label{fig:comp3pfilter} \includegraphics[clip=true,trim=5cm 1cm 4cm 1cm,width=15cm]{images/3_pfilter_23_21_7}}\\
\subfigure[SPM]{\label{fig:comp3spm} \includegraphics[clip=true,trim=5cm 1cm 4cm 1cm,width=15cm]{images/3_spm_23_21_7}}
\caption{}
\label{fig:comp3}
\end{figure}

\begin{figure}
\subfigure[Particle Filter]{\label{fig:comp4pfilter} \includegraphics[clip=true,trim=5cm 1cm 4cm 1cm,width=15cm]{images/4_pfilter_26_15_7}}\\
\subfigure[SPM]{\label{fig:comp4spm} \includegraphics[clip=true,trim=5cm 1cm 4cm 1cm,width=15cm]{images/4_spm_26_15_7}}
\caption{}
\label{fig:comp4}
\end{figure}

\begin{figure}
\subfigure[Particle Filter]{\label{fig:comp5pfilter} \includegraphics[clip=true,trim=5cm 1cm 4cm 1cm,width=15cm]{images/5_pfilter_25_34_25}}\\
\subfigure[SPM]{\label{fig:comp5spm} \includegraphics[clip=true,trim=5cm 1cm 4cm 1cm,width=15cm]{images/5_spm_25_34_25}}
\caption{}
\label{fig:comp5}
\end{figure}


\chapter{Discussion}
\label{sec:Discussion}
\section{Review of Results}
% Overview of results
%% Parameters under-constrained
%% Output estimates good
There are two possible causes for the large variance in
parameter estimates. First, it could be that the particle filter is
incapable of learning the model. However, given the high quality
of BOLD estimates, this is almost certainly not the case. The other 
possible explanation is that the given stimulus is not sufficiently
varied to bring out all the properties of the system. While it is 
possible that this is the case, it is unlikely that even the best 
stimulus sequence could differentiate all the parameters. The fact
that \autoref{sec:VeryLongSim}, which used randomly spaced impulses,
still had a high correlation definitely indicates that the parameters
are ill-defined. 

Thus there is little doubt that the parameters of the BOLD model are under
constrained. While unsurprising given the sensitivity analysis by Deneux et al.,
it is a notable conclusion \cite{Deneux2006}. Other methods
that depend on a point estimate of the parameters, such as least squares 
or Kalman filters (which uses the first two moments) are limited in their
capacity to estimate such parameters. While estimates of
the  BOLD signal may still be correct, the 
underlying parameters and state variables cannot be described without using
a joint probability distribution function. In this sense, particle 
filters represent an important step forward in BOLD parameter 
estimation. Representing uncertainty with a mean
and variance is insufficient; so using a particle filter
or Bayesian estimate of the posterior is not simply an enhancement, 
but a necessary precaution.

Because of the ill-defined nature of the parameters, the results 
of Friston et al. may have been wider than true parameter distribution
\cite{Friston2002}. Future works that attempt to calculate parameters
of the BOLD model should be run multiple times to ensure consistency,
which most likely cannot be attained. This result is particularly
troublesome given the number of papers that use those parameters for
a prior distribution.

As the results in \autoref{sec:RealData} show, the heatmaps, especially
those of mutual information, closely resembled the results of SPM. While the activation
tests were more sensitive, there were some additional false positives, 
though the problem is difficult to quantify. Regions with M.I. above 
$0.15$ consistently fit the FMRI data well, and simulations showed that
the particle filter performed extremely well in the face of significant
noise. In sum, the estimated BOLD output remained consistent despite
large swings in parameters.

% Pros/Cons
%% Cons
%%% Computation longer than SPM
%%% Harder to interpret
%% Pro
%%% Non-parametric
%%% Real time
%%% Full Posterior
%%% More intuitive
\section{Particle Filter Review}
The Particle Filter algorithm was originally designed for on-line parameter 
estimation. For this reason, there is no guarantee of optimality or even 
convergence for finite measurements. However, for the BOLD nonlinear ODE 
this is less of a concern than it might first appear to be. For this
particular problem there can be no guarantee of a global minimum, and although
other techniques guarantee a local minimum, tests show that the particle
filter did converge relatively quickly \autoref{sec:VeryLongSim}.

One difficulty
with the use of a particle filter when given a finite number of measurements is finding
a good weight function $P(y_k | x_k)$. This is more important for a finite
number of measurements because $P(y_k | x_k)$ needs to converge in finite time.
In spite of this potential problem, \autoref{sec:VeryLongSim} managed
to converge in less than 500 seconds.  Given sufficient
measurements it is better to let the algorithm take longer to converge, because the
convergence will be less prone to particle deprivation. The particle filter takes longer
to run than Volterra approximation method from \autoref{sec:Background Linear Approximation};
however, it is free from the uncertainty of whether a quadratic approximation is 
sufficient for the BOLD model. 

The particle filter also has advantages over other estimation procedures
discussed in \autoref{sec:Prior Works}. The most important advantage is that it provides
an estimate of the posterior probability, rather than a single estimate. While researchers
often want a simple estimate of parameters, such an estimate is impossible with this particular
model. The fact that the final distribution does not need to conform to a
parametric distribution is also advantageous, given the nonlinearities in the system.
While the particle filter took a day to calculate for full brain calculations, its speed
was sufficient on a quad core machine to perform real time calculations of small regions
(approximate run time .27 seconds per voxel-measurement). Today it would be possible
 to perform real time analysis of 10 voxels on an average quad core. The algorithm also scales
well and does not require burdensome amounts of memory (approximately 11 megabytes). 

A more practical benefit with the particle filter is that it is mathematically
simple. An understanding of Bayesian statistics is all that is necessary to understand
how the particle filter works.  Additionally very few assumptions
are needed for the particle filter. Few assumptions  reduces the risk of those assumptions
being violated. Fewer and more realistic assumptions also make the particle 
filter more robust to unforeseen difficulties in FMRI data. 

%\section{Particle Filter Options}
%There are a large number of options available in the particle filter, that 
%were of little use. 
%\section{Didn't Work}
%A more subtle difference that could be made in detrending is how the percent difference
%is calculated. In this work the spline was subtracted, and then the result
%was divided by the average of the initial signal. A more correct method may be dividing
%by the spline value at that particular point. The reason for not dividing by the spline
%at that point is that it could be less stable, and in a sense the trend has already
%been removed. The difference is not particularly large though 
%(dividing by 1020 rather than 1000 for instance) as long as the spline didn't have heavy swings. 

%Finally, rather than adding a constant to the detrended data before
%applying the particle filter, a DC gain parameter could be added to the model. In
%tests, this could work well, however, it often did not. The problem with this
%could simply be the addition of another degree of freedom without any increase in
%the number of particles. Increasing the number of particles further though
%can be computationally intractable. Thus, while this is in a sense the correct
%solution, it is an impractical one. 

%Starting with a flattened
%prior (by setting weights to be non-uniform after the initial particles have
%been drawn) was not used in the final analysis. The reason for not flattening
%the prior was a practical issue with weighting points in a 
%7 dimensional joint distribution. Often differences in particle weights approached machine 
%precision, meaning that the flattened prior was actually far from flat. Instead
%the distribution and results became unpredictable. This
%is quantization issue that will exist unless the prior were made to be deterministic
%or the initial number particles was raised to computationally prohibitive levels.

%Another improvement that could be made to the prior is increasing the initial variance.
%This would allow the particle filter to learn a wider range of parameters; at the cost
%of decreased convergence rate and support density. In tests, both real and simulated,
%increasing the variance tended to allow the particle filter to over-smooth 
%(ex \autoref{fig:badfit_param1}). In 
%essence the particle filter tended to converge to a mean where the time constants
%dominated the results by smoothing out all the peaks and troughs. The result is that
%no other parameters alter the estimated BOLD signal. 
%\begin{figure}
%\includegraphics[clip=true,trim=6cm 2cm 6cm 3.5cm,width=7cm]{images/badfit_param1}
%\caption{Large variance in time constants over-smoothing.}
%\label{fig:badfit_param1}
%\end{figure}
%In \autoref{fig:badfit_param1}, the particle filter still had not fully converged, and
%given more measurements it could still converge to a better estimate. Thus, increases
%in the prior variance necessarily must be accompanied by more measurements,
%although re-presenting data could be used to artificially extend the measurements. 

%Even more basic, its possible to artificially increase the number of 
%measurements by presenting them the time-series several times. 
%This activity is similar to the process used in neural networks,
%and gives the particle filter longer to converge to the optimal estimate. On the 
%downside, this increases runtime and artificially reduces variance. 
%Thus it does not necessarily improve the results. In tests, it did reduce the
%final covariance but did not lead to better estimates. 

\chapter{Future Work}
\label{sec:FutureWork}
% Enhancements/Dehancements 
%% prior
%%% flatten prior
%%% Larger variance
%% Detrending
%%% \% difference using spline?
%%% "smart knots"
%%% DC gain as a parameter
%%% linearizing
%% Experimental Changes
%%% Variations in stimuli
%%% longer timeseries or artificial
%%% Automatic estimation of measurement error.
\section{Algorithm Improvements}
\label{sec:Particle Filter Variations}
There are a few areas where future research may improve upon the current
work. The first is modifying the prior 
distribution. The prior distribution used in this work is listed in \autoref{tab:Prior}, 
and is based on the findings of Friston et al. \cite{Friston2000}. Unfortunately, that
result depended on a quadratic approximation (Volterra Series) which has
not been extensively tested (at least not in a published work). 
Additionally, the current prior is
based only around the BOLD output, which is why it is good at generating
estimates of the BOLD signal, but weak in estimating parameters. 
The best solution would be to have
in-vivo estimates of the actual parameters, although this
is unlikely to happen. Instead, better estimates could be found
using population studies with additional measurements as discussed in 
\autoref{sec:Sideways Measurements}. Better knowledge of the prior distribution
would make it possible to decrease variance of certain 
parameters, reducing the mobility of the parameters.

Another major area for improvement is removing drift. 
A large number of detrending methods were tested, but every method was in some way limited. 
One viable method that could be utilized, given the right experimental design, is detrending
based on areas of low activity. This has the advantage that it wouldn't require an arbitrary
constant to be added to the pre-processed signal for the BOLD model to fit properly.
The disadvantage of this approach is that it could hide long fall times by normalizing them
out. It also requires periodic breaks in the stimulus, which could reduce value of those
samples for fitting purposes. 

In terms of dealing with drift, linearizing is another possibility.
This would use the delta between measurements for fitting rather than the 
direct value. This has the advantage of not requiring detrending and thus 
gives nearly raw data to the particle filter
for processing. The effectiveness of this method depends directly on the type of stimulus.
In a test with rapid impulse stimuli, this could be extremely effective because
the DC level has minimal data; conversely prolonged flat level levels make this less effective.
In tests I found that large drift-low white noise
type signals performed much better with a linearization approach, as one might expect. 
For the stimulus sequence used in \autoref{sec:RealData}, the results tended to be worse
than using spline detrending.

As discussed in \autoref{sec:Methods Weighting Function}, choosing a weighting
function is difficult, and the optimal solution varies based on the input.
Automatically estimating measurement error could improve
the quality of the particle filter results. Although I made some attempts to do
this, finding a generic, consistent solution is complex, and often depends on the
experimental design. Part of the problem is that SNR varies greatly across
the brain, and there is no way to actively measure noise without also knowing
the underlying signal. One possible solution is having the particle filter automatically
set the weight based on the total particle weight, or on the prevalence of resampling.
 
\subsection{Experimental Changes}
\label{sec:Sideways Measurements}
One definite way of improving the results of the particle filter is additional
measurements. While increasing the sample rate of FMRI scanners may
not be possible, simultaneous measurements of volume and flow is
possible, albeit at 9T in a cat \cite{Hu2009}. Just one of those measurements
though would be extremely powerful when incorporated into the BOLD
model. By adding another measurement, the
variance in the parameter estimates would significantly drop. The fact
that such a measurement would be closer to the true stimulus would make it
all the more powerful. 
A more conventional method of adding measurements is to 
simply perform longer FMRI tests. Although this wouldn't be 
groundbreaking, it would certainly increase the ability of the 
particle filter to develop the posterior distribution. 
It may also be possible to concatenate multiple FMRI tests together.
It should be possible to link the multiple runs simply through
the final parameter distribution from the previous run. 

Given the non-linearities in the system, differences in stimuli
could make a large difference in the observability of parameters. The mentality 
for using a physiologically based nonlinear model for BOLD signal is to model 
those nonlinearities. Its logical then that certain nonlinear parameters may not
be identifiable when the input is primarily an impulse response. It 
has been reported that short responses are disproportionately large 
in FMRI data \cite{Miller2001, Deneux2006}. Therefore,
a wide range of activation hold time may shed further light on 
the parameters' distribution as well as the validity of the BOLD model. 

% Future works
%% Using s as the input to other regions. 
%% Comparison of posterior of parameters
\section{Future Applications}
There are a number of advantages to the particle filter approach presented
here. In the past, FMRI data has been analyzed strictly for determining
correlation between a stimulus and response. With this new method
the correlation is simply a means to determining the joint posterior distribution
of the parameters. While only regions that correlate with the input will 
be calculable, this method exploits that correlation to constrain the 
prior distribution of the parameters. The resulting distribution, while difficult 
to visualize because of the high-dimensionality, could nevertheless be 
correlated with neural pathologies. In spite of the fact that the parameters
are under-determined, the final distribution is still a reduction in the
uncertainty of the parameters. The availability of a full posterior distribution
opens up many avenues for further inquiry, for instance
to compare normal vs. symptomatic
populations. This would be especially useful in patients whose symptoms
do not include structural changes in brain. This would effectively be
a way to differentiate the way the brain operates. 

Because of limitations present in every imaging modality,
its becoming increasingly clear that combining data from multiple sources
will be necessary to push Neurology forward. In order to do so however, the
output of each source needs to fully represent what information that source
can provide. Combining the sources using Bayesian statistics is promising
yet often difficult because full probability distributions are hard to come by.
However, in this case, the particle filter provides a full posterior which
is extremely versatile. Therefore, future works will easily be able to
plug in data from multiple sources if they all output Bayesian posteriors.
For instance, if two different modalities have calculated the probability
distribution of neural efficiency, those two beliefs may be combined into
one conditional belief for the probability of neural efficiency. 

An advantageous aspect of using a physiological model such as this, is that
it permits estimates of otherwise hidden parameters. In particular the BOLD
model gives an estimate of the value of the flow inducing signal, $s$. Having
this value available opens up new avenues for determining inter-regional
dependencies. The values of $s$ could serve as a proxy for activation in 
that region and thus could be used to drive other inputs regions' $u(t)$. 
Thus, the particle filter could be re-run
with the time-series of a particular voxel's $s$ value as an additional stimulus.
In this way, it could be possible to determine chains of events. This is just
one possible benefit being able to determine the time course of the hidden
state variables present in the BOLD equations; the potential benefits of being
able to determine this information is limitless. Of course the quality of these
estimates will continue to improve with the model priors. 



\chapter{Conclusion}
\label{sec:Conclusion}
This work has proposed and demonstrated the use of a particle filter
for the estimation of BOLD model parameters. Because many of the parameters
were under constrained from the point of view of the BOLD response, a
full posterior estimate was a more logical approach to parameter estimation
than typical point estimates of parameters. The particle filter also converged
quickly to correct output estimates, and thus provides a framework
for future real-time FMRI experiments. Final BOLD estimates were also comparable
to the results of SPM. 

% Overview of results
%% Parameters under-constrained
%% Output estimates good
\section{Review of Results}
The results unequivocally show that the parameters of the BOLD model are under
constrained. While unsurprising given the sensitivity analysis in \cite{Deneux2006},
it is still an important limitation when calculating parameters. Thus, other methods
that depend on a single point estimate of the parameters, such as kalman filters
or even least squares are limited to estimating the BOLD signal, but the estimate
of the underlying parameters and state variables are suspect at best. Similarly, as the
histograms in \autoref{sec:Multi-voxel Simulation} and \autoref{sec:Real Data Parameter Estimates}
show, using the mean to estimate parameters also makes little sense in the particle filter
case. Attributing the variance of estimate to the underlying parameter distribution would
also be a mistake. While it is definitely true that parameters vary from person to person
and region to region, the estimated distributions in \autoref{sec:Real Data Parameter Estimates} 
are smoothed versions of those distributions. For this reason, any analysis of parameters should
take the entire estimated posterior distribution into account, and perform F-tests on that
distribution, rather than an idealized Gaussian. In this sense, particle filters represent
an important step forward in BOLD parameter estimation. It is now clear that representing
the uncertainty in parameters as a Gaussian is insufficient; so using a particle filter
or Bayesian estimate of the posterior is not simply an enhancement, but a necessary precaution.

Although point estimates of parameters are not dependable estimates of the 
true parameters, they are by no means usefulness. Analysis of differences in parameters
could be medically relevant. Additionally, a single estimate for parameters is still able
to form a good estimate of the BOLD signal. This factor adds bridge to 
earlier types of activation tests, such as the statistical
parametric mapping. As the results in \autoref{sec:RealData} show, the heatmaps, especially
those of mutual information closely resembled the results of SPM. While the activation
tests were more sensitive, they were also more prone to false positive. However, it is
worth noting that certain enhancements could be useful in reducing false positives; for instance
by using the maximum likelihood of the final distribution, or even the median. The 
usefulness of these techniques depend on the degree to which the algorithm converged, 
and thus on the experimental design. Future work may shed further light on these
techniques. 

% Pros/Cons
%% Cons
%%% Computation longer than SPM
%%% Harder to interpret
%% Pro
%%% Non-parametric
%%% Real time
%%% Full Posterior
%%% More intuitive
\section{Particle Filter Approach}
The Particle Filter algorithm was originally designed for on-line parameter 
estimation. For this reason, there is no guarantee of optimality or even 
convergence for finite measurements. However, for the BOLD nonlinear ODE there
can be no guarantee of an optimal solution. A guaranteed local minimum,
which nonlinear least squares provides, however would be helpful. One difficulty
with the use of a particle filter with finite data is the calculation of a good
weight function $P(y_k | x_k)$  that will converge at a decent rate. If the weight
function does not sufficiently differentiate particles, the final distribution will 
no be significantly different from the prior distribution. On the other hand, if the 
weight function is too thin, it will unfairly eliminate viable particles. Given sufficient
measurements it is better to let the algorithm take longer to converge, because the
convergence will be better (more accurate). The particle filter takes longer
to run than Volterra approximation method from \autoref{sec:Background Linear Approximation};
however, it is free from the uncertainty of whether a quadratic approximation is 
sufficient for the BOLD model. 

The particle filter certainly has significant advantages over other estimation procedures
discussed in \autoref{sec:Prior Works}. The most important advantage is that it provides
an estimate of the posterior probability, rather than a single estimate. While it is natural
to want a simple estimate of parameters, such an estimate is impossible with this particular
model. The results are more difficult to interpret, but this is a necessity. The fact that 
the final distribution is not dependent on any particular distribution is also advantageous.
Of particular note; the final distribution does not need to conform to any parametric
distribution. While the particle filter was not fast for full brain calculations, its speed
was sufficient on a quad core machine to perform real time calculations of small regions
(approximate run time .27 seconds per voxel-measurement). Today it would be possible
 to perform real time analysis of 10 voxels on an average quad core. The algorithm also scales
well and does not require burdensome amounts of memory (approximately 11 megabytes). 
For this reason this algorithm is perfect for extension to vector or video card
based processors. 

A more practical concern with the particle filter is that it is mathematically
simple. Only a basic understanding of Bayesian statistics is needed to understand
how the particle filter works. This is in contrast to the Volterra based approach
of \cite{Friston2000}, which is quite complex. Additionally very few assumptions
are needed for the particle filter. It is a common problem in traditional statistics 
for assumptions to be unrealistic which means the results may be invalid or at 
least in question. 
Fewer and more realistic assumptions mean that the particle filter is more robust
to unforeseen difficulties in FMRI data. 

% Enhancements/Dehancements 
%% flatten prior
%% \% difference using spline?
%% "smart knots"
%% DC gain as a parameter
%% linearizing

% Future works
%% Extensive test of quality of volterra kernel estimate from friston
%% Better de-trending
%% Automatic estimation of measurement error.

As result, it is possible to use
the particle filter to localize areas where a known stimuli most directly drives
neural activation. In the future it will be possible to use the estimated states
($v,q,s,f$) to drive other models and learn more about how regions of the brain
interact. 

A limitation often reached in previous works was an inconsistency of
parameter estimates, most likely because of covariance between the model parameters.
Although individual studies got consistent results, those results often differed
widely from other similar studies. The reason for this is rather clear from the
simulation results in \autoref{sec:SimLowNoise}. There is a significant amount of
trade off between parameters to the point that a signal set of parameters
is most likely not possible to derive from the BOLD response alone. It
will therefore be beneficial to combine BOLD studies with cerebral blood
flow or cerebral blood volume studies to gain multiple more measurements and
further constrain the model. That said, the benefit of the particle filter
is that it provides a full posterior distribution at the final time step.
As such the true solution should be encoded in the particle filter's final 
distribution given priors that encompass the true parameters. This is beneficial
in two ways; first, if, after the fact, some parameter becomes known 
from outside observation, it is then possible to construct a new probability
conditional on the new observation. Secondly the results from multiple runs
may be reasonably concatenated, using the final distribution from the previous
run as the prior distribution of the next run. Rather than simply providing a 
staring point for parameters to converge from, it in fact continues convergence
from the previous stopping point. 

Although many versions of the BOLD model exist, and it is tempting to use 
more detailed models; from the results found here the issue of bias error
from the BOLD model is not the biggest concern. Clarifying the distributions
of parameters for the prior should be the first concern; currently no multi-patient
full-volume studies have been done to estimate parameters.
One future study that would be beneficial in this way  would
be an extensive study of what the priors should truly be. Although \cite{Friston2000}
gives an estimate of what is thought to be reasonable values, and later studies
published their estimate of the distributions, given the interplay between
parameters it is unlikely that these priors are true to actual distribution
that occurs in vivo. As I mentioned previously, the addition of simultaneous
flow or volume measurements are another potentially powerful way to further confine the model,
and thus deal with the elasticity of parameters. As I mentioned in 
\autoref{sec:BackgroundConclusion}, a chief advantage of using physiologically
plausible models is that such data may in fact be added with relative ease.

Automatic detection of the noise level in the signal, to get a decent wieghting
function. Large scale activation to get more parameter estimates. 
Viscoelastic effects from \cite{Buxton2004}, discussed in \cite{Deneux2006}.

In conclusion, using particle filters to estimate the BOLD response are 
a powerful method of fitting to noisy data. The technique also
holds great promise as extensible platform to build more advanced models
and techniques on top of. Integrating further information is necessary to
move beyond the traditional Statistical Parametric Mapping and moving
toward biologically and medically relevant FMRI scanning techniques. 


\bibliographystyle{plain}
%\bibliographystyle{abbrvnat}
%\bibliographystyle{apalike}
%\bibliographystyle{abbrvnat}
%\bibliographystyle{abbrv}
\bibliography{library}

\end{document}

%\chapter{Introduction}
%\markright{Albert J. Kippleby \hfill Chapter 1. Introduction \hfill}
%
%William Shakespeare has profoundly affected the field of literature
%worldwide.  In the United States there was a surge of Shakespearean
%literature starting in the 1960s, with the opening of the Montgomery
%Shakespearean festival and continuing into the present ...

%%%%%%%%%%%%%%%%%
%
% Include an EPS figure with this command:
%   \epsffile{filename.eps}
%

%%%%%%%%%%%%%%%%
%
% Do tables like this:

%%%%%%%%%%%%%%%%%%%%%%%%%%%%%%%%

%%
%% PROJECT: <ETD> Electronic Thesis and Dissertation Initiative
%%   TITLE: LaTeX report template for ETDs in LaTeX
%%  AUTHOR: Neill Kipp, nkipp@vt.edu
%%     URL: http://etd.vt.edu/latex/
%% SAVE AS: etd.tex
%% REVISED: September 6, 1997
%% 
%
%% Instructions: Remove the data from this document and replace it with your own,
%% keeping the style and formatting information intact.  More instructions
%% appear on the Web site listed above.
%
%\documentclass[12pt,dvips]{report}
%
%\setlength{\textwidth}{6.5in}
%\setlength{\textheight}{8.5in}
%\setlength{\evensidemargin}{0in}
%\setlength{\oddsidemargin}{0in}
%\setlength{\topmargin}{0in}
%
%\setlength{\parindent}{0pt}
%\setlength{\parskip}{0.1in}
%
%% Uncomment for double-spaced document.
%% \renewcommand{\baselinestretch}{2}
%
%% \usepackage{epsf}
%
%\begin{document}
%
%\thispagestyle{empty}
%\pagenumbering{roman}
%\begin{center}
%
%% TITLE
%{\Large 
%Use of Metaphor in Shakespeare's Plays and its Potential
%Application in Twenty-first Century Literature
%}
%
%\vfill
%
%Albert J. Kippleby
%
%\vfill
%
%Dissertation submitted to the Faculty of the \\
%Virginia Polytechnic Institute and State University \\
%in partial fulfillment of the requirements for the degree of
%
%\vfill
%
%Doctor of Philosophy \\
%in \\
%Literature and Technology
%
%\vfill
%
%Neill A. Kipp, Chair \\
%Emilio J. Arce \\
%Scott A. Guyer \\
%Laura Weiss
%
%\vfill
%
%July 16, 1997 \\
%Blacksburg, Virginia
%
%\vfill
%
%Keywords: Metaphysics, Information Retrieval, Spacecraft
%\\
%Copyright 1997, Albert J. Kippleby
%
%\end{center}
%
%\pagebreak
%
%\thispagestyle{empty}
%\begin{center}
%
%{\large Use of Metaphor in Shekespeare's Plays and its Potential
%Application in Twenty-first Century Literature}
%
%\vfill
%
%Albert J. Kippleby
%
%\vfill
%
%(ABSTRACT)
%
%\vfill
%
%\end{center}
%
%The need for concrete examples increases when technology becomes
%difficult to explain.  In documentation for computer systems
%especially, we see a wide audience of field experts attempting to
%comprehend documentation for computer software and hardware of which
%they should only require a cursory understanding.  Additionally, as
%the pace of the information age quickens we see document authors
%struggle for \textit{examplia-concretes} with wide applicability, and
%consistently rely on excerpts from Shakespearean literature as a
%public-domain source for their various explications.
%
%We predict the twenty-first century will be no different.  Actuarial
%studies show explosion in the information industry such that four out
%of five persons will be \textit{bona fide} electronic document
%authors; many of those will have one or more college degrees.  We
%prove through computer simulation \textsc{Machinum Simitatores} that
%authors of twenty-first century literature will be affected by these
%examples and will include metaphor with Shakespearean source into
%their writing with increasing frequency.
%
%\vfill
%
%% GRANT INFORMATION
%
%That this work received support from the Southeastern Universities
%Research Association (SURA) ``Monticello Library Project'' is purely
%coincidental.
%
%\pagebreak
%
%% Dedication and Acknowledgments are both optional
%% \chapter*{Dedication}
%% \chapter*{Acknowledgments}
%
%\tableofcontents
%\pagebreak
%
%\listoffigures
%\pagebreak
%
%\listoftables
%\pagebreak
%
%\pagenumbering{arabic}
%\pagestyle{myheadings}
%
%\chapter{Introduction}
%\markright{Albert J. Kippleby \hfill Chapter 1. Introduction \hfill}
%
%William Shakespeare has profoundly affected the field of literature
%worldwide.  In the United States there was a surge of Shakespearean
%literature starting in the 1960s, with the opening of the Montgomery
%Shakespearean festival and continuing into the present ...
%
%%%%%%%%%%%%%%%%%%
%%
%% Include an EPS figure with this command:
%%   \epsffile{filename.eps}
%%
%
%%%%%%%%%%%%%%%%%
%%
%% Do tables like this:
%
% \begin{table}
% \caption{The Graduate School wants captions above the tables.}
%\begin{center}
% \begin{tabular}{ccc}
% x & 1 & 2 \\ \hline
% 1 & 1 & 2 \\
% 2 & 2 & 4 \\ \hline
% \end{tabular}
%\end{center}
% \end{table}
%
%%%%%%%%%%%%%%%%%%%%%%%%%%%%%%%%%
%
%% If you are using BibTeX, uncomment the following:
%% \thebibliography
%%
%% Otherwise, uncomment the following:
%% \chapter*{Bibliography}
%
%% \appendix
%
%% In LaTeX, each appendix is a "chapter"
%% \chapter{Program Source}
%
%
%% Finally, the VITA
%\chapter*{Vita}
%
%Albert was born on a sunny day...
%
%\end{document}
