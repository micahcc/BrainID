\chapter{Conclusion}
\label{sec:Conclusion}
This work has demonstrated the use of the particle filter to
learn parameters of the BOLD model. Since the inception of the
BOLD equations, many attempts have been made to learn the parameters,
given FMRI data. These attempts have typically either been extremely 
slow or made extensive approximations. While the particle filter method
is not quick, 60 seconds to analyze each voxel is well within the capabilities
of the typical research lab. 

One significant finding of this work, that would not be clear without 
calculating a full posterior distribution, is the interplay 
between the parameters. The results of simulations clearly demonstrate
that identifying a single set of parameters is not possible with this 
model. Although sensitivity tests in \cite{Deneux2006} certainly hinted
at this, the current work clearly demonstrates this fact. Therefore,
any single estimate of parameters is insufficient for analysis. As such,
approaches to the BOLD model that do not treat the parameters as distributions
will not be able to overcome the inherent indeterminability of the
parameters. Besides the Unscented Kalman Filter, this is the only approach
that accomplishes this task without recalculating the parameters over and 
over to build a distribution. The Unscented Kalman Filter, though, is limited
to Gaussian estimation. 


