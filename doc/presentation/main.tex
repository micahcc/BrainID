%Micah Chambers

\documentclass{beamer}
\documentclass[handout,dvips,11pt,gray]{beamer}
%\usepackage[orientation=landscape,size=custom,width=16,height=10,scale=0.5]{beamerposter} 

\usepackage{pgfpages}
%\setbeameroption{show notes on second screen}
\setbeameroption{show only notes}

\mode<presentation>
{
  \usetheme{Antibes}
  % or ...

  %\setbeamercovered{transparent}
  % or whatever (possibly just delete it)
}

%\usefonttheme[onlylarge]{structurebold}
%\usecolortheme{crane}
%\setbeamerfont*{frametitle}{size=\normalsize,series=\bfseries}
%\setbeamertemplate{navigation symbols}{}
%\setbeamercovered{transparent}


\usepackage{color}
\usepackage{colortbl}
\usepackage[pdftex]{graphicx}
\graphicspath{{pdf/}{png/}}
%\usepackage{times}
%\usepackage[T1]{fontenc}
%\usepackage[small]{caption}
\usepackage{algorithm}
\usepackage{algorithmic}
\usepackage{subfigure}

\usepackage{amsmath}	% math fonts
\usepackage{amsthm}
\usepackage{amsfonts}

%plots
\usepackage{tikz}

\definecolor{mblue}{RGB}{178,34,34}
\definecolor{mred}{RGB}{0,0,128}
\definecolor{mgreen}{RGB}{0,100,0}
\definecolor{myellow}{RGB}{176,169,0}

\title{Full Brain BOLD Signal Parameter Estimation \\
using Particle Filters}

%\subtitle{}

\author{Micah Chambers}
\institute{Virginia Tech Bioimaging Systems Lab}

\subject{Medical Imaging}

% If you have a file called "university-logo-filename.xxx", where xxx
% is a graphic format that can be processed by latex or pdflatex,
% resp., then you can add a logo as follows:

%\pgfdeclareimage[width=1.5cm]{university-logo}{logo}
\logo{\includegraphics[width=1.5cm]{logo}}


% If you wish to uncover everything in a step-wise fashion, uncomment
% the following command: 

%\beamerdefaultoverlayspecification{<+->}

\AtBeginSection{
\begin{frame}
\begin{center}
\structure{\Huge \insertsection}
\end{center}
\end{frame}
}

\AtBeginSubsection{
\begin{frame}
\begin{center}
\structure{\Huge \insertsubsection}
\end{center}
\end{frame}
}

\begin{document}
\begin{frame}
  \titlepage
\end{frame}

%0
\begin{frame}{Outline}
  \tableofcontents
  % You might wish to add the option [pausesections]
  \note{
  \begin{itemize}
  \item Introduction - FMRI and the Balloon Model
  \item Nonlinear Regression - overview prior works and the particle filter
  \item Methods - how the particle was configured, and the preprocessing
  \item Results - Several different types of simulation and real data 
  \item Conclusion - Why this is important
  \end{itemize}
  }
\end{frame}

\section{Introduction}
\begin{frame}{Balloon Model}
\begin{figure}
\includegraphics[width=10cm]{model2}
\caption{
    \tiny
    \cite{Riera2003}
}
\end{figure}
\note{
    \begin{itemize}
        \item FMRI has been around for about 20 years
        \item 10 Years Since the First BOLD Model
        \item Basis of FMRI, T2$^*$ relaxation time, blood oxygen levels
        \item Windkessel/Balloon Model, Volume Changes driving fluctations
            in blood oxygen 
        \item Brain has no local oxygen storage, dependent on blood
        \item brain active, firing rates of neurons increase 10 or 100x
        \item recharging requires additional oxygen ... metabolism. 
        \item Increased metabolism drives increased blood flow
        \item Increased blood flow increases volume
        
        \item Rates from below 1 Hz to 5 to 6 Hundred Hz
    \end{itemize}
}
\end{frame}

\begin{frame}{Activation Chain}
\begin{figure}
    \includegraphics[width=\textwidth]{graphs}
\caption{
    \tiny
    \cite{Buxton2004}
}
\end{figure}
\note{
    \begin{itemize}
        \item CMRO2 Increased Metabolism Causes Initial Drop In Oxygen
        \item CBF Flow Quickly Compensates Increases Oxygen
        \item CBV Increases, Decides the outflow, local oxygen content
        \item This delays result in the characteristic BOLD response
        \item CBF CMRO2 Locked 
    \end{itemize}
}
\end{frame}

%more expansion of all the parameters
\begin{frame}{BOLD State Space Model}
  \scriptsize
  \begin{columns}
  \begin{column}{.7\textwidth}
    \begin{itemize}
    \item State Vector:
    \begin{eqnarray}
    \color{mblue}\dot{s}(t) &=& {\color{mred}\epsilon} {\color{mgreen}u(t)} - 
                \frac{\color{mblue}s(t)}{\color{mred}\tau_s} - \frac{{\color{mblue}f(t)}-1}{\color{mred}\tau_f} \nonumber \\
    \color{mblue}\dot{f}(t) &=& \color{mblue}s \nonumber \\
    \color{mblue}\dot{v}(t) &=& \frac{{\color{mblue}f(t)} - 
                {\color{mblue}v(t)} ^ {1/{\color{mred}\alpha}}}{\color{mred}\tau_0} \nonumber \\
    \color{mblue}\dot{q}(t) &=& \frac{1}{\color{mred}\tau_0}
            \left(\frac{{\color{mblue}f(t)}(1-(1-{\color{mred}E_0})^{1/\color{mblue}f(t)})}{\color{mred}E_0} -
        \frac{\color{mblue}q(t)}{{\color{mblue}v(t)}^{1-1/\color{mred}\alpha}}\right) \nonumber 
    \end{eqnarray}
    \item Output:
    $${\color{myellow}y(t)} = {\color{mred}V_0}(a_1( 1 - {\color{mblue}Q(t)}) - a_2(1 - {\color{mblue}V(t)}))$$
    \end{itemize}
  \end{column}

  \begin{column}{.3\textwidth}
    \begin{itemize}
        \item State Variables:
        $${\color{mblue}s(t)}, {\color{mblue}f(t)}, {\color{mblue}v(t)}, {\color{mblue}q(t)}$$
        \item Parameters:
        $${\color{mred}\epsilon}, {\color{mred}\tau_s}, {\color{mred}\tau_f}, {\color{mred}\alpha}, 
                    {\color{mred}\tau_0}, {\color{mred}E_0}, {\color{mred}V_0}$$
        \item Constants:
        $$a_1, a_2$$
        \item Input:
        $$\color{mgreen}u(t)$$
    \end{itemize}
  \end{column}
  \end{columns}
  \note{
    \tiny
    \begin{itemize}
    \item System is dissipative,
    \item Constant input leads to steady state 
    \item $u(t)$ is the input
    \item States
    \begin{itemize}
    \tiny
    \item s - Neural Response (Flow Inducing Signal)
    \item f - Inflow of Oxygenated Blood
    \item v - Local Blood Volume
    \item q - Deoxygenated hemoglobin content
    \end{itemize}
    \item Parameters
    \begin{itemize}
    \tiny
    \item $\epsilon$ - efficiency
    \item $\tau_s$ - Falloff time for Flow Inducing Signal
    \item $\tau_f$ - Onset time for Flow Inducing Signal
    \item $\alpha$ - Grubbs Constant determining outflow volume
    \item $\tau_0$ - Time Constant for volume, deoxygenated hemoglobin
    \item $E_0$ - Resting Oxygen Metabolism
    \item $V_0$ - Resting Metabolism
    \end{itemize}
    \item $a_1$,$a_2$ are functions of the imaging modality,
            and to a vary somewhat with $E_0$. 
    \item $a_1 = k_1*E_0 + k_2*E_0$, $a_2 = k_2*E_0 + k_3$
    \end{itemize}
  }
\end{frame}

\section{Nonlinear Regression}
\subsection{Prior Works}
\begin{frame}{Approximation Method}
  \begin{itemize}
    \item 2nd Order Volterra Kernel \cite{Friston2000}
    \note{Quadratic Approximation}
    \begin{itemize}
        \item Quadratic Convolution used to approximate Jacobian Matrix.
        \note{Because of the nonlinearities in the BOLD model, step size for
                integrating would otherwise be extremely small}
        \item Volterra approximation quality is not known.
    \end{itemize}
    \note{Volterra Approximation depends on sparsity etc}
    {\footnotesize
    $$y(t) = k_0 + \int_{-\infty}^{\infty} k_1(s_1) x(t-s_1) ds_1
        + \int_{-\infty}^{\infty} k_2(s_1,s_2) x(t-s_1)x(t-s_2) ds_1 ds_2$$
        \note{Its possible this isn't even knowable}
    }
    \item Canonical Hemodynamic Response Function
    \note{Linear Approximation}
    \begin{itemize}
        \item No Parameters Estimated
        \item Maximum Likelihood Possible
        \item Inflexible - even to changes in onset time
    \end{itemize}
    \begin{center}
    \includegraphics[clip=true,trim=4.88cm 0cm 5cm 1.5cm,height=2.7cm,width=8cm]{HRF}
    \end{center}
    
  \end{itemize}
\end{frame}

\begin{frame}{Nonlinear Methods}
\begin{itemize}
    \begin{columns}
    \begin{column}{.42\textwidth}
    \item Local Linearization filter, \cite{Riera2003}
    $$f(t)- f(t-1) \sim N(0, \sigma^2)$$
     \note{Jacobian of the $\dot{x}$ is available, making gradient descent possible}
    \item Genetic Algorithms and Simulated Annealing, \cite{Vakorin2007}
     \note{Genetic Algorithms were used to optimize spike points, simulated annealing for the parameters}
    \item Particle Filters to estimate States, \cite{Murray2009}
    \begin{itemize}
        \item ML estimate of $\Theta$, \cite{Johnston2007}
         \note{Particle Filter used to estimate states, ML of parameters based on the posterior of states and so on}
    \end{itemize}
    \item Unscented Kalman Filter \cite{Hu2009}
     \note{Technically Kalman Filters and Particle Filters are Bayesian Filters}
     \note{Essentially filtering out inconsistent parameters.}
    \end{column}

    \begin{column}{.58\textwidth}
    \begin{figure}
    \includegraphics[clip=true,trim=0cm 0cm 0cm 1cm,width=.6\textwidth]{simulated_annealing}
    \caption{Simulated Annealing can escape local minima with chaotic jumps. \cite{Dama}}
    \end{figure}
    \begin{figure}
    \includegraphics[width=\textwidth]{kalman}
    \caption{UKF: (a) Initial Belief  (b) Noisy Measurement  (c) incorporated into the belief. \cite{Thrun2005}}
    \end{figure}
    \end{column}
    \end{columns}
\end{itemize}
\end{frame}

\subsection{Particle Filter}
%frame with example
\begin{frame}{Example System Identification}
\begin{itemize}
\item Given:
    \begin{itemize}
    \item Elevation Map (1D)
    \item Ability To Measure Elevation
    \end{itemize}
\item Particle Consists of the unknowns:
    \begin{itemize}
    \item State: Location  $[0, 20]$
    \item Constant: Direction  $\{Left, Right\}$
    \end{itemize}
\end{itemize}

\begin{center}
\includegraphics[clip=true,trim=2cm 3cm 3cm 3cm,width=.7\textwidth]{setup}
\end{center}
\end{frame}

\begin{frame}{Initial Distribution}
\centering
\includegraphics[width=\textwidth]{init}\\
\end{frame}

\begin{frame}{Measurement 1}
\centering
\includegraphics[width=\textwidth]{meas1}\\
\end{frame}

\begin{frame}{Step Forward}
\centering
\includegraphics[width=\textwidth]{move1}\\
\end{frame}

\begin{frame}{Measurement 2}
\centering
\includegraphics[width=\textwidth]{meas2}\\
\end{frame}

\begin{frame}{Step Forward}
\centering
\includegraphics[width=\textwidth]{move2}\\
\end{frame}

\begin{frame}{Particle Filter Concepts}
\small
\begin{columns}
\begin{column}{.5\textwidth}
\begin{itemize}
    \item Weighting Function
    \item Resampling
    \begin{itemize}
        \item Remove Useless Particles
    \end{itemize}
    \item Regularized Resampling
    \begin{itemize}
        \item Prevent Redundant Particles
        \item Reduces Quantization Error
    \end{itemize}
\end{itemize}
\end{column}

\begin{column}{.5\textwidth}
\includegraphics[clip=true,trim=0cm 6cm 0cm 6cm,width=\textwidth]{particle_filter2}
\end{column}
\end{columns}
\end{frame}

\begin{frame}{Construction}
\begin{itemize}
    \item Mixture PDF at time $k$, with $N_p$ particles:\\
        $$P(x_k) = \sum_{i=0}^{N_p} w^i\delta(x_k - x^i_k ) dx$$
    \item Weight of Particle $i$, at time $t_k$:\\
        $$w^i_k \propto \frac{P(x^i_{0:k} | y_{0:k})}{q(x^i_{0:k} | y_{0:k})}$$
      \note{$q$ is the density/location of the particle}
      \note{$p$ is the actual probability }
    \item Incorporating a Measurement, $y_k$:\\
        $$w^i_k \propto & w^i_{k-1}P(y_k| x_k) $$
      \note{Because of the proportion, note that the weights do not
            have to integrate to 1, and can be extremely large or extremely small}
\end{itemize}
\end{frame}

\chapter{Methods}
\label{sec:Methods}
Although the particle filter  is a standard Regularized
Particle filter, as described in \cite{Arulampalam2002a}, optimizing the 
particle filter for use with FMRI data is non-trivial. 


\section{Model}
As originally written in \autoref{sec:BOLD Physiology} the state variables
for the BOLD model are as follows:
\begin{eqnarray}
\dot{s} &=& \epsilon u(t) - \frac{s}{\tau_s} - \frac{f - 1}{\tau_f} \\
\dot{f} &=& s\\
\dot{v} &=& \frac{1}{\tau_0}(f - v^\alpha)\\
\dot{q} &=& \frac{1}{\tau_0}(\frac{f(1-(1-E_0)^f)}{E_0} - \frac{q}{v^{1-1/\alpha}})
\end{eqnarray}
The original assumption regarding particle filter models (\autoref{sec:Particle Filter Model})
included noise in the update of $x$, however that is not included here.
The reason for the difference is that cloud of particles is, to some extent,
able to account for that noise. It is common, however, to model that noise
in particle filters by adding a random value to each updated state variable. 
Because the purpose of this particle filter is to learn the underlying distribution
of the static parameters, rather than precisely model the time course of the 
in the dynamic parameters ($\{s,f,v,q\}$) this noise is left out. It also helps
that detrending is applied before the particle filter and that the
BOLD model is dissipative. When no stimuli are applied, all the particles 
decay to ($\{0,1,1,1\}$). Typical particle filters 
also use this state noise as an exploratory measure; however this method is
less necessary when good priors are available.

For all the analyses  in this work, $1400$ integration points
per second were used.  Typically a step size of $0.001$ was sufficient,
however, from time to time $0.001$ can still be too high for the
BOLD model.

\section{Preprocessing}
\label{sec:Methods Preprocessing}
The normal pipeline for analyzing
FMRI involves a several preprocessing steps. The first and most important
task is motion correction. To do this, a single volume in time is chosen, and
volumes at every other time are registered to this one volume. This corrects
for motion by the patient as well as small changes in the magnetic
fields that cause the image to shift. 
In conventional statistical parametric mapping, a Gaussian smoothing
filter is applied across the image as discussed in \autoref{sec:RFT}.
After this, detrending is performed which is discussed in \autoref{sec:Detrend}.
Recall that FMRI signal levels are unit-less and though detrending is not
always necessary, the data must always be converted 
into \% difference from baseline. 
The generally accepted method is to use a high pass filter, although the
cutoff frequency is application dependent and often applied haphazardly.
Before going into the detrending used in this work, it is necessary to 
discuss the type of noise present in FMRI.

\subsection{BOLD Noise}
\label{sec:Introduction Noise}
As demonstrated in \autoref{sec:BOLD Physiology} the BOLD response has been
extensively studied and despite minor discrepancies, the cause of the BOLD 
signal is well known. However, as FMRI detects an  
aggregate signal over the space of cubic centimeters, there are
plenty of noise sources . Though local neurons act
together (i.e. around the same time), the density of neurons, the
density of capillaries, and slight differences in activation across 
a particular voxel can all lead to signal attenuation and noise. 

A particularly difficult form of noise present in FMRI is a low frequency
drift, often characterized as a Wiener process (\cite{Riera2004}). 
Though not present in all regions, as many as ten to fifteen percent
of voxels can be affected (\cite{Tanabe2002}), thus it is prevalent enough to cause significant
inference problems \cite{Smith2007}. It is still not
clear what exactly causes this noise, although one possibility is 
the temperature difference in scanner magnetic coils\cite{Smith2007}. 
It is clear that this drift signal is not solely
due to a physiological effects, given its presence in cadavers and phantoms 
\cite{Smith1999}. Interestingly, it is usually spatially correlated, and
more prevalent at interfaces between regions. Though one potential source
could be slight movement, co-registration is standard, making this unlikely. 
Regardless, the problem mandates the use of a high pass filter \cite{Smith2007}.

In order to characterize the noise, I analyzed resting state data.
During resting state, the patient is shown no images, and he is asked
to avoid movement and complex thought.  Overall though there should be 
very little activation, and thus the signal consists entirely of noise. 
Therefore resting state data is perfect for analyzing noise. 
The locations were chosen from points all around the brain, 
all in grey matter voxels. These time
series were chosen because they were representative of different types
of noise found in the resting state data.

The resting state was gathered in the exact same way as the data in 
\autoref{sec:ExperimentConfig}, except without the stimuli.

\begin{figure}
\centering
\subfigure[]{\label{fig:QQDC:A}\includegraphics[trim=6cm 1cm 6cm 1cm,width=13cm]{images/noise2_0009_29_49_9}}
\subfigure[]{\label{fig:QQDC:B}\includegraphics[trim=6cm 1cm 6cm 1cm,width=13cm]{images/noise2_0009_34_43_24}}
\subfigure[]{\label{fig:QQDC:C}\includegraphics[trim=6cm 1cm 6cm 1cm,width=13cm]{images/noise2_0009_22_38_23}}
\subfigure[]{\label{fig:QQDC:D}\includegraphics[trim=6cm 1cm 6cm 1cm,width=13cm]{images/noise2_0009_37_29_24}}

%\subfigure{\includegraphics[trim=6cm 1cm 0 0cm,width=17cm]{images/noise_0009_19-24-10.pdf}}
%\subfigure{\includegraphics[trim=6cm 1cm 0 0cm,width=17cm]{images/noise_0009_20-45-18.pdf}}
%\subfigure{\includegraphics[trim=6cm 1cm 0 0cm,width=17cm]{images/noise_0009_23-47-18.pdf}}
%\subfigure{\includegraphics[trim=6cm 1cm 0 0cm,width=17cm]{images/noise_0009_35-49-9.pdf}}

\caption{Q-Q Plots of normalized resting state data}
\label{fig:QQDC}
\end{figure}

\begin{figure}
\centering
\subfigure[]{\label{fig:QQDDelta:A}\includegraphics[trim=6cm 1cm 6cm 1cm,width=13cm]{images/noise2_0009d_29_49_9}}
\subfigure[]{\label{fig:QQDDelta:B}\includegraphics[trim=6cm 1cm 6cm 1cm,width=13cm]{images/noise2_0009d_34_43_24}}
\subfigure[]{\label{fig:QQDDelta:C}\includegraphics[trim=6cm 1cm 6cm 1cm,width=13cm]{images/noise2_0009d_22_38_23}}
\subfigure[]{\label{fig:QQDDelta:D}\includegraphics[trim=6cm 1cm 6cm 1cm,width=13cm]{images/noise2_0009d_37_29_24}}
\caption{Q-Q Plots of resting state data, using the BOLD signal changes}
\label{fig:QQDelta}
\end{figure}


\begin{figure}
\centering
\subfigure[]{\label{fig:QQs:A}\includegraphics[trim=6cm 1cm 6cm 1cm,width=13cm]{images/noise2_0009s_29_49_9}}
\subfigure[]{\label{fig:QQs:B}\includegraphics[trim=6cm 1cm 6cm 1cm,width=13cm]{images/noise2_0009s_34_43_24}}
\subfigure[]{\label{fig:QQs:C}\includegraphics[trim=6cm 1cm 6cm 1cm,width=13cm]{images/noise2_0009s_22_38_23}}
\subfigure[]{\label{fig:QQs:D}\includegraphics[trim=6cm 1cm 6cm 1cm,width=13cm]{images/noise2_0009s_37_29_24}}
\caption{Q-Q Plots of resting state data, after the de-trending}
\label{fig:QQSpline}
\end{figure}

Because most methods (including the one used in this paper)
assume the noise realizations are independent of each other, the auto-
correlation is of particular interest (which is a necessary but not
sufficient condition for independence). Gaussianity is also a common
assumption made in studies of FMRI data, though that assumption is not
needed in this work. Regardless, comparing the distribution to a Gaussian
is informative, so Q-Q plots are used to compare example data with the
Normal distribution. Additionally, in FMRI data the noise is often considered 
to be Wiener \cite{Riera2003}. Recall that a Wiener random process is
characterized by steps that are Gaussian and independent. The simulations discussed in 
\autoref{sec:Single Voxel Simulation} make use of this, 
by adding a Wiener random process to the overall signal. To determine
whether the noise is in fact Wiener, the distribution of 
the steps were plotted against a Gaussian. 

Finally, removal of the drift is often performed with a high pass filter,
so analyzed the distribution after subtracting of a spline, (see \autoref{sec:Methods Preprocessing}).

\autoref{fig:QQDC} shows the 
results with a regression line fit to the points.
Recall that in a Quartile-Quartile (Q-Q) plot, if the points plotted on the 
x-axis and the points
on the y-axis come from the same type of distribution, then all the points will
be collinear. Differences in the variance will cause the line to have a slope
other than 1, while differences in the expected value will cause the fitted line
to be shifted. In these Q-Q plots, the points are being compared to the standard
Gaussian distribution. Note that in \autoref{fig:QQDC} the points have all been 
normalized (changed to percent difference).

Note that \autoref{fig:QQDC:A} and \autoref{fig:QQDC:B}
are well described by a Gaussian process with a small autocorrelation, 
\autoref{fig:QQDC:C} and \autoref{fig:QQDC:D} are not. In particular the tails of \autoref{fig:QQDC:C}
do not seem to fit the Gaussian well. Also note the significant autocorrelation in
\autoref{fig:QQDC:C} and \autoref{fig:QQDC:D}. As expected, the noise is not strictly
Gaussian white noise.  On the other hand, the steps do conform rather
closely to the normal distribution.
As expected, most of the autocorrelation disappears for the step data. Given
that the steps seem to fit the Normal distribution, the low autocorrelation
indicates that the steps could be Independently Distributed. 
Therefore, the noise does seem to come close to a Wiener process. 

De-trending the time-series by subtracting a spline fit to the distribution
removed much of the autocorrelation present in \autoref{fig:QQDC:C} and \autoref{fig:QQDC:D},
though not perfectly. Though the distributions still do not exactly fit
the Normal, \autoref{fig:QQs:D} is much improved compared to \autoref{fig:QQDC:D}.
In all, the de-trending is effectively removing Wiener noise. 

\subsection{Detrending}
\label{sec:Detrend}
The non-stationary
aspect of a Weiner process, presumably the result of integrating some
$\nu_x$ is difficult to compensate for, and so many methods
have been developed to compensate for it. \cite{Tanabe2002} and \cite{Smith1999} have
demonstrated that this component is prevalent, and may in fact be an inherent  characteristic
of FMRI. It has been reported that in some studies as many as half the voxels 
benefited from detrending (\cite{Smith2007}). In a head to head comparison, 
\cite{Tanabe2002}, showed that in most cases subtracting off
a spline worked the best. The benefit of the spline versus wavelets, high pass 
filtering or other DC removal techniques is that the frequency response is not set.
Rather, the spline is adaptive to the input. Unfortunately no method will 
perfectly remove noise, and no method will leave the signal untouched.

The method I used to calculate the spline was picking one knot for every 20
measurements in an image. Thus a 10 minute session at a repetition time of 
2.1 seconds would have 19 knots. The knot first and last knots were each 
given half the number of samples as the rest of the knots; which were all 
located at the center of their sample group. The median of each sample group
was then taken and used as the magnitude for the group. Taking the median 
versus the mean seemed to work better, given the presence of outliers. 
There is potential to optimize the spline further using a canonical 
HRF to find resting points; however, for this to work the experiment would have
to be designed with this in mind. 

Problematically, after removing the DC component of the signal,
by definition the signal will have a median near zero. 
Unfortunately this is not the natural state of the BOLD signal. More specifically,
when the signal is inactive, the BOLD response should be at 0\% change from
the base level; activation may then increase, or for short periods decrease from this base.
Because most of the BOLD signal is above baseline, after removing the spline
the BOLD resting state will be below 0\%.  This reduces the ability of an algorithm to learn.
One method of accounting for this is to simply add a DC gain model parameter.
Like all the other model parameters, with enough measurements, a viable
parameter would fall. Yet adding another dimension increases the
complexity of the model, when the parameter is relatively easy to estimate
by visual inspection.  In this work a simpler approach was used. To determine
the DC gain I used a robust estimator of scale. The Median Absolute Deviation (MAD)
proved to be accurate in determining how much to shift the signal up
by. I tested both methods during the course of analysis, and found that the increase 
in model complexity far outweighed the slight increase in flexibility. Other
methods may work better, however the MAD worked well, 
as \autoref{fig:PreprocessedLowNoise} and \autoref{fig:PreprocessedHighNoise} show. 

\begin{equation}
y_{\text{gain}, 0:K} = 1.4826\text{median}_{i=0:K}(y_i - \text{median}(y_{0:K}))
\label{eq:mad}
\end{equation}

A serious concern when adding and subtracting arbitrary values to 
real data is whether this will create false positives. This is a legitimate
concern; however, a boosted response does not effect how well the BOLD model 
predicts the actual measurements. 

%\subsection{Linearizing Noise}
%\label{sec:Methods Delta Based Inference}
%The alternative to these sorts of low frequency manipulation is to
%go around the problem in another way. Here, I propose a 
%different method of dealing with the drift. 
%Instead of comparing the direct output of the particle filter with the direct
%measurement, the algorithm would compare the change in signal over a single TR,
%with the result of integrating the model for the same period. 
%In most signal processing cases this would be foolish, but that is because the 
%general assumption is that all noise is high frequency. Considering 
%the fact that every BOLD analysis pipeline uses a high pass filter,
%whereas low poss temporal filter are rarely applied, it makes sense
%that a derivative type method could work. The benefit of particle filters
%is that they are a robust method of inference, and I would assert 
%that the particle filter ought to be given as \emph{raw} data as possible. 
%While taking direct measurements
%without de-trending would give awful results, using the difference removes the 
%DC component and turns what is usually assumed to be a Weiner process into 
%a simple Gaussian random variable. 
%
%\begin{equation}
%\Delta y = y(t) - y(t-1) = g(x(t)) - g(x(t-1)) + \nu_y(t) - \nu_y(t-1) + \nu_d(t) - \nu(t-1)
%\label{eq:measass_delta}
%\end{equation}
%
%Even if $\nu_d$ is some other additive process, the difference will still be closer
%to I.I.D. than a Wiener process, as the autocorrelation of the $\delta y$ shows
%in \autoref{fig:QQDelta} in \autoref{sec:Introduction Noise}. 
% All the assumptions made originally
%for the particle filter still hold, and all of the parameters may be distinguished based on
%the step sizes, thus it is not unreasonable to consider matching the string of step sizes
%rather than string of direct readings. 
%
%\begin{figure}
%\label{fig:FrequencyGraphs}
%\caption{frequency response graphs, highlighting noise frequency range and signal frequency range}
%\end{figure}

\section{Particle Filter Settings}
There are quite a few options when using a particle filter; those
options will be discussed in this section.

\subsection{Prior Distribution}
\label{sec:PriorDist}
For the BOLD model described in \autoref{sec:BOLD Physiology}, several
different studies have endeavored to calculate parameters. The results
of these studies may be found in \autoref{tab:Params}, and the methods 
used for each may be found in \autoref{sec:Prior Works}. Unfortunately,
\cite{Friston2000} only studied regions deemed active by the General 
Linear Model; and most other research (including \cite{Friston2001}) used these results as 
the source for their priors. 
The one exception is \cite{Johnston2008}, which came to a extremely different
distributions. For a particle filter, the choice of a prior is
the single most important design choice. A very wide prior will require
more particles to be sufficiently dense, a very thin (low variance) prior may miss
the true parameters. Consequently, for this work it was natural
to use priors that will give results consistent with previous works, 
\cite{Friston2000}. This constrains the usefulness of the model to
areas that fall within the prior distribution, yet will allow results
to be comparable to other works. There is a significant need for better
estimates of the physiological parameters; and, while physical experiments
may not be possible, it would not be unreasonable to do a study with
exhaustive simulated annealing or hill climbing tests for multiple
regions and multiple patients.

There is an interesting anomaly with the priors found in virtually all
the works that characterized the parameters, except \cite{Johnston2008}.
The BOLD signal is universally recognized to be around $2-3\%$, maybe
reaching $5\%$ in extreme activation. Yet using the mean priors
from \cite{Friston2000}, the signal response for a $0.1$ second
impulse only reaches half a percent, as \autoref{fig:MeanResponseF}
shows.

\begin{figure}
\centering
\includegraphics[trim=6cm 3cm 6cm 3cm,width=16cm]{images/mean_response}
\caption{Response to $0.1s$ impulses with the mean parameters from \cite{Friston2000}}
\label{fig:MeanResponseF}
\end{figure}

While this could be the result of a stimulus
being too short to lead to strong activation, a similar stimulus
scheme in real data showed a much larger response than 
half a percent as well. In fact, after applying de-trending,
converting the image to percent-difference, and removing 
outliers ($ BOLD > 10\% \text{ or } BOLD < -10\%$) the total variance
across all \emph{active} voxels was still around .02, indicating
that in active voxels a signal peaking below .005 seems unlikely. 
Of course, if more restriction were placed on the outliers, its possible
this standard deviations could be brought down. 
The parameter estimates by \cite{Johnston2008} are even more 
confusing, with peaks of well below $.1\%$ (\autoref{fig:MeanResponseJ}).

\begin{figure}
\centering
\includegraphics[trim=6cm 3cm 6cm 3cm,width=16cm]{images/mean_response_johnston}
\caption{Response to $0.1s$ impulses with the mean parameters from \cite{Johnston2008}}
\label{fig:MeanResponseJ}
\end{figure}

Its likely that these differences are due to some difference in preprocessing,
although in \cite{Deneux2006} the signals were found to be peaking around
$1\%$, unlike \cite{Friston2000} which shows signals peaking at up to
$3\%$ or $4\%$. In my own tests, it seemed necessary for $\epsilon$ to
reach well over $1.5$ and $V_0$ to reach more than $.4$ to reach these
peaks; of course other methods may be equally able. 
Therefore, to account for these discrepancies, somewhat broader
distributions are used than the numbers used in \cite{Friston2000}
(which are widely used, \cite{Hu2009}). The 
priors used in the particle filter may be found in \autoref{tab:Prior}.

\begin{table}[t]
\centering
\begin{tabular}{|c || c | c | c |}
\hline 
Parameter & Distribution & $\mu$ & $\sigma$ \\
\hline
$\tau_0$ & Gamma & .98 & .25 \\
$\alpha$ & Gamma & .33 & .045\\
$E_0$    & Gamma & .34 & .03  \\
$V_0$    & Gamma & .04 & .03 \\
$\tau_s$ & Gamma & 1.54  & .25\\
$\tau_f$ & Gamma & 2.46  & .25\\
$\epsilon$ & Gamma & .7  & .6 \\
\hline
\end{tabular}
\caption{Prior distributions used in the particle filter.}
\label{tab:Prior} 
\end{table}

Note that although the mean remains the same for all the 
parameters other than $\epsilon$, the standard deviation is set
much higher to account for the disagreement between studies
(\autoref{tab:Params}). 
Because all the parameters are taken to be strictly positive, and the
standard deviations are approaching the mean, I used a gamma distribution.
This prevents the Gaussian from placing parameters in the nonsensical 
territory of negative activation, or negative time constants.

Another aspect of the prior is using enough particles to get a 
sufficiently dense approximation of the prior. For 7 dimensions,
getting a dense prior is difficult. Insufficiently
dense particles will result in inconsistent results. Of course the
processing time will scale up directly with the number of particles.
A dense initial estimate is important so that some particles land
near the solution; but as the variance decreases the number of 
particles needed decreases as well. Thus, as a heuristic, initially
the number of particles was set to 16,000, but after resampling,
the number of particles was dropped to 1,000. Typically during the 
first few measurements the variance dropped precipitously because
most particles were far from a solution.  The particles that are left are in a
much more compact location, allowing them to be estimated with 
significantly fewer particles. These numbers aren't set in stone,
and depending on the complexity of the system or desired accuracy
they could be changed; however, they seem to be the minimum that
will give consistent results.

\subsection{Resampling}
\label{sec:Resampling}
The algorithm for resampling is described in \autoref{sec:Particle Filter Resampling}.
When regularizing, the Gaussian kernel is convenient,
because it is simple to sample from and long tailed.
As discussed in \autoref{sec:Particle Filter Resampling},
as long as resampling is kept as a last resort, some over-smoothing
doesn't impair convergence. Therefore, for this work I chose a Gaussian kernel of
bandwidth equal to the original distribution's covariance. Obviously this will
apply a rather large amount of smoothing to the distribution; however, on average
resampling is only applied every 20 to 30 measurements, and because randomization
is being applied to model updates this gives the filter some mobility. 

Re-sampling is a not strictly necessary, but increases the effectiveness
of the particle filter by adjusting the support to emphasize areas
of higher probability. Re-sampling is slow because it requires re-drawing
all the particles. It also closes off avenues of investigation, and is
designed to over-smooth to prevent overly thinning the support. For all these
reasons, resampling was only performed when the $N_{eff}$ dropped below
50 (for 1000 particles).  As a measure against sharp drops in the $N_{eff}$ 
caused by a large spike in error, resampling was only performed when 
two consecutive low ($<50$) $N_{eff}$'s were found. 

\subsection{Choosing $P(y_k | x_k)$}
\label{sec:Methods Weighting Function}
The choice of $P(y_k | x_k)$ is the second most important design decision, behind 
the prior. While the conventional choice for an unknown distribution is the 
Gaussian, there are reasons why it may not be the best in this case.  
As noted in \autoref{sec:Introduction Noise}, the noise is not strictly Gaussian,
nor is it strictly Wiener. As with any unknown noise however, it is necessary 
to make some assumption. If the weighting function ($P(y_k | x_k)$) exactly
matches the measurement error, then the ideal particle filter will result.
Particles with $x_k$'s that repeatedly estimate $y_k$ with large residual 
will quickly have weights near 0. Thus, a weighting function that
exactly matches $P(Y(t) | X(t))$ will easily, and correctly throw out incorrect
particles.  The cost of choosing an overly broad distribution for this
function is slow convergence.  On the other hand, an overly thin 
distribution will lead to particle deprivation (all particles
being zero-weighted).  I tested several weighting
functions: in addition to the Gaussian I also tested the Laplace and Cauchy
distributions, both of which have much wider tales than the Gaussian. 
Wider tailed distributions don't down-weight
particles as fast; and converge more slowly (and perhaps more accurately). 
The Laplace distribution also has the
benefit of a non-zero slope at the origin; which means that even
it will distinguish between particles even near the origin.

After trial and error, for this work I chose a zero-mean Gaussian with standard deviation 
of $.005$. While I made some attempts to automatically set the standard 
deviation, results were often unpredictable. If the weight function and scale aren't
fixed across voxels, very noisy time series with no actual signal 
converged to nonsensical results. 
In the future, it may be possible to set the standard deviation by
taking a small sample from resting data and using the sample standard deviation.
Since this is the first attempt at using particle filters for modeling the 
BOLD model, in this work I set the standard deviation manually at $.005$,
because it gave the best consistency. 

\subsection{Runtime}
The run-time for a single voxel depends on the several factors. First, the
overall length of the signal being analyzed. For 1000 measurements it takes
about 6 minutes. On the other hand, in real circumstances the
length is only around 150 measurements and takes around 40 seconds (for 1000 
particles, 1500 integration points and a Quad Core CPU). The size of 
local linearization steps are also crucial although going above $0.001$ seconds
per step is not recommended. In most cases millisecond resolution
is fine; however, when generating simulated data I found that at times it was
still not enough every once. This is problematic in the actual particle
filter since, given the large number of simultaneous integrations taking 
place, its probable that a few particles will fail and be unfairly thrown away.
To prevent such events, 1500 integration points per second were used throughout
the tests. 

Another crucial factor for run time is how long before the first re-sampling 
occurs. Because the prior is represented initially with significantly more
particles, if for some reason the effective
number of particles stays high, resampling could take a long time to occur.
For this reason, rather than allowing the particle filter to continue on 
with this large number of particles, after 20 seconds have passed the
algorithm forces resampling. The choice of 20 seconds is 
arbitrary, but at the very least it gives a more optimized version of the
prior. 


\chapter{Results}
\section{Single-Voxel Simulation}
The results 

\section{Single-Voxel Analysis}
This section discusses the results when the particle filter was
applied on a single voxel. The parameters are the same as
those used later for entire image analysis; however, the results
are more in-depth. 

The run-time for a single voxel depends on the several factors. First, the
overall length of the signal being analyzed. For 1000 measurements it takes
about 10 to 15 minutes. On the other hand, in real circumstances the
length is only around 150 measurements. The number of  integration
can certainly make a large difference, however dropping below 1000 (.001 seconds)
is definitely not recommended. 

For a period I considered 1000 to be a
fine number; however when generating simulated data I found that every once
in a while 1000 was not enough. This is problematic in the actual particle
filter since, given the large number of simultaneous integrations taking 
place, its likely that a few particles will fail and be weighted at 0 because
of this problem. Additionally, because the typical case where a failure would
occur is at fast moving times/parameters the particles all tend to fail together.
The result is particle deprivation - no particles with non-zero weights remain.
The other possible outcome is that low time constant particles get pruned resulting
in excessively smoothed estimates for the time series'. Its possible that a
kind of stop-gap measure could be put into place; wherein particles that are
about to be set to NaN are integrated again with finer grained steps. However
many times the non-real results don't occur until several time steps after the 
numbers get strange. So for instance, the timestep was too long, allowing 
$f$ to go negative, resulting in extremely large values of $q$. There are many
different ways where this sort of event can occur, and unfortunately sometimes
there is no way to get back to before the state starting going out of control.

Another crucial factor to run time is how long before the first re-sampling 
occurs. Because the prior is represented initially with significantly more
particles, if the model fits very well, or for some reason the effective
number of particles stays high, resampling could take a long time to occur.
When this happens the particles filter can take a factor of 10 longer to run.
However, if the particle count isn't initially set high, there is a much larger
chance of particle deprivation occurring. Since there is no real way to know
how long it will take to resmaple the first time, there is little the 
algorithm can do to fix this. On the other end of the specture, if the time
series don't match the model at all, particle deprivation will occur extremely
quickly, and even the recovery technique discussed in \autoref{sec:Resampling}
won't help. The upshot of this is that the particle filter is able to 
identify these sections very quickly, and thus not waste much time there.
Of course, if a fat-tailed distribution is used for the weighting function,
or the standard deviation of a Gaussian weighting function is very large,
the particle filter will simply converge to meaningless values.

\begin{figure}
\caption{Particle Filter converging to values that make little sense,
because the voxel did not correlate with the input in any known way}
\end{figure}

\section{Weighting Function Comparison}
\label{sec:Results Weights}

\section{Single Time-Series Simulation}

Graphs: 

For simulated data, single timeseries:

For \{delta, DC/Spline\}, \{exponential, gaussian, cauchy\}, \{biased, unbiased initial\},
\{100, 500, 1000\} particles
\begin{enumerate}
\item Ground truth vs. Estimated signal during particle filter run
\item Ground truth vs. Estimated signal with final parameter set
\item True Parameters vs. Final Parameter Sets
\item Variance of final parameters when faced with same ground truth, different noise
\item MSE of (a new timeseries based on X(t) vs. ground truth) for all t
\item Estimator Variance based on different noise runs
\item Final Particle Distribution
\end{enumerate}

For Simulated Data, Full Volume:

%note to self, epsilon should probably be uniform between 0 and something
\section{Simulated Volume}
\begin{enumerate}
\item Parameter Map 
\item Error map of parameters
\item Histogram of \%errors between parameters
\item Activation Map based on a single region with high $\epsilon$, compared with linear
\end{enumerate}

Final parameter distribution among active regions.
Q-Q plots?

\section{FMRI Data}
....

image comparing epsilon-map with GLM activation map



\section{Conclusion}
\begin{frame}{Conclusion}
\begin{itemize}
    \item Summary:
    \begin{itemize}
    \item BOLD Parameters Ill-Defined
    \item Particle Filter Capable of good parameter BOLD estimate with 1000 particles
    \item Mutual Information performs well as estimate of Quality
    \end{itemize}
    \item Future Works
    \begin{itemize}
        \item Further limitations should be placed on priors, Deneux 
            et al. \cite{Deneux2006} shows that parameters could imposed.
        \item Analysis of joint parameter distribution for populations
        \item Real Time analysis possible for multiple Voxels, similar to
            De Charms et al. \cite{DeCharms2005}

    \end{itemize}
\end{itemize}
\end{frame}

\begin{frame}{Questions?}
\end{frame}

%% All of the following is optional and typically not needed. 
%\appendix
%\section<presentation>*{\appendixname}
%\subsection<presentation>*{For Further Reading}

\appendix

\section{Backup}
\begin{frame}{Balloon Flowchart}
\begin{figure}
%\includegraphics[width=\textwidth]{backup_balloon}
\end{figure}
\end{frame}

\begin{frame}{SPM vs. Mutual Information Map, SPM}
\begin{figure}
\setcounter{subfigure}{0}
\subfigure[SPM Results]{\includegraphics[width=.8\textwidth]{images/spm_hm}}
\subfigure{\includegraphics[scale=.5]{images/scale1}}
\note{SPM results. Units of activation are in Student's $t$-scores; higher 
        indicates higher assurance that the signal cannot have occurred 
        through noise alone.}
\end{figure}
\end{frame}

\begin{frame}{SPM vs. Mutual Information Map, $M.I. > .15$}
\begin{figure}
\setcounter{subfigure}{1}
\subfigure[Particle Filter Results]{\includegraphics[width=.8\textwidth]{hm_mi_strict}}
\subfigure{\includegraphics[width=.1\textwidth]{scale4}}
\note{Particle filter results. Units of activation are in mutual information;
    higher indicates more assured activation.}
\end{figure}
\end{frame}

\begin{frame}{SPM vs. Mutual Information Map, $M.I. > .1$}
\begin{figure}
\setcounter{subfigure}{2}
\subfigure[Particle Filter Results]{\includegraphics[width=.8\textwidth]{mi_hm}}
\subfigure{\includegraphics[width=.1\textwidth]{scale6}}
\note{Particle filter results. Units of activation are in mutual information;
    higher indicates more assured activation.}
\end{figure}
\end{frame}

\begin{frame}{1: 37-14-7}
\setcounter{subfigure}{0}
\begin{figure}
\centering
\subfigure[Particle Filter]{\label{fig:comp1pfilter} \includegraphics[clip=true,trim=5cm 1cm 4cm 1cm,width=.4\textwidth]{images/1_pfilter_37_14_7}}
\subfigure[SPM]{\label{fig:comp1spm} \includegraphics[clip=true,trim=5cm 1cm 4cm 1cm,width=.4\textwidth]{images/1_spm_37_14_7}}
\caption{Section 1, Estimated vs. Actual BOLD response. $t$-Score: $10.71$, Mutual Information: $0.33$, Residual: $0.72$.}
\label{fig:comp1}
\end{figure}
\end{frame}

\begin{frame}{2: 34-12-7}
\setcounter{subfigure}{0}
\begin{figure}
\centering
\subfigure[Particle Filter]{\label{fig:comp2pfilter} \includegraphics[clip=true,trim=5cm 1cm 4cm 1cm,width=.4\textwidth]{images/2_pfilter_34_12_7}}
\subfigure[SPM]{\label{fig:comp2spm} \includegraphics[clip=true,trim=5cm 1cm 4cm 1cm,width=.4\textwidth]{images/2_spm_34_12_7}}
\caption{Section 2, Estimated vs. Actual BOLD response. $t$-Score: $6.97$, Mutual Information: $0.04$, Residual: $1.02$.}
\label{fig:comp2}
\end{figure}
\end{frame}

\begin{frame}{3: 23-21-7}
\setcounter{subfigure}{0}
\begin{figure}
\centering
\subfigure[Particle Filter]{\label{fig:comp3pfilter} \includegraphics[clip=true,trim=5cm 1cm 4cm 1cm,width=.4\textwidth]{images/3_pfilter_23_21_7}}
\subfigure[SPM]{\label{fig:comp3spm} \includegraphics[clip=true,trim=5cm 1cm 4cm 1cm,width=.4\textwidth]{images/3_spm_23_21_7}}
\caption{Section 3, Estimated vs. Actual BOLD response. $t$-Score: $2.85$, Mutual Information: $-0.03$, Residual: $0.81$.}
\label{fig:comp3}
\end{figure}
\end{frame}

\begin{frame}{4: 33-40-4}
\setcounter{subfigure}{0}
\begin{figure}
\centering
\subfigure[Particle Filter]{\label{fig:comp4pfilter} \includegraphics[clip=true,trim=5cm 1cm 4cm 1cm,width=.4\textwidth]{images/4_pfilter_26_15_7}}
\subfigure[SPM]{\label{fig:comp4spm} \includegraphics[clip=true,trim=5cm 1cm 4cm 1cm,width=.4\textwidth]{images/4_spm_26_15_7}}
\caption{Section 4, Estimated vs. Actual BOLD response. $t$-Score: $0.50$, Mutual Information: $0.06$, Residual: $0.95$. }
\label{fig:comp4}
\end{figure}
\end{frame}

\begin{frame}{5: 29-9-13}
\setcounter{subfigure}{0}
\begin{figure}
\centering
\subfigure[Particle Filter]{\label{fig:comp5pfilter} \includegraphics[clip=true,trim=5cm 1cm 4cm 1cm,width=.4\textwidth]{images/5_pfilter_29_9_13}}
\subfigure[SPM]{\label{fig:comp5spm} \includegraphics[clip=true,trim=5cm 1cm 4cm 1cm,width=.4\textwidth]{images/5_spm_29_9_13}}
\caption{Section 5, Estimated vs. Actual BOLD Response. $t$-Score: $4.17$, Mutual Information: $0.02$, Residual: $1.14$.}
%\caption{Section 5, Below threshold in both particle filter checks, but above threshold in SPM. Mutual Information of $0.0212822$, $t$-Value
%of $4.17399$ and $MSE$ of $1.14171$.}
\label{fig:comp5}
\end{figure}
\end{frame}

\begin{frame}{6: 36-17-19}
\setcounter{subfigure}{0}
\begin{figure}
\centering
\subfigure[Particle Filter]{\includegraphics[clip=true,trim=5cm 1cm 4cm 1cm,width=.4\textwidth]{images/6_pfilter_36_17_19}}
\subfigure[SPM]{\includegraphics[clip=true,trim=5cm 1cm 4cm 1cm,width=.4\textwidth]{images/6_spm_36_17_19}}
\caption{Section 6, Estimated vs. Actual BOLD Response. $t$-Score: $2.49$, Mutual Information: $.34$, Residual: $0.78$.}
%\caption{Section 6, MI of $0.335504$, T Value: $2.49154$, normalized error: $0.783348$ Not visible in SPM}
\end{figure}
\end{frame}

\begin{frame}{Particle Filter Results: Histogram}
\begin{figure}
\centering
\includegraphics[clip=truew,trim=8cm 4cm 8cm 4cm,width=.8\textwidth]{realhist}
\note{Histogram of parameters in active regions ($M.I. > .15$).}
\end{figure}
\end{frame}

\section{Bibliography}
\begin{frame}[allowframebreaks]
  \bibliographystyle{plain}
  \bibliography{../library}
\end{frame}


\end{document}



