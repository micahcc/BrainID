%Micah Chambers

\documentclass{beamer}
%\usepackage[orientation=landscape,size=custom,width=16,height=10,scale=0.5]{beamerposter} 

%\usepackage{pgfpages}
%\setbeameroption{show notes on second screen}

\mode<presentation>
{
  \usetheme{Antibes}
  % or ...

  %\setbeamercovered{transparent}
  % or whatever (possibly just delete it)
}

%\usefonttheme[onlylarge]{structurebold}
%\usecolortheme{crane}
%\setbeamerfont*{frametitle}{size=\normalsize,series=\bfseries}
%\setbeamertemplate{navigation symbols}{}
%\setbeamercovered{transparent}


\usepackage[pdftex]{graphicx}
\graphicspath{{pdf/}{png/}}
%\usepackage{times}
%\usepackage[T1]{fontenc}
%\usepackage[small]{caption}
\usepackage{algorithmic}

\title{Full Brain Blood-Oxygen-Level-Dependent Signal \\
Parameter Estimation using Sequential Monte Carlo Methods}

%\subtitle{}

\author{Micah Chambers}
\institute{Virginia Tech Bioimaging Systems Lab}

\subject{Medical Imaging}

% If you have a file called "university-logo-filename.xxx", where xxx
% is a graphic format that can be processed by latex or pdflatex,
% resp., then you can add a logo as follows:

%\pgfdeclareimage[width=1.5cm]{university-logo}{logo}
\logo{\includegraphics[width=1.5cm]{logo}}


% If you wish to uncover everything in a step-wise fashion, uncomment
% the following command: 

%\beamerdefaultoverlayspecification{<+->}


\begin{document}
\begin{frame}
  \titlepage
\end{frame}

\begin{frame}{Outline}
  \tableofcontents
  % You might wish to add the option [pausesections]
\end{frame}

\section{Introduction}
\begin{frame}{Balloon Model}
\begin{figure}
\includegraphics[width=10cm]{model}
\caption{
    \tiny
    \cite{Riera2003}
}
\end{figure}
\note{Rates range from below 1 Hz to 5 to 6 Hundred Hz}
\note{Change in average firing rates increase metabolism and sympathetic response causes
        increased inflow of oxygenated blood.}
\note{In the version of the Balloon model I used, the flow is locked to metabolism.}
\note{According the balloon model this increases the local blood volume, and the outflow pressure.}
\note{So the signal is altered by changes in the ratio of Deoxyhemoglobin and Oxygenated Hemoglobin,
        which are a function of Volume and Oxygenation.}
\end{frame}

\begin{frame}{Time Changing Parameters}
    \note{Should include a picture of the different parts of the BOLD signal}
\end{frame}

\begin{frame}{BOLD State Equations}
  \begin{itemize}
    \item Flow Inducing Signal
    $$\dot{s}(t) = \epsilon u(t) - s(t)/\tau_s - (f(t)/\tau_f - 1)$$
    \item Normalized Cerebral Blood Inflow:
    $$\dot{f}(t) = s $$
    \item Normalized Cerebral Blood Volume:
    $$\dot{v}(t) = (1/\tau_0)( f(t) - v(t) ^ {1/\alpha}) $$
    \item Normalized Deoxyhaemoglobin Content:\\
    $$\dot{q}(t) = \frac{1}{\tau_0}\left(\frac{f(t)(1-(1-E_0)^{1/f(t)})}{E_0} -
            \frac{q(t)}{v(t)^{1-1/\alpha}}\right)$$
    \item Hemodynamic Response - BOLD Signal
    $$y(t) = V_0(a_1( 1 - Q(t)) - a_2(1 - V(t)))$$
  \end{itemize}
  \note{System is dissipative:}
  \note{If no input is provided, it will rather quickly decay.}
  \note{Constant input leads to steady state response.}
  \note{First term in $\dot{q}$ is the metabolism term.}
  \note{Outflow is the $v^{1/\alpha}$ term}
  \note{Significant nonlinearity.}
\end{frame}

\section{Regression}
\begin{frame}{Approximate Method}
  \begin{itemize}
    \item 2nd Order Volterra Kernel \cite{Friston2000}
    \note{Quadratic Approximation}
    \begin{itemize}
        \item Quadratic Convolution used to approximate Jacobian Matrix.
        \note{Because of the nonlinearities in the BOLD model, step size for
                integrating would otherwise be extremely small}
        \item Volterra approximation quality is not known.
    \end{itemize}
    \note{Volterra Approximation depends on sparsity etc}
    {\footnotesize
    $$y(t) = k_0 + \int_{-\infty}^{\infty} k_1(s_1) x(t-s_1) ds_1
        + \int_{-\infty}^{\infty} k_2(s_1,s_2) x(t-s_1)x(t-s_2) ds_1 ds_2$$
        \note{Its possible this isn't even knowable}
    }
    \item Canonical Hemodynamic Response Function
    \note{Linear Approximation}
    \begin{itemize}
        \item No Parameters Estimated
        \item Maximum Likelihood Possible
        \item Inflexible - even to changes in onset time
    \end{itemize}
    \begin{center}
    \includegraphics[clip=true,trim=4.88cm 0cm 5cm 1.5cm,height=2.7cm,width=8cm]{HRF}
    \end{center}
    
  \end{itemize}
\end{frame}

\begin{frame}{Nonlinear Modeling}
\begin{itemize}
    \item Local Linearization filter \cite{Riera2003}
     \note{Jacobian of the $\dot{x}$ is available, making gradient descent possible}
    \item Genetic Algorithms and Simulated Annealing \cite{Vakorin2007}
     \note{Genetic Algorithms were used to optimize spike points, simulated annealing for the parameters}
    \item Particle Filters to estimate $P(X_t | \Theta, Y_T)$ \cite{Murray2009}
    \begin{itemize}
        \item ML estimate of $\Theta$ based on $P(X_t | \Theta, Y_t)$ \cite{Johnston2007}
         \note{Particle Filter used to estimate states, ML of parameters based on the posterior of states and so on}
    \end{itemize}
    \item Unscented Kalman Filter to estimate parameters \cite{Hu2009}
    \item Direct Particle Filter, used in this work
     \note{Technically Kalman Filters and Particle Filters are Bayesian Filters}
     \note{Essentially filtering out inconsistent parameters.}
\end{itemize}
\end{frame}

%\begin{frame}{BOLD Signal Properties}
%  \begin{itemize}
%    \item Exact variables and parameters are unknown and are
%        difficult to calculate.
%    \item Significant Amount of Lag between activation
%        and a measurable output - can be as much as 8 seconds.
%    \item Slow Temporal Resolution
%    \item Noise characterized by brownian motion, which clashes with low 
%        frequency elements.
%    \begin{center}
%    \includegraphics[width=10cm,height=4cm]{signal_fft.pdf}
%    \end{center}
%  \end{itemize}
%\end{frame}
%
%\begin{frame}{Preprocessing}
%  \begin{itemize}
%    \item Low Pass Filter (Gaussian Filter, not recommended)
%    \item Drift Removal (not always performed)
%    \begin{itemize}
%        \item High Pass Filter
%        \item Linear 
%        \item Quadratic
%        \item Wavelet 
%        \item Spline (Which I am using)
%    \end{itemize}
%  \end{itemize}
%  \begin{figure}
%    \includegraphics[width=10cm]{detrending_effect.png}
%    \caption{
%        \tiny
%        \cite{detrending}
%    }
%  \end{figure}
%\end{frame}
%
%\section{Statistical Parametric Mapping}
%\begin{frame}{Method}
%  \begin{figure}
%    \includegraphics[width=11cm]{glm_pipeline.png}
%    \caption{
%        \tiny
%        \cite{spm_pipeline}
%    }
%  \end{figure}
%\end{frame}
%
%\begin{frame}{Limitations}
%\begin{columns}
%  \column{2.5in}
%  \begin{itemize}
%    \item Linear, for a signal which is known to be nonlinear
%    \item Essentially the weighted sum of a set of "expected" responses.
%    \item Parametric
%    \begin{itemize}
%        \item Forced to make assumptions about underlying distributions
%        \item No time-scaling.
%    \end{itemize}
%  \end{itemize}
%  \column{3in}
%  \begin{figure}
%    \includegraphics[width=2.5in]{glm.png}
%    \caption{
%        \tiny
%        \cite{spm_pipeline}
%    }
%  \end{figure}
%\end{columns}
%\end{frame}
%
%\section{Nonlinear Regression}
%
%
%\section{Parameter Identification}
%\begin{frame}{Particle Filters}
%\begin{itemize}
%    \item Non-parametric, no assumptions are violated
%    \item Model based, fit parameters to input, constrained by physical variables
%    \item Fits a mixture PDF to the posterior of all parameters
%    \item Non-trivial computation cost
%    \item I use a Regularized Particle Filter
%    \begin{enumerate}
%        \item Regularized Re-sampling prevents particles from de-generating into a 
%            small number of unique particles
%        \item Allows distributions to move more freely
%    \end{enumerate}
%\end{itemize}
%\end{frame}
%
%\begin{frame}{Particle Filter}
%    \begin{columns}
%    \column{1.8in}
%    \begin{itemize}
%        \item $S_t = \{p_{0,t}, ... , p_{N,t}\}$, the set of particles
%        \item $w_{i,t}$, weight of particle $p_{i,t}$
%        \item $y_t$, measurement at time $t$, there is not a $y_t$ for every $t$.
%        \item $f(p_{i,t},y_t)$, weighting function 
%        \item $s(p_{i,t})$, step function 
%    \end{itemize}
%    \column{3.2in}
%    \begin{algorithmic}
%     \STATE Draw $S_0$ from prior distribution 
%     \FOR{$t = 0:t_{step}:t_{end}$}
%        \FOR{each $p_{i,t-1} \in S_{t-1}$}
%           \STATE $p_{i,t} = s(p_{i,t-1})$ 
%           \IF{There is a measurement at time $t$}
%               \FOR{every $p_{i,t}$}
%                    \STATE $w_{i,t} = w_{i,t-1}f(p_{i,t},y_t)$
%               \ENDFOR 
%               \STATE Resample if weights are unevenly distributed\\
%           \ENDIF
%        \ENDFOR
%     \ENDFOR
%%                    \item If a weights are not evenly distributed enough, then resample
%     \end{algorithmic}
%     \end{columns}
%\end{frame}
%
%\subsection{Single-Region}
%\begin{frame}{Single Timeseries Results}
%  \begin{figure}
%    \includegraphics[height=3in]{noise.png}
%  \end{figure}
%\end{frame}
%
%\begin{frame}{Single Timeseries Results, Measurement Convergence}
%  \begin{figure}
%    \includegraphics[height=3in]{bold.png}
%  \end{figure}
%\end{frame}
%
%\begin{frame}{Single Timeseries Results, State Convergence}
%  \begin{figure}
%    \includegraphics[height=3in]{state.png}
%  \end{figure}
%\end{frame}
%
%\begin{frame}{Factors Affecting Convergence}
%    \begin{enumerate}
%        \item Weighting function
%            \begin{itemize}
%                \item Needs to be continuous and defined for any input, should go to 0
%                \item Too wide a weighting function results in under-sensitivity, 
%                            slow or no convergence
%                \item Too thin a weighting function reduces robustness to noise
%            \end{itemize}
%        \item How often re-sampling is done, re-sampling should be minimized
%            \begin{itemize}
%                \item Stratified Resampling can result in truncated tails on posterior
%                \item Regularized Resampling can result in reduced robustness to noise 
%            \end{itemize}
%        \item Number of particles
%            \begin{itemize}
%                \item More particles give higher fidelity of posterior
%            \end{itemize}
%    \end{enumerate}
%\end{frame}
%
%\subsection{Multi-Region}
%\begin{frame}{Parameter Map Generation/Simulation}
%\begin{columns}
%  \column{2.5in}
%  \begin{enumerate}
%    \item Generate a parameter map, with a set of parameters for each voxel
%    \item Simulate every set of parameters, and use as input to possum
%    \item Perform preprocessing (de-trend and normalize)
%    \item Run particle filter on every grey matter voxel in image, generating a new
%            parameter map
%  \end{enumerate}
%  \column{3in}
%  \begin{figure}
%    \includegraphics[width=2.8in]{equation.png}
%    \caption{
%    }
%  \end{figure}
%\end{columns}
%\end{frame}
%
%\begin{frame}{Simulation Results, $\tau_0$}
%  \begin{figure}
%    \includegraphics[height=3in]{results.png}
%    \caption{Percent Difference from "Actual"}
%  \end{figure}
%\end{frame}
%
%\begin{frame}{Simulation Results, multiple}
%  \begin{figure}
%    \includegraphics[height=3in]{results2.png}
%    \caption{Percent Difference from "Actual"}
%  \end{figure}
%\end{frame}
%
%\section{Conclusion}
%
%% All of the following is optional and typically not needed. 
%\appendix
%\section<presentation>*{\appendixname}
%\subsection<presentation>*{For Further Reading}

\begin{frame}[allowframebreaks]
  \bibliographystyle{plain}
  \bibliography{../library}
\end{frame}

\end{document}



